\documentclass{book}
%nerd stuff here
\pdfminorversion=7
\pdfsuppresswarningpagegroup=1
% Languages support
\usepackage[utf8]{inputenc}
\usepackage[T2A]{fontenc}
\usepackage[english,russian]{babel}
% Some fancy symbols
\usepackage{textcomp}
\usepackage{stmaryrd}
% Math packages
\usepackage{amsmath, amssymb, amsthm, amsfonts, mathrsfs, dsfont, mathtools}
\usepackage{cancel}
% Bold math
\usepackage{bm}
% Resizing
%\usepackage[left=2cm,right=2cm,top=2cm,bottom=2cm]{geometry}
% Optional font for not math-based subjects
%\usepackage{cmbright}

\author{Коченюк Анатолий}
\title{Алгоритмы и структуры данных}

\usepackage{url}
% Fancier tables and lists
\usepackage{booktabs}
\usepackage{enumitem}
% Don't indent paragraphs, leave some space between them
\usepackage{parskip}
% Hide page number when page is empty
\usepackage{emptypage}
\usepackage{subcaption}
\usepackage{multicol}
\usepackage{xcolor}
% Some shortcuts
\newcommand\N{\ensuremath{\mathbb{N}}}
\newcommand\R{\ensuremath{\mathbb{R}}}
\newcommand\Z{\ensuremath{\mathbb{Z}}}
\renewcommand\O{\ensuremath{\emptyset}}
\newcommand\Q{\ensuremath{\mathbb{Q}}}
\renewcommand\C{\ensuremath{\mathbb{C}}}
\newcommand{\p}[1]{#1^{\prime}}
\newcommand{\pp}[1]{#1^{\prime\prime}}
% Easily typeset systems of equations (French package) [like cases, but it aligns everything]
\usepackage{systeme}
\usepackage{lipsum}
% limits are put below (optional for int)
\let\svlim\lim\def\lim{\svlim\limits}
%\let\svlim\int\def\int{\svlim\limits}
% Command for short corrections
% Usage: 1+1=\correct{3}{2}
\definecolor{correct}{HTML}{009900}
\newcommand\correct[2]{\ensuremath{\:}{\color{red}{#1}}\ensuremath{\to }{\color{correct}{#2}}\ensuremath{\:}}
\newcommand\green[1]{{\color{correct}{#1}}}
% Hide parts
\newcommand\hide[1]{}
% si unitx
\usepackage{siunitx}
\sisetup{locale = FR}
% Environments
% For box around Definition, Theorem, \ldots
\usepackage{mdframed}
\mdfsetup{skipabove=1em,skipbelow=0em}
\theoremstyle{definition}
\newmdtheoremenv[nobreak=true]{definition}{Определение}
\newmdtheoremenv[nobreak=true]{theorem}{Теорема}
\newmdtheoremenv[nobreak=true]{lemma}{Лемма}
\newmdtheoremenv[nobreak=true]{problem}{Задача}
\newmdtheoremenv[nobreak=true]{property}{Свойство}
\newmdtheoremenv[nobreak=true]{statement}{Утверждение}
\newmdtheoremenv[nobreak=true]{corollary}{Следствие}
\newtheorem*{note}{Замечание}
\newtheorem*{example}{Пример}
\renewcommand\qedsymbol{$\blacksquare$}
% Fix some spacing
% http://tex.stackexchange.com/questions/22119/how-can-i-change-the-spacing-before-theorems-with-amsthm
\makeatletter
\def\thm@space@setup{%
  \thm@preskip=\parskip \thm@postskip=0pt
}
\usepackage{xifthen}
\def\testdateparts#1{\dateparts#1\relax}
\def\dateparts#1 #2 #3 #4 #5\relax{
    \marginpar{\small\textsf{\mbox{#1 #2 #3 #5}}}
}

\def\@lecture{}%
\newcommand{\lecture}[3]{
    \ifthenelse{\isempty{#3}}{%
        \def\@lecture{Lecture #1}%
    }{%
        \def\@lecture{Lecture #1: #3}%
    }%
    \subsection*{\@lecture}
    \marginpar{\small\textsf{\mbox{#2}}}
}
% Todonotes and inline notes in fancy boxes
\usepackage{todonotes}
\usepackage{tcolorbox}

% Make boxes breakable
\tcbuselibrary{breakable}
\newenvironment{correction}{\begin{tcolorbox}[
    arc=0mm,
    colback=white,
    colframe=green!60!black,
    title=Correction,
    fonttitle=\sffamily,
    breakable
]}{\end{tcolorbox}}
% These are the fancy headers
\usepackage{fancyhdr}
\pagestyle{fancy}

% LE: left even
% RO: right odd
% CE, CO: center even, center odd
% My name for when I print my lecture notes to use for an open book exam.
% \fancyhead[LE,RO]{Gilles Castel}

\fancyhead[RO,LE]{\@lecture} % Right odd,  Left even
\fancyhead[RE,LO]{}          % Right even, Left odd

\fancyfoot[RO,LE]{\thepage}  % Right odd,something additional 1  Left even
\fancyfoot[RE,LO]{}          % Right even, Left odd
\fancyfoot[C]{\leftmark}     % Center

\usepackage{import}
\usepackage{xifthen}
\usepackage{pdfpages}
\usepackage{transparent}
\newcommand{\incfig}[1]{%
    \def\svgwidth{\columnwidth}
    \import{./figures/}{#1.pdf_tex}
}
\usepackage{tikz}
\begin{document}
    \maketitle
    \section{Введение}
    курс будет идти 4 семестра. 

    1 лекций + 1 практика в неделю

    баллы: практика -- выходишь к доске и делаешь задание.

    практика -- до 30 баллов

    0-25 -- по 5 баллов
    25-40 -- по 3 балла
    40+ -- по 1 баллу

    лабораторные: 50 баллов

    экзамен: в каком-то виде будет. до 20 баллов.

    \chapter{I курс}
    \section{Алгоритмы}

    Алгоритм: входные данные $\to $ \tikz \draw (0,0) -- (0,1) -- (1,1) -- (1,0) -- cycle; $\to $ выходные данные

    входной массив $a[0\ldots n-1]$, выходная сумма $\sum a_i$

    \begin{verbatim}
        S = 0           \\ 1
        for i = 0 .. n-1\\ 1+2n
            S+=a[i]     \\ 3n

        print(S)        \\ 1

    \end{verbatim}

    Модель вычислений.

    RAM - модель. симулирует ПК. За единицу времени можно достать/положить в любое место памяти.

    Время работы (число операций)
    В примере выше  $T(n) = 3+5n$

    мотивация: 3 становится мальньким, а 5 -- не свойство алгоритма

    $T(n) = O(n)$

    $f(n) = O(g(n)) \iff \exists n_0, c\quad \forall n\geqslant n_0\quad f(n)\leqslant c\cdot g(n)$

    $n_0 = 4, c = 6\quad 3+5n\leqslant 6n, n\geqslant 4\quad 3\leqslant n$

    $f(n) = \Omega (g(n)) \iff --||-- f(n) \geqslant c g(n)$

    $3+5n = \Omega(n)\quad n_0 =1, c=1$

    $3+5n \geqslant n, n\geqslant 1$

    $T(n) = O(n), T(n) = \Omega(n) \iff T(n) = \Theta(n)$

    \begin{verbatim}
        for i = 0 .. n-1
            for j = 0 .. n-1
        \end{verbatim}
        -- $O(n^2)$

    \begin{verbatim}
    
        for i = 0 .. n-1
            for j = 0 .. i-1
    \end{verbatim}

    $\sum_{i=0}^{n-1} i =\sum  \frac{n\cdot (n-1)}{2} = \Theta(n^2)$

    \begin{verbatim}
        i=1
        while i\cdot i<n
            i++
        i=1
        while i < n
            i=i\cdot \cdot 2
    \end{verbatim}

    $O(\sqrt{n}), O(\ln n) $

    \begin{verbatim}
        f(n):
            if n=0
                ..
            else
                f(n-1)
    \end{verbatim}

    $n$ рекурсивных вызовов $O(n)$


    \begin{verbatim}
        f(n):
            if n=0
                ..
            else
                f(n/2)
                f(n/2)
    \end{verbatim}

    $2^{\ln  n} = n$

    если добавить третий вызов: $2^{\log_2 n} = n^{\log_2 3}$

    \section{Сортировки}
    \subsection{Сортировка вставками}

    Берём массив, идём слева направо: берём очередной элемент и двигаем в влево, пока он не упрётся

    \begin{verbatim}
        for i = 0 .. n-1
            j=i
            while j>0 and a[j]<a[j-1]
                swap(a[j-1], a[j])
                j--
    \end{verbatim}

    Докажем, что алгоритм работает. по индукции. Если часть отсортирована и мы рассматриваем новый элемент, то он будет двигаться, пока не вставиться на своё место и массив снова будет отсортированным.

    Если массив отсортирован $(1,2,\ldots, n)$ -- $O(n)$

    Если нет(n, n-1, \ldots, 1), то $O(n^2)$

    Рассматривать мы дальше будем худшие случаи.

    \subsection{Сортировка слияниями}

    Слияние: из двух отсортированных массивов делает один отсортированный.

   как найти первй элемент. Он наименьший, значит либо самый левые в массиве $a$, либо в массиве $b$. Мы забыли нужный первый элемент и свели к такой же задаче поменьше.

    \begin{verbatim}
        merge(a,b):
            n = a.size()
            m = b.size()
            i=0, j=0
            while i<n or j<m:
                if j==m or (i<n and a[i]<b[j]):
                    c[k++] = a[i++]
                else
                    c[k++] = b[j++]
            return c
    \end{verbatim}
    $O(n+m)$

    Сортировка: берём массив, делим его пополам, рекурсивно сортируем левую и правую часть, а потом сольём их в один отсортированный массив.

    \begin{verbatim}
        sort(a):
            n = a.size()
            if n<=1:
                return a
            al = [0, .. n/2-1]
            ar = [n/2 .. n-1]
            al = sort(al)
            ar = sort(ar)
            return merge(al, ar)
    \end{verbatim}

    порядка $n$ рекурсивных массивов.

    $T(n) = 2 \cdot T(\frac{n}{2}) + n$

    красиво и понятно: 

    математически и хардкорно: по индукции $T(n) \leqslant \ln  n$

    База: $n=1$ -- не взять из-за логарифма, но можоно на маленькие $n$ не обращать внимания

    Переход:  $$T(n) = 2T(\frac{n}{2}) + n \leqslant 2\cdot \frac{n}{2}\ln  \frac{n}{2} + n = n(\ln  n-1)+n = n \ln  n + n(1-1) \leqslant \ln  n$$

    \begin{theorem}
        [Мастер-теорема]

        $T(n) = aT\left( \frac{n}{b} \right) + f(n) $

        Если без $f(n) = O(n^{\log _ba-\varepsilon})$, то $a^{\log _b^n} = n^{\log _ba}$

        Если без $f(n) = O(n^{\log _ba+\varepsilon})$, тогда $T(n) = O(f(n))$

        Если без $f(n) = O(n^{\log _ba})$, то $T(n) = n^{\ln _ba}\cdot \ln  n$
    \end{theorem}
%
%    ---------------------------------
%
    \section{Структуры данных}

    Структура, которая хранит данные

    Операции: структура данных определляется операциями, которые она умеет исполнять

    Массив:
    \begin{itemize}
        \item get(i) (return a[i])
        \item put(i,v) (a[i] = v)
    \end{itemize}

    Время работы на каждую операцию

    \subsection{Двоичная куча}

    Куча:
    \begin{itemize}
        \item  храним множество ($x<y$)
        \item insert(x) $\qquad A = A\cup \{x\}$
        \item remove\_min() 
    \end{itemize}

    Варианты:
    \begin{enumerate}
        \item Массив
            \begin{itemize}
                \item  insert(x) $a[n++] = x$ (O(1))
                    
                \item remove\_min() (O(n))
                    \begin{verbatim}
                        j=0
                        for i=1 .. n-1
                            if a[i]<a[j]: j=i
                        swap(a[j], a[n-1])
                        return a[--n]
                    \end{verbatim}
            \end{itemize}
        \item Отсортированный массив (по убыванию)
            \begin{itemize}
                \item remove\_min()
                    \begin{verbatim}
                        return a[--n]
                    \end{verbatim}
                \item insert(x)
                    \begin{verbatim}
                        a[n++] = x
                        i=n-1
                        while i >0 and a[i-1]<a[i]
                            swap(a[i], a[i-1])
                            i--
                    \end{verbatim}
            \end{itemize}
            \item Куча. Двоичное дерево, каждого элемента -- 2 ребёнка. У каждого есть один родитель (кроме корня). В каждый узел положим по элементу. Заполняется по слоям. Правило: у дети больше родителя. Минимум в корне -- удобно находить.

                Занумеруем все элементы слева направо. Из узла i идёт путь в $2i+1$ и $2i+2$

                \begin{verbatim}
                    insert(x)
                        a[n++] = x
                        i=n-1
                        while i>0 and a[i]<a[(i-1)/2]
                            swap(a[i], a[(i-1)/2])
                            i = (i-1)/2     
                \end{verbatim}

                $O(\log n)$

                Идея убирания минимума: поставить вверх вместо минимума последний элемент и сделать просеивание вниз.
\newpage
                \begin{verbatim}
                    remove_min()
                        res = a[i]
                        a[0] = a[--n]
                        i=0
                        while True:
                            j=i
                            if 2i+1<n and a[2i+1]<a[j]:
                                j=2i+1
                            if 2i+2<n and a[2i+2]<a[j]:
                                k=2i+2
                            if j == i: break
                            swap(a[i], a[j])
                            i=j
                        return res
                \end{verbatim}
    \end{enumerate}

    \subsection{Сортировка Кучей (Heap Sort)}
    \begin{verbatim}
        sort(a):
            for i = 1 .. n-1: insert(a[i])
            for i = 1 .. n-1: remove_min()
                
        heap_sort(a)
            for i = 0 .. n-1
                sift_up(i)
            for i = n-1 .. 0
                swap(a[0], a[i])
                sift_down(0, i) // i -- размер кучи
    \end{verbatim}
    ---------------------

    \section{Быстрая сортировка}

    \subsection{Рандомизированные алгоритмы}

    Алгоритм: Пусть есть массив и все элементы различны. Давайте выберем случайный элемент
    Поделим массив на две части: $<x$ и $\geqslant x$ -- $O(n)$

    Рекурсивно запускаем от каждого куска.

    \begin{verbatim}
        a // глобальнный массив
        sort(l, r):
            x = a[random(l..r-1)]
            if r-l =1:
                return
            m=l
            for i = l .. r-1:
                if a[i]<x:
                    swap(a[i],a[m])
                    m++
            sort(l,m)
            sort(m,r)
    \end{verbatim}

    Вместо изучения худшего случая рандомизированного алгоритма мы изучаем мат ожидание.

    $E(T(n))$

    $E(x) = \sum t\cdot p(x=t) $ 

    Покажем, что мат ожидание времени работы нашего алгоритма $O(n\log n)$

    Подход №1: посмотрим. был массив, поделили на две части, от каждой части запустились. каждая часть примерно $\frac{n}{2}\qquad T(n) = n+2T(\frac{n}{2})\quad O(n\log n)$

    Скорее всего поделимся не ровно пополам. 

    Подход №2: поделим на 3 части. средняя -- хорошая часть, выбрав элемент в которой части получаются $\leqslant \frac{2}{3}n$. Каждый третий раз пилим пополам примерно. $E(T(n)) \leqslant 3\cdot \log _{\frac{3}{2}} n = O(\log n)$


    \begin{definition}
        К-я порядковая статистика: ровно $k$ элементов меньше выбранного.
    \end{definition}

    Сортировкой: $O(n\log n)$

    Можно быстрее: Аогоритм Хоара

    Возьмём массив $a$, выберем случайный элемент $x$. Распилим массив на 2 куска : $<x, \geqslant x$

    Если знаем $k$, которое ищем, то выбираем одну часть и смотрим там.

    \begin{verbatim}
        a // глобальнный массив
        find(l, r, k): // l<=k<r
            x = a[random(l..r-1)]
            if r-l = 1: // l=k, r = k+1
                return
            m=l
            for i = l .. r-1:
                if a[i]<x:
                    swap(a[i],a[m])
                    m++
            if k<m:
                find(l,m,k)
            else:
                find(m,r,k)
    \end{verbatim}

    \section{Алгоритм Блют-Флойд-Пратт-Ривест-Тарьян}

    Разобьём массив на блоки по 5 элементов. $\frac{n}{5}$ блоков. В каждом блоке выбираем медиану. Выбираем медиану среди всех медиан. Если брать медиану из медиан, то это будет неплохой средний элемент

    $T(n) = n + T(\frac{n}{5}) + T(\frac{7n}{10}) = O(n)$

    $T(n)\leqslant c\cdot n$

    $T(n)\leqslant n + c\cdot \frac{n}{5} + c\cdot \frac{7n}{10} = n\left( 1+\frac{9}{10}\cdot c \right) <c\cdot n\qquad 10\leqslant c$


    Recap: Изучили
    \begin{itemize}
        \item MergeSort
        \item HeapSort
        \item QuickSort
    \end{itemize}

    Все за $O(n\log n)$

    Что мы можем делать: сравнивать элементы и перекладывать их.

    Программа: запустилась, что-то делает, сравнилось: $a[i]<a[j]$. От неё две ветки на два случая. Потом появляются сравнения в подслучаях, дающие больше случаев.
\newpage
    Есть три элемента: $x, y, z$. Отсортируем их.

     $x<y$
      \begin{itemize}
          \item [<|] $x<z$
               \begin{itemize}
                   \item [<|] $x$ -- минимальный.

                       $y<z$
                        \begin{itemize}
                            \item [<|] $xyz$
                            \item [$\not <|$] $xzy$
                       \end{itemize}
                   \item [$\not <|$] $zxy$
              \end{itemize}
          \item [$\not <|$] $y<z$
              \begin{itemize}
                  \item [<|] $x<z$
                       \begin{itemize}
                           \item [<|] $yxz$
                           \item [$\not <|$]  $yzx$
                      \end{itemize}
                  \item [$\not <|$] $zyx$
              \end{itemize}

     \end{itemize}

     В листьях перестановки листьев. $n!$ листьев. Глубина хотя бы $\log n!$

     $T\geqslant \log n! = \sum_{i=1}^{n} \log i = \Omega(n\log n)$

     Можно ли сделать что-то быстрее не только сравнивая.

     Пусть есть массив $a[0\ldots n-1]$. Какие-то маленькие целые числа: $a[i]\in [0\ldots m-1]$, $m$ -- маленькое.

     \section{Сортировка подсчётом}

     $a = [2,0,2,1,1,1,0,2,1]$

      \begin{verbatim}
        cnt = [0,0,0] увеличиваем, проходя по массиву
     \end{verbatim}

     $\p a = [0,0,1,1,1,1,2,2,2]$

     $O(n+m)$

     Но часто сортируются не числа, а объекты, к которым приделаны числа.
      \begin{verbatim}
         class Item {
            int key; // in [0..m-1]
            Data data;
         }
     \end{verbatim} 
     Создадим ящики для объектов с ключами 0,1 и 2.

     В реальности лучше создавать один массив и просто раскладывать в блоки внутри этого массива. 

     А дальше заполняем эти ящики проходя по массиву. А потом соберём их в один большой массив.

     $x = [0\ldots m^2-1]\quad x = a\cdot m + b\quad a,b\in [0\ldots m-1]$ -- двухзначное число в $m$-ичное число

     $a[i] = b[i]\cdot m + c[i]$

     Если нужно остортировать $a[i]$, то это то же самое, что отсортировать пары $\left( b[i],c[i] \right) $ по первому элементу, а при равенстве по второму.

     Пусть есть числа: a = [02 21 01 11 21 20 02 00]


     Можно сортировать по первой цифре, а потом по второй. Но это работает довольно долго.

     А если сортировать слева направо, то лучше:

     \begin{verbatim}
         cnt = [2,4,2]
         a' = [20 00 | 21 01 11 21 | 02 02]
         cnt  = [4,1,3]
         a'' = [00 01 02 02  | 11 | 20 21 21]
     \end{verbatim}
     $O(n+m)$ 

     Дофиксим: если $a[i] \in [0 \ldots m^k-1]$

     $O(k\cdot (n+m))$

     \section{Сортирующие сети} 
     Идея: одна операция -- компаратор

     \begin{verbatim}
         cmp(i,j):
             if a[i] > a[j]
                 swap(a[i],a[j])
     \end{verbatim}

    $n=2\quad cmp(0,1)$

    $n=3\quad cmp(0,1)\quad cmp(0,2)\quad cmp(1,2)$

    сортировка выбором: на первую позицию ставится минимальный элемент, потом сортируются оставшиеся.

    Возьмём сортировку вставками

\begin{figure}[ht]
    \centering
    \incfig{net}
    \caption{net}
    \label{fig:net }
\end{figure}

Порядка $n^2$ компараторов нужно. Рзрешим делаться нескольким компараторам сразу: cmp(0,1), cmp(2,3) друг другу не мешают. Разрешим неконфликтующим компараторам скольким угодно выполняться одновременно.

\begin{figure}[ht]
    \centering
    \incfig{net2}
    \caption{net2}
    \label{fig:net2}
\end{figure}

Элементов уже порядка $n$

 \begin{statement}
    Если сеть сортирует любой массив из 0 и 1, то она сортирует любой массив
\end{statement}
\begin{proof}
    Пускай сеть сортируют любую последовательность 0 и 1. Дадим ей какой-то массив. Наименьший элемент отметим 0, он придёт наверх в сети. Возьмём следующий элемент, он вылезет следом за 1.
\end{proof}

\section{Bitonic sort}

Битонная последовательность -- сначала возрастает, потом убывает.

Рассмотрим только 0 и 1. 

a = [0 0 1 1 1 1 1 0 0 0]

\begin{figure}[Ht]
    \centering
    \incfig{btonic}
    \caption{bitonic}
    \label{fig:btonic}
\end{figure}

Пилим последовательность пополам. Применяем сеть, где сравниваются соответствующие элементы. Обе части битонные, правая больше, чем левая.


В общем случае: сравниваем парами, каждый второй разворачиваем. Получаем битонные последовательности через 4. Применяем к ним Bitonic Sort.

\section{Двоичный поиск}

Есть массив $a$ отсортированный по возрастанию

Как писать \underline{НЕ} надо:
 \begin{verbatim}
    x = 8, i: a[i] = x
    a 2 5 8 13 21 27 35
      l              r
    x in a[l..r] -- инвариант. Будем сужать.
    m=floor((l+r)/2)    l<=m<=r
    a[m] >x => a[j] >x, j>m

    l=0, r=n-1
    while (r-l+1)>0:
        m=(l+r)/2
        if a[m]>x// a[m..r]>x
            r = m - 1
        else if a[m]<x // a[l..m]<x
            l = m + 1
        else
            return m
 \end{verbatim}

 $O(\log n)$

 Задача: найти минимальный $i:a[i]\geqslant x$
 Как писать надо:
 \begin{verbatim}
    l, r
    a[l]<x
    a[r]>=x
    l=-1, r=n
    while (l-r)>1:
        m = (l+r)/2
        if a[m]<x:
            l = m
        else
            r=m
     return r // if r=n: такого элемента вообще нет
 \end{verbatim}

 Задача: найти максимальный $i: a[i]\leqslant x$
 \begin{verbatim}
    l, r
    a[l]<=x
    a[r]>x
    l=-1, r=n
    while (l-r)>1:
        m = (l+r)/2
        if a[m]<=x:
            l = m
        else
            r=m
     return l // if l=-1: такого элемента вообще нет
 \end{verbatim}

$x\in \N  \begin{cases}
    \text{хорошие}\\
    \text{плохие}
\end{cases}$

$x$ -- хорошее $\implies x+1$ -- хорошее

\begin{problem}
    найти минимальное хорошее число.
\end{problem}

$n$ штук прямоугольничков  $w\cdot h$. Мы хотим запихать их в квадрат. Вопрос: какой наименьший квадрат подойдёт.

Если можно впихнуть в $x^2$, то и в $(x+1)^2$ тоже.

\begin{verbatim}
    l -- плохое
    r -- хорошее

    l = 0
    r = max(h,w)*n

    good(l) = 0
    good(r) = 1

    while (l-r)>1:
        m = (l+r)/2
        if good(m):
            r = m
        else
            l = m
    return r
    
    good(x):
      return (x/h)*(x/w)>=n
\end{verbatim}
$O(\log (l-r)) = O(\log \left( \max(h,w) \cdot  n \right) = O(\log (h+w+n))$

Нужно аккуратно брать правое число, чтобы не было переполнений.

$r = max(h,w)\cdot \left\lceil \sqrt{n}  \right\rceil $

$2^k$ -- плохое, $2^{k+1}$ -- хорошее

$\log (a+b) = O(\log  a + \log b)$

\begin{problem}
    Есть прямая 

    На неё в точках $x_i$ живут котики

    $v_i$ -- скорость

    Собраться в одной точке за  $min$ время

     $t$ -- хорошее, если за  $t$ можно собраться
\end{problem}

Как писать \underline{НЕ} надо:
\begin{verbatim}
    r = 0, l = 10^10, EPS = 10^-6
    while (r-l)>EPS:
        m = (l+r)/2
        //fix
        if m <= l || m >= r
            break
        //xif
        if good(m):
            r = m
        else:
            l = m
    good(x):
        x >= max(li)
        x <= max(ri)
        x -- существует, если max(li) <= min(ri)
\end{verbatim}

$O(\log \left( \frac{r-l}{EPS} \right) \cdot n$

так писать не надо, потому что не все числа с нужной точностью можно получить. Точности может не хватать и цикл станет вечным.

Как исправить: можно сделать for


\section{Троичный поиск}
\begin{figure}[ht]
    \centering
    \incfig{tri}
    \caption{tri}
    \label{fig:tri}
\end{figure}

$m_1 = \frac{2l+r}{3}\quad m_2 = \frac{l+2r}{3}$ 

\begin{verbatim}
    if f(m2)>f(m1):
        l = m1
    else:
        r = m2
\end{verbatim}

$O(\log _{\frac{3}{2}} \frac{r-l}{EPS} )$
\end{document}
