\documentclass{book}
%nerd stuff here
\pdfminorversion=7
\pdfsuppresswarningpagegroup=1
% Languages support
\usepackage[utf8]{inputenc}
\usepackage[T2A]{fontenc}
\usepackage[english,russian]{babel}
% Some fancy symbols
\usepackage{textcomp}
\usepackage{stmaryrd}
% Math packages
\usepackage{amsmath, amssymb, amsthm, amsfonts, mathrsfs, dsfont, mathtools}
\usepackage{cancel}
% Bold math
\usepackage{bm}
% Resizing
%\usepackage[left=2cm,right=2cm,top=2cm,bottom=2cm]{geometry}
% Optional font for not math-based subjects
%\usepackage{cmbright}

\author{Коченюк Анатолий}
\title{Математический анализ}

\usepackage{url}
% Fancier tables and lists
\usepackage{booktabs}
\usepackage{enumitem}
% Don't indent paragraphs, leave some space between them
\usepackage{parskip}
% Hide page number when page is empty
\usepackage{emptypage}
\usepackage{subcaption}
\usepackage{multicol}
\usepackage{xcolor}
% Some shortcuts
\newcommand\N{\ensuremath{\mathbb{N}}}
\newcommand\R{\ensuremath{\mathbb{R}}}
\newcommand\Z{\ensuremath{\mathbb{Z}}}
\renewcommand\O{\ensuremath{\emptyset}}
\newcommand\Q{\ensuremath{\mathbb{Q}}}
\renewcommand\C{\ensuremath{\mathbb{C}}}
\newcommand{\p}[1]{#1^{\prime}}
\newcommand{\pp}[1]{#1^{\prime\prime}}
\newcommand{\ov}[1]{\overline{#1}}
\renewcommand\phi{\varphi}
% Easily typeset systems of equations (French package) [like cases, but it aligns everything]
\usepackage{systeme}
\usepackage{lipsum}
% limits are put below (optional for int)
\let\svlim\lim\def\lim{\svlim\limits}
%\let\svlim\int\def\int{\svlim\limits}
% Command for short corrections
% Usage: 1+1=\correct{3}{2}
\definecolor{correct}{HTML}{009900}
\newcommand\correct[2]{\ensuremath{\:}{\color{red}{#1}}\ensuremath{\to }{\color{correct}{#2}}\ensuremath{\:}}
\newcommand\green[1]{{\color{correct}{#1}}}
% Hide parts
\newcommand\hide[1]{}
% si unitx
\usepackage{siunitx}
\sisetup{locale = FR}
% Environments
% For box around Definition, Theorem, \ldots
\usepackage{mdframed}
\mdfsetup{skipabove=1em,skipbelow=0em}
\theoremstyle{definition}
\newmdtheoremenv[nobreak=true]{definition}{Определение}
\newmdtheoremenv[nobreak=true]{theorem}{Теорема}
\newmdtheoremenv[nobreak=true]{lemma}{Лемма}
\newmdtheoremenv[nobreak=true]{problem}{Задача}
\newmdtheoremenv[nobreak=true]{property}{Свойство}
\newmdtheoremenv[nobreak=true]{statement}{Утверждение}
\newmdtheoremenv[nobreak=true]{corollary}{Следствие}
\newtheorem*{note}{Замечание}
\newtheorem*{example}{Пример}
\renewcommand\qedsymbol{$\blacksquare$}
% Fix some spacing
% http://tex.stackexchange.com/questions/22119/how-can-i-change-the-spacing-before-theorems-with-amsthm
\makeatletter
\def\thm@space@setup{%
  \thm@preskip=\parskip \thm@postskip=0pt
}
\usepackage{xifthen}
\def\testdateparts#1{\dateparts#1\relax}
\def\dateparts#1 #2 #3 #4 #5\relax{
    \marginpar{\small\textsf{\mbox{#1 #2 #3 #5}}}
}

\def\@lecture{}%
\newcommand{\lecture}[3]{
    \ifthenelse{\isempty{#3}}{%
        \def\@lecture{Lecture #1}%
    }{%
        \def\@lecture{Lecture #1: #3}%
    }%
    \subsection*{\@lecture}
    \marginpar{\small\textsf{\mbox{#2}}}
}
% Todonotes and inline notes in fancy boxes
\usepackage{todonotes}
\usepackage{tcolorbox}

% Make boxes breakable
\tcbuselibrary{breakable}
\newenvironment{correction}{\begin{tcolorbox}[
    arc=0mm,
    colback=white,
    colframe=green!60!black,
    title=Correction,
    fonttitle=\sffamily,
    breakable
]}{\end{tcolorbox}}
% These are the fancy headers
\usepackage{fancyhdr}
\pagestyle{fancy}

% LE: left even
% RO: right odd
% CE, CO: center even, center odd
% My name for when I print my lecture notes to use for an open book exam.
% \fancyhead[LE,RO]{Gilles Castel}

\fancyhead[RO,LE]{\@lecture} % Right odd,  Left even
\fancyhead[RE,LO]{}          % Right even, Left odd

\fancyfoot[RO,LE]{\thepage}  % Right odd,something additional 1  Left even
\fancyfoot[RE,LO]{}          % Right even, Left odd
\fancyfoot[C]{\leftmark}     % Center

\usepackage{import}
\usepackage{xifthen}
\usepackage{pdfpages}
\usepackage{transparent}
\newcommand{\incfig}[1]{%
    \def\svgwidth{\columnwidth}
    \import{./figures/}{#1.pdf_tex}
}
\usepackage{tikz}
\usepackage{pgfplots}
\begin{document}
    \maketitle
    \tableofcontents
    \newpage
    \section{Введение}
    Семёнова Ольга Львовна

    o\_ semenova@mail.ru

   
   Литература: \begin{enumerate}
       \item Виноградов О.Л. Курс Математического анализа
       \item Виноградов, громов --||--
       \item Фихтенгольц (курс)
       \item Зорич (курс, двухтомник)
       \item Кудрявцев (сборник задач, 1 том из трёх)
       \item Виноградова, Олехник, Саровничий (1 том из двух)
   \end{enumerate}

    \section{Баллы}

    практика -- 70/100\\
    теория -- 30/100 -- 2-4 теста по теории (3 балла за присутствие на $\sim $всех лекциях

    Если меньше 18/30 баллов, то всю теорию нужно будет пересдавать. Иначе можно воспользоваться этим как баллами за экзамен.

    \chapter{Множества, отображения, $\R$}

    \section{Множества}
    ''Множество'' -- неопределямое слово. Синонимы: набор, совокупность, класс

    Множество состоит из элементов. $M = \{1,3,7,9\}, \N , \Q, \Z , \Z_+ = \{0,1,2,3,4,\ldots\}, \R, \R_+$

    способы описания:
    \begin{itemize}
        \item явное описание $\{1, 2, 3\}$
        \item через некоторое свойство $M = \{x: P(x)\}\quad :$ -- читается как ''таких что''. Тот же смысл имеет $\mid$.
            $P(x)$ обозначает какое-то свойство.
            $M = \{x:x\text{$x$ -- человек и $x$ 2002 г.р.}\}$
    \end{itemize}
    Кванторы:
    \begin{itemize}
        \item $\forall $ -- ''для любого'', любой, каждый, всякий $\ldots$
        \item $\exists $ -- ''существует''
    \end{itemize}

    Пример: $\forall \varepsilon>0 \exists  \delta>0 :\ldots$

   Для любого положительного эпсилон существует положительное число дельта, т.ч. $\ldots$
    Обозначения:
    \begin{itemize}
        \item $\iff $ -- равносильно
        \item $\wedge$ -- ''и''
        \item $\vee$ -- ''или''
        \item $\sqsupset $ пусть
        \item $\sphericalangle$ -- допустим, рассмотрим
    \end{itemize}

    \begin{note}
        Множество всех множеств не существует

        $\neg$ -- отрицания

        $\neg \exists $ -- не существует
        
        $\O $ -- пустое множество

        $x\in M \iff x$ -- элемент множества $M$

        $A\subseteq B \iff  (x\in A \implies  x\in B)$

        $B\supseteq A$ -- то же самое
    \end{note}
    $\forall $ множества $M\quad \O \subseteq M$

    $A = B \iff  \left( x\in A \iff x\in B \right)  \iff \begin{cases}
        A\subseteq B\\
        B\subseteq A\\
    \end{cases}$
    
    
        $A,B$ -- множества 

        $A\cup B = \{x: (x\in A \vee x\in B)\}$
        
        $A\cap B = \{x: (x\in A \wedge x\in B)\}$

        $x\in A\cap B \iff  \begin{cases}
            x\in A\\
            x\in B\\
        \end{cases}$

        $A\setminus B = \{x:x \in A, x\not\in B \}$

        $A\subset C$

        $A^c = X\setminus A$ -- дополнение $A$ в $X$

        \begin{definition}
            $A,X_{\alpha}$ -- множества, $\forall \alpha \in A$

            $\{X_{\alpha}\}_{\alpha\in A}$ -- семейство множеств

            $A$ -- индексное множество

            $\bigcup\limits_{\alpha\in A}X_{\alpha} = \{x:\exists \alpha\in A\quad x\in X_{\alpha}\} $

            $\bigcap\limits_{\alpha\in A}X_{\alpha} = \{x: \forall \alpha\in A\quad x\in x_{\alpha}\} $
        \end{definition}

        \begin{example}
            $\{(x-1,x+1)\}_{x\in (0;1)}$

            $\bigcup\limits_{x\in (0;1)}(x-1, x+1) = (-1,2) , \bigcap\limits_{x\in (0,1)} (x-1,x+1) = (0;1) $
        \end{example}

        \begin{definition}
            [Формула Де Моргана]

            $A,B\subseteq X$

            $(A\cup B)^c = A^c\cap B^c$

            $(A\cap B)^c = A^c\cup B^c$
                

            $\{A_i\}$ -- семейство

            $(\bigcup\limits_{i\in I}A_i)^c  = \bigcap A_i^c$

            $\left( \bigcap\limits_{i \in  I} A_i \right) ^c = \bigcup A_i^c $
        \end{definition}
        \begin{note}
            $A^{c c} = A$ -- проверить-упражнение
        \end{note}

        \begin{proof}
            $x\in \left( \bigcup\limits_{i \in  I} A_i \right) ^c \iff  x\not\in \left( \bigcup\limits_{i \in  I} A_i \right) \iff \forall i\in I x\not\in A_i \iff  \forall i\in I x\in A_i^c \iff x\in \bigcap\limits_{i \in  I} A_i^c$

            $\left( \bigcap A_i \right) ^c = \left( \bigcap A_i^{c c} \right) ^c = \left( \bigcup A_i^c \right) ^{c c} = \bigcup A_i$
        \end{proof}

        \begin{definition}
            [упорядоченная пара]
            $A,B$

            $(a,b)$ -- упорядоченная пара, $a\in A, b\in B$. В этой паре важен порядок

            $\{a,b\}$ -- непорядоченная пара (двухэлементное множество), если $a\neq b$

            {a,a} = {a} (в множестве не различаются копии)
            
            Пример: координаты точек плоскости. 
        \end{definition}

        $X_1, \ldots, X_m\quad x_1\in X_1 \ldots x_m\in X_m\quad (x_{1}, \ldots, x_{m} )$ -- упорядоченная пара

        \begin{definition}
            [Декартово произведение]

            $X_1\times \ldots\times X_m = \{(x_1, \ldots, x_{m} ):x_k\in X_k\quad k = 1:m\}$

            $R^m = (R)^m$
        \end{definition}

        \begin{example}
            $X = \{1,2\}$

            $Y = \{3,5\}$

            $Z = \{0\}$

            $X\times Y\times Z = \{(1,3,0),(2,3,0),(2,3,0),(2,5,0)\}$
        \end{example}

        \section{Отображения}
        формальное определение, которое не будет использовано или потребовано нигде (в том числе на экзамене)
        \begin{definition}
           $X,Y$ -- множества

           если $R\subset X\times Y$ и $(x,y_1)\in R \vee (x,y_2)\in R \iff y_1=y_2$



    R называется отображением или графиком
    \end{definition}
    
        

    \begin{definition}
            Отображение -- это тройка $(X,Y,f)$, где $X,Y$ -- множества, а $f$ -- некое правило по которому каждому элементу $x\in X$ сопоставляется некоторый единственный элемент $y\in Y$
            
            $f:X\to Y$ -- синоним. читают ''$f$ действует из $X$ в $Y$''

            $X$ -- множество определения отображения

            $Y$ -- множество значений

            $\{y\in Y: \exists x\in X f(x) = y\}\subset Y$ (т.е. $Y$ -- необязательно точное множество значений)
    \end{definition}
    
    \begin{example}
        $x = \R, Y = \R \quad x\mapsto x^2$

        если $y=f(x)$, то $y$ называется образом элемента $x$ при отображении $f$
    \end{example}

    $A\subseteq X\quad f(A) = \{f(x): x\in A\}$ -- образ множества $A$ под действием $f$

    $B\subseteq Y\quad f^{-1}(B) = \{x:f(x)\in B\}$

    $f^{-1}(\{y\})$ -- необязательно одноэлементное.


упражнения:
\begin{enumerate}
    \item $f(A\cup B), f(A)\cup f(B) $
    \item $f(A\cap B), f(A) \cap f(B)
$
        $y\in f(A\cap B) \implies  \exists x\in A\cap B:f(x) = y. x\in A, x\in B, y\in f(A), y\in f(B) \implies  y\in f(A)\cap f(B)$

        $f(x) = const\quad f(A)\cap f(B)\neq \O , f(A\cap B) = \O$ , если $A\cap B = \O $
    \item $f^{-1}(A\cup B), f^{-1}(A)\cup f^{-1}(B)$
    \item $f(A\cap B), f(A) \cap f(B)$
\end{enumerate}

\begin{definition}
    Если $f:X\to Y, g:X_1\to Y\quad X_1\subseteq X $

    и $\forall x\in X_1 \quad g(x) = f(x)$, то $g$ называется сужением $f$ на $X_1$

    Обозначение: $g = f\mid_{x_1}$

    При этом $f$ называется продолжением $g$ с $X_1$ на $X$
\end{definition}

\begin{figure}[h]
    \centering
    \begin{tikzpicture}
        \begin{axis}[
            axis equal,
            xmin= -1.57, xmax= 1.57,
            ymin= -1.5, ymax = 1.5,
            axis lines = middle,
        ]
        \addplot[domain=-1.57:1.57, samples=100,color=green]{sin(deg(x))};
        \end{axis}
    \end{tikzpicture}
    \caption{sinus}
    \label{sinus}
\end{figure}

\begin{example}
    $f(x) = \sin x\quad f\mid_{[-\frac{\pi}{2}, \frac{\pi}{2}]}$
    
\end{example}
   
\begin{definition}
    Если $f:X\to Y, g:Y\to Z$

    $g\circ f:X\to Z$

    $g\circ f(x) = g(f(x)) \forall x\in X$

    $g\circ f$ называется композицией $f$ и $g$
\end{definition}
\begin{example}
    Изобразить эскизы графиков функций для всех случаев
    \begin{enumerate}
        \item $f(x) = \sin  x, g(x) = x^2$
        \item $f(x) = x^2, g(x) = \sin  x$
        \item f(x) = g(x)

\begin{tikzpicture}[scale=2]
	\draw[<->](0,1.5) -- (0,0) -- (1.5,0);
	\draw (0,0) -- (0.5, 1);
	\draw[dotted](0.5,0) -- (0.5,1);
	\draw[dotted](0,1) -- (0.5,1);
	\draw (0.5,0.5) -- (1,0);
	\draw[dotted](0,0.5) -- (0.5,0.5);
	\node[left] at (0,1) {1};
	\node[below] at (1,0) {1};
	\node[below] at (-0.1,0) {0};
	\node[below] at (0.5,0) {$\frac{1}{2}$};
	\node[left] at (0,0.5) {$\frac{1}{2}$};
	
\end{tikzpicture}

построить $f^{(2)}, f^{(3)}, f^{(4)}$ без формул
    \end{enumerate}

\end{example}
   \begin{definition}
       Пусть $f:X\to Y, f$ называется инъекцией, если $\begin{cases}
           f(x_1) = y\\
           f(x_2) = y\\
       \end{cases} \implies  x_1=x_2$
   \end{definition}
    \begin{example}
        $f(x) = kx+b$ -- инъекция $,k\neq 0$

        $f(x) = \sin x$ -- не инъекция

    $f(x)\mid_{[-\frac{\pi}{2}, \frac{\pi}{2}]  }$ -- инъекция

    \end{example}
    \begin{definition}
        $f:X\to Y, f$ называется сюръекцией, если $\forall y\in Y\quad \exists x\in X:f(x) = y$
    \end{definition}
    \begin{example}
        $\sin :\R\to \R$ -- не сюъекция

        $\sin :\R\to [-1,1]$ -- сюръекция

        $y = kx+b, k\neq 0$ -- сюръекцией
    \end{example}

    \begin{definition}[биективность]
        $f:X\to Y$ -- инъекция и сюръекция $\implies f$ называется биекцией
    \end{definition}

    \begin{example}
        $y = kx+b, k\neq 0$ -- биекция
    \end{example}

    \begin{definition}
        $f:X\to Y, g:Y\to X$

        $g$ называется обратным к $f$ отображением, если $f(x) = y \iff  x = g(y)$

        Обозначается: $g = f^{-1}$
    \end{definition}
    \begin{note}
        Обратимая функция должна быть биективной:
        
        Инъективной -- обратная иначе не будет функцией

        Сюръективной -- обратная не будет определена на всём $Y$
    \end{note}
    \begin{note}
        $f^{-1}(A)$ -- обычно прообраз $A$ под действием $f$, а не образ обратной функции (которая может не существовать)
   \end{note}

   \begin{example}
       $\arcsin = (\sin _{[-\frac{\pi}{2}, \frac{\pi}{2}]  })^{-1}$

       $\log_a(x) = (a^x)^{-1}$

       $\sqrt{x} = (x_{2}\mid_{[0,+\infty )} , \sqrt[3](x) = (x^3)^{-1} $
   \end{example}
   \section{Вещественные числа}
    \subsection{Аксиоматическое определение вещественных чисел}

        $(\R, +, \cdot, \leqslant )$ -- множество и две операции, и отношение порядка, удовлетворяющее следующим 16 аксиомам:
        
        Аксиомы поля:
        \begin{enumerate}
            \item $\forall a, b\in \R\quad a+b=b+a$ (коммутативность сложения)
            \item $\forall a, b, c\in \R\quad (a+b)+c = a+(b+c)$ (ассоциативность сложения)
            \item $\exists $ нейтральный элемент $0$ по сложению $\forall a\in \R\quad a+0=a$ (существование нейтрального элемента по сложению)
            \item Существует обратный элемент по сложению. $\forall a\in \R \exists (-a)\in \R: \quad a+(-a) = 0$
            \item $\forall a, b\in \R\quad a\cdot b = b\cdot a$ (коммутативность умножения)
            \item $(a\cdot b)\cdot c = a\cdot (b\cdot c)$ (ассоциативность умножения)
            \item $\exists 1\in R\setminus \{0\}, \forall a\in \R\quad a\cdot 1 = a$
            \item $\forall a\neq 0 \exists  a^{-1}\in \R: a\cdot a^{-1} = 1$
            \item $(a+b)\cdot c = a\cdot c + b\cdot c$ (дистрибутивность)
        \end{enumerate}
        примеры: $\R, \Q, \C, \{0,1\} (1+1=0,\text{ остальное как обычно}$

        Элементарные следствия:
        \begin{itemize}
            \item $\forall a\in K$ -- поля, обратный по сложению единственный. Если $b, \p b$ -- два обратных.$b= b+(a+\p b)=(b+a)+\p b = \p b$
            \item обратный по умножению, нейтральные -- все единственны
        \end{itemize}
        Аксиомы порядка:
        \begin{enumerate}
            \item $\forall a, b\in \R\quad a\leqslant b \vee b\leqslant a$
            \item $a\leqslant b, b\leqslant c \implies a\leqslant c$ (транзитивность)
            \item $a\leqslant b, b\leqslant a \implies a=b$ 
            \item $a\leqslant b, c\in \R\implies a+c\leqslant b+c$
            \item $a\geqslant 0, b\geqslant 0$, то $a\cdot b\geqslant 0$
        \end{enumerate}
        $0\cdot x + x\cdot x = (0+x)\cdot x = x\cdot x \implies  0\cdot x = 0$

        Упражнения:
        \begin{enumerate}
            \item $-x = (-1)\cdot x$
            \item $(-a)(-b) = a\cdot b$
            \item $1\geqslant 0$
        \end{enumerate}
        \begin{definition}
            Индуктивным множеством в упорядоченном поле $(K, +, \cdot , \leqslant )$ называется множество $N$:
           \begin{enumerate}
               \item $1\in N$
               \item $\forall x\in N \implies  x+1\in N$
           \end{enumerate}

           $\N $ -- наименьшее индуктивное множество. $\N  = \bigcap\limits_{N\text{ -- индуктивных}, N\subseteq \R}N $
        \end{definition}
        \begin{note}
            $x>b \iff \begin{cases}
                x\geqslant b\\
                x\neq b\\
            \end{cases}$
        \end{note}
        Аксиома Архимеда: $\forall x, y\in R: x>0, y>0 \exists  n\in \N : nx>y$

        Аксиома вложенных промежутков: 
        
        $\forall \{[a_n, b_n]\}_{n\in \N }: \forall n\in \N [a_{n+1}, b_{n+1}] \subseteq [a_n, b_n]\quad \bigcap\limits_{n\in \N } [a_n, b_n]\neq \O  $

        В аксиоме о вложенных промежутках предполагается, что $\forall  n\in \N , [a_n, b_n] \neq \O  \iff  a_n \leqslant  b_n$

        $[a,b] = \{x\in \R,: a\leqslant x\leqslant b\}$ -- замкнутый отрезок, промежуток, сегмент, замкнутый промежуток

        $(a,b) = \{x\in \R: a<x<b\}$ -- интервал, открытый промежуток

        $(a,b], [a,b)$ -- полуоткрытый промежуток

        $<a,b>$ -- некоторый промежуток $a\leqslant b, <a,b> \neq \O $

        \begin{note}
            [Расиширенная вещественная прямая]

            $\overline{R} = \R\cup \{+\infty \}\cup \{-\infty \}$

            $\forall a\in \R\quad a + (+\infty ) = (+\infty ) + a = +\infty $

            $a+(-\infty ) = (-\infty ) + a = -\infty $

            $(+\infty ) + (+\infty ) = +\infty $

            $(-\infty ) + (-\infty ) = -\infty $

            $(+\infty ) + (-\infty ), (-\infty ) + (+\infty )$ -- не определены

            $\forall a>0$
            \begin{itemize}
                \item $(+\infty )\cdot a = a\cdot (+\infty ) = +\infty $
                \item $(-\infty ) \cdot  a = a\cdot (-\infty ) = -\infty $
                \item $\pm \infty \cdot (-1) = \mp \infty $
                \item $(+\infty ) \cdot  (+\infty ) = (-\infty )\cdot (-\infty ) = +\infty $
                \item $(+\infty )\cdot (-\infty ) = (-\infty )\cdot (+\infty ) = -\infty $
                \item $(\pm \infty )\cdot 0, 0\cdot (\pm \infty )$ -- не определены
            \end{itemize}

            $\forall a\in \R$
            \begin{itemize}
                \item $+\infty \geqslant a\geqslant -\infty $
            \end{itemize}

            $[a,+\infty ] = [a,+\infty ) \cup \{+\infty \}$

            В ``$+\infty $'' иногда + опускают, но подразумевают её, если рассматривается $\ov R$
        \end{note}

        \section{Модуль}

        $a\in \R\qquad |a| = \begin{cases}
            a&,\text{если} a\geqslant 0\\
            -a&, \text{если} a<0\\
        \end{cases}$

        \begin{property}
            $b=|a| \iff  \begin{cases}
                b\in \{a, -a\}\\
                b\geqslant 0\\
            \end{cases}$
        \end{property}

        Элементарные свойства модуля:
        \begin{enumerate}
            \item $\forall a\in \R\quad |-a| =|a|$
            \item $\forall a\in \R\quad \pm a \leqslant  |a|$
            \item $\forall  a, b\in \R\quad |a\cdot b| = |a|\cdot |b|$
            \item $\forall a\in \R, b\in \R\setminus \{0\}\quad(\left| \frac{a}{b} \right| ) = \frac{|a|}{|b|}$
            \item $\forall a, b\in \R\quad \left| |a| - |b| \right| \leqslant |a\pm b| \leqslant |a|+|b|$
        \end{enumerate}
        \begin{note}
            $a-b := a+(-b)$

            $\frac{a}{b}:= a\cdot b^{-1}, b\neq 0$
        \end{note}
        \begin{note}
            $a\leqslant b\quad\forall c\in \R\quad a+c\leqslant b+c$

            $a\leqslant b\quad b\leqslant c \implies a\leqslant c$

            $a\leqslant b, c\leqslant d\quad a+c\overset{?}{\leqslant} b+d$

            $a+c\leqslant b+c\quad b+c\leqslant b+d \implies a+c\leqslant b+d$
        \end{note}
        \begin{proof}
            $\forall a, b\in \R\quad \left| |a| - |b| \right| \leqslant |a\pm b| \leqslant |a|+|b|$

            $\pm a\leqslant |a|,\pm b\leqslant |b|\quad \pm(a+b)\leqslant |a| + |b| $ (аксиома порядка 4)

            $\implies |a+b|\leqslant |a| + |b| \implies |a-b|\leqslant |a| + |-b| = |a| + |b|$

            $|a| = |a-b+b| \leqslant |a-b| + |b|$

            $|a| - |b|\leqslant |a-b|\quad |b| - |a| \leqslant |b-a| = |a-b|$

            $\left| |a| - |b| \right| =  \pm (|a| - |b|) \leqslant |a-b|$
        \end{proof}
        \begin{note}
            $|+\infty | := +\infty \quad |-\infty | := +\infty $
        \end{note}

        \section{Комплексные числа}
        $\C$ -- обозначение для множества комплексных чисел

        $\C = \{(x,y)|x\in \R, y\in \R\}$

        \begin{tikzpicture}
            \draw [<->] (0,1) -- (0,-1) -- (0,0) -- (-1,0) --  (1,0);

        \end{tikzpicture}
        Удобно представлять на плоскости.

        $(x_1,y_1) + (x_2,y_2) := (x_1+x_2,y_1+y_2)$

        $(x_1,y_1)\cdot (x_2,y_2) = (x_1x_2-y_1y_2,x_1y_2+x_2y_1) $
        \begin{note}
            $\C$ -- поле

            аксиомы для сложения очевидны.

            $0 = (0,0)$

            $1 = (1,0)$

            $-(x,y) = (-x,-y)$

            $(x,y) \cdot  (1,0) = (x,y) \forall x, y\in \R$

            $i = (0,1)$

            $i^2 = (0,1) \cdot  (0,1) = (-1,0)$
        \end{note}

        $\R \leftrightarrow \{(x,0) : x\in \R\}$ (именно такие пары, потому что так сохраняются операции

        $F: \R \to \{(x,0)\}\quad F$ сохраняет + и сохраняет $\cdot $ 

        $(x_1,0) + (x_2,0) = (x_1+x_2,0)$

        $(x_1, 0) \cdot  (x_2,0) = (x_1x_2-0,0)$

        \begin{tikzpicture}
            \draw [<->] (0,1) -- (0,-1) -- (0,0) -- (-1,0) --  (1,0);
        \end{tikzpicture}


        Оси: вещественная(x) и мнимая(y)

        $(0,y)^2 = (-y^2,0) \forall y\in \R$

        $(x,y) = x+iy\quad i = (0,1)$ -- мнимая единица. Алгебраическая форма записи комплексных чисел.

        $z = x+iy\quad x$ -- вещественная часть $z,\quad y$ -- мнимая часть $z$

        $Re z = x,\quad Im z = y\quad $ иногда встречается $rp, ip$ -- real/imaginary part

        \begin{note}
            [Комплексное сопряжение]
            $z = x+iy\quad \ov z = x-iy$ -- отражённое от оси $x$, если смотреть на плоскость.

            $Re z = \frac{z+\ov z}{2}\qquad Im z = \frac{z-\ov z}{2i}$ -- вещественные числа!
        \end{note}

        \begin{note}
            [Модуль и аргумент]

            $|z| = \sqrt{z\cdot \ov z}  = \sqrt{x^2+y^2} $ 

            ($z\cdot \ov z = (x+iy)\cdot (x-iy) = x^2-(iy)^2 = x^2+y^2$)

            $r = |z|$

            Аргумент -- угол (ориентированный) между осью $Ox$ и $\overset{\to }{Oz}$

            Аргументов много $Arg z, z\neq 0$ -- совокупность всех аргументов

            Если $\varphi_0\in Arg (z)$, то $Arg z = \{\varphi_0 + 2\pi\cdot k, k\in \Z \}$

            $\begin{cases}
                \varphi_0\in Arg z\\
                \varphi_0\in (-\pi , \pi ]\\
            \end{cases} \implies  \varphi_0$ Называется главным значением аргумента $\varphi_0 = arg(z)$ 
        \end{note}

        $z = (x,y) = (r,\phi),r$ -- длина радиус-вектора, $\phi$ --  аргумент.

        $(r, \phi)$ -- полярные координаты, совмещённые с прямоугольными
        \begin{note}
            $r = \sqrt{x^2+y^2} $

            $x = r\cos \phi$

            $y = r\cos \phi$

            $x>0\quad arg z = \arctg \frac{y}{x} = \arcsin \frac{y}{\sqrt{x^2+y^2} }$

            $y>0\quad arg z = \arccos \frac{x}{\sqrt{x^2+y^2} }$

            $y<0\quad arg z = -\arccos \frac{x}{\sqrt{x^2+y^2} } $

            $\begin{cases}
                x<0\\
                y>0\\
            \end{cases}\quad arg z = \arctg \frac{y}{x} + \pi $

            остальное -- упражнение
        \end{note}

        Изобразить кривую заданную в полярных координатах 
        \begin{enumerate}
            \item $r = 3$
            \item $r = \phi$ -- спираль Архимеда
            \item $r = e^{\phi}$
            \item $r = \frac{1}{\cos \phi}$
            \item $r = \frac{2}{\sin \phi}$
            \item $r = \frac{3}{\cos \phi + \sin \phi}$
            \item $r = 1+\cos \phi$
        \end{enumerate}

        $(0,0)$ -- полюс

        $r(\phi) \uparrow$ -- удаление от полюса

        $z = x+iy = r(\cos \phi + i\sin \phi)$ -- в скобках точка на единичной окружности с аргументом таким же, что и у $z$

        Это называется тригонометрической формой записи числа.

        $-\frac{1}{2}(\cos \phi_0, \sin \phi_0) = \frac{1}{2} \cdot  \left( \cos (\phi_0+\pi ) + i\sin (\phi_0+\pi ) \right) $

        $e^{i\phi}:=\cos \phi + i\sin\phi\quad \phi\in \R$

        $r\cdot e^{i\phi}$ -- экспоненциальная (показательная) форма числа

        $\cos \phi = \frac{e^{i\phi} + e^{-i\phi}}{2}\quad \sin \phi = \frac{e^{i\phi} + e^{-i\phi}}{2i}$

        Если $z_1 = r_1 \cdot  e^{i\phi_1}, z_2 = r_2 \cdot  e^{i\phi_2}$, то 

        $z_1\cdot z_2 = r_1r_2\cdot e^{i\left( \phi_1 + \phi_2 \right) }$ (см. курс алгебра)

        $n\in \N \quad z^n = r^n \cdot  (\cos (n\phi) + i\sin(n\phi))$ -- формула Муавра

        \section{Дополнение к разделу\\``Действия над множествами''}

        \begin{statement}
            $\sqsupset B$ -- $\forall $ множество, $\{A_i\}_{i\in I} -- \forall $ семейство множеств

            $B\cap \left( \bigcup\limits_{i \in  I} A_i \right)  = \bigcup\limits_{i \in  I} \left( B\cap A_i \right) $
        \end{statement}
        \begin{proof}
            $x\in B\cap \left( \bigcup\limits_{i \in  I} A_i \right) \iff \begin{cases}
                x\in B\\
                x\in \cup A_i\\
            \end{cases} \iff  \begin{cases}
                x\in B\\
                \exists i:x\in A_i\\
            \end{cases} \iff  \exists i: \begin{cases}
                x\in B\\
                x\in A_i\\
            \end{cases} \iff \exists i: x\in B\cap A_i \iff x\in  \bigcup\limits_{i \in  I} \left( B\cap A_i \right) $
        \end{proof}

        \section{Принцип математической индукции}

       $P_n$ - утверждение, зависящее от $n$

       Если $\begin{cases}
           P_1 \text{-- верно}\\
           P_n \to P_{n+1}\\
       \end{cases} \implies  \forall n\in \N\quad P_n$ -- верно

       $\{n: P_n \text{-- верно}\}$ -- индуктивно $\implies  \N \subseteq \{n:P_n \text{ -- верно}\}$

       Первый шаг (проверка $P_1$) называется базой индукции, а второй -- переходом

       \begin{example}
           $2^n\geqslant n^2\quad \forall n\geqslant 4, n\in \N $

           $P_4\quad 2^4\geqslant 4^2\quad 16\geqslant 16$ -- верно

           $\sqsupset P_n$ -- верно

           $P_{n+1}\quad 2^{n+1}\geqslant (n+1)^2$

           $2^{n+1} = 2^n\cdot 2 \geqslant  n^2\cdot 2 \overset{?}{\geqslant} (n+1)^2 \iff \left( \frac{n+1}{n} \right)^2 \leqslant 2  $

           $\frac{n+1}{n} = 1+\frac{1}{n}\leqslant 1+\frac{1}{4}$

           $\left( 1+\frac{1}{n} \right) ^2 \leqslant \left( 1+\frac{1}{4} \right) ^2 = \frac{25}{16}\leqslant 2$
       \end{example}


       \begin{definition}
           $\forall n\in \N \quad n! := 1\cdot 2\cdot \ldots\cdot n$

           $0! := 1$ -- соглашение

           $(n+k)! = n!\cdot (n+1) \cdot  \ldots \cdot  (n+k)$

           $n!! = n\cdot (n-2) \cdot  \ldots$ (заканчивается либо 1, либо 2)

           $n$ -- чётно, $n!! = 2\cdot 4\cdot \ldots\cdot n$

           $n$ -- нечётно, $n!! = 1\cdot 3\cdot 5\cdot \ldots\cdot n$
       \end{definition}

       \begin{definition}
           [биноминальный коэффициент]

           $C_n^k = \frac{n!}{k!(n-k)!}$ -- биномиальный коэффициент, число сочетаний из $n$ по $k$

           $\begin{pmatrix} n\\k \end{pmatrix} $
       \end{definition}
       Элементарный свойства биномиальных кэффициентов:
       \begin{enumerate}
           \item $C_n^k = C_n^{n-k}$
           \item $C_n^0 = C_n^n = 1$
           \item $C_n^1 = C_n^{n-1} = n$
           \item $C_n^k + C_n^{k-1} = C_{n+1}^k$

               $\frac{n!}{k!(n-k)!} + \frac{n!}{(k-1)!\cdot (n-k+1)!} = \frac{n!}{k!(n+1-k)!} \cdot  (n+1-k+k) = C_{n+1}^k$ 
       \end{enumerate}

       $$1$$ $$1\quad 1$$ $$1\quad 2\quad 1$$ $$1\quad 3\quad 3\quad 1$$ $$1\quad 4\quad 6\quad 4\quad 1$$

       \begin{statement}
           $\forall a, b\in \C\quad \forall n\in \Z _+\quad (a+b)^n = \sum_{k=0}^{n} C_n^ka^kb^{n-k}$ -- бином Ньютона
       \end{statement}

       \begin{note}
           $\sum_{k=1}^{N} a_k := a_1 + a_2 + \ldots + a_N$

           $\sum_{k=m}^{m+p} a_k := a_m + a_{m+1} + \ldots + a_{m+p}$

           $\prod_{k=m}^{m+p} a_k := a_m\cdot a_{m+1}\cdot \ldots\cdot a_{m+p} $
       \end{note}
        \begin{note}
            $x^0:=1 \forall x\in \C $ -- определили функцию
        \end{note}
       \begin{proof}
           [Доказательство бинома по индукции]
           База: $n=1\quad (a+b)^1 = \sum_{k=0}^{1} C_1^k a^kb^{1-k} = C_1^0 a^0b^1 + C_1^1a^1b^0 = a+b$

           Переход: Пусть верно для $n$. Докажем для $n+1$

           $(a+b)^{n+1} = (a+b)^n(a+b) = \left(\sum_{k=0}^{n} C_n^k a^k b^{n-k}\right) \cdot  (a+b) = \sum_{k=0}^{n} C_n^k a^{k+1}b^{n-k} + \sum_{k=0}^{n} C_n^ka^kb^{n-k+1} \underset{(j=k+1)} = \sum_{j=1}^{n+1} C_n^{j-1}a^jb^{n+1-j} + \sum_{k=0}^{n} C_n^k a^kb^{n+1-k} \underset{k=j}= \sum_{k=1}^{n+1} C_n^{k-1} a^k b^{n+1-k} + \sum_{k=0}^{n} C_n^ka^kb^{n+1-k}  = C_{n}^na^{n+1}b^0 + \sum_{k=1}^{n}\left( C_n^{k-1}a^kb^{n+1-k} + C_n^k a^kb^{n+1-k}\right) + C_n^0a^0b^{n+1} = C_{n+1}^{n+1}a^{n+1}b^0 + \sum_{k=1}^{n} C_{n+1}^k a^{n+1}b^{n+1-k} + C_{n+1}^0 a^0b^{n+1} = \sum_{k=0}^{n+1} C_{n+1}^ka^kb^{n+1-k}$

           \begin{align*}
               (a+b)^{n+1} &= (a+b)^n(a+b) = \left(\sum_{k=0}^{n} C_n^k a^k b^{n-k}\right) \cdot  (a+b)\\  
           .\end{align*}
           что и требовалось доказать
       \end{proof}
    \section{Метрические пространства}
    \begin{definition}
        $\sqsupset X$ -- любое множество, а $\rho: X\times X \to [0,+\infty )$

        Тогда пара $(X, \rho)$ называется метрическим пространствам, если функция $\phi$ удовлетворяет аксиомам метрики:
        \begin{enumerate}
            \item $\rho(x,y) = 0 \iff x = y$ (невырожденность)
            \item $\rho(x,y) = \rho(y,x)$ (симметричность)
            \item $\rho(x,z) \leqslant \rho(x,y) + \rho(y,z) \forall x, y, z\in X$ (неравенство треугольника)
        \end{enumerate}

        Тогда $\rho$ называется метрикой или расстоянием на $X$.
    \end{definition}
    \begin{example}
        \begin{enumerate}
            \item $(X, \rho_{D}$ -- метрическое пространство

                $\rho_D(x,y) = \begin{cases}
                    0&, \text{если} x=y\\
                    1&, \text{если} x\neq y\\
                \end{cases}$
            \item $X=\R, \rho(x,y) = |x-y|$

                $x-y = a, y-z = b\quad \rho(x,z) = |a-b|\leqslant |a| + |b| = \rho(x,y) + \rho(y,z)$

                Обычная или Евклидова метрика
            \item [$\overset{\sim }2$] $X = \C\quad \rho(z,w) = |z-w|$ (аксиома 3 будет проверена позже)
            \item [$\overset{\sim }{\overset{\sim }{2}}$]  $X = \R^n, x = (x_1, \ldots, x_{n} ), y = (y_1, \ldots, y_{n} )$ $\rho(x,y) = \sqrt{\sum_{k=1}^{n} \left( x_k-y_k \right) ^2} $

                $v = (v_1, \ldots, v_n)\quad \|v\| = \sqrt{\sum_{k=1}^{n} v_k^2} $ -- евклидова норма вектора $v$
            \item $\sqsupset (X, \rho)$ -- метрическое пространство

                $\sqsupset X_1\subseteq X\quad \rho_1 = \rho|_{X_1\times X_1}$

                $Тогда (X_1, \rho_1)$ -- есть метрическое пространство, а $\rho_1$ называется индуцированной метрикой.
            \item $X$ -- множество станций метрополитена г. Санкт-Петербурга. Пусть между соседними станциями расстояние -- 2 минуты. $\rho(u,v) = \min$ длин путей из $u$ в $v$

               $\rho$ -- метрика
        \end{enumerate}
    \end{example}

    \begin{definition}
        Открытый шар с центром в точке $a$ радиусом $R$ в метрическом пространстве $\left( X,\rho \right) :$

        $B_R(a) = \{x\in X:\quad\rho(X,a)<R\}$

        $B_R[a] = \{x\in X:\quad \rho\left(x, a)\leqslant R  \right) \}$
    \end{definition}
    \begin{example}
        \begin{enumerate}
            \item ($0 \vee 1$) $B_R(a) = \begin{cases}
                    \{a\} &, \text{если } R\leqslant 1\\
                    X&, \text{если } R>1\\
            \end{cases}$
        \item $(a-R,a+R)$
        \item круг (без окружности)
        \item $n$-мерный шар
        \end{enumerate}

        в $\R^n\quad \|v\|_1 = \sum_{k=1}^{n} |v_k|\quad \|v\|_{\infty } = \max\limits_{k=1:n} |v_k|$
    \end{example}
    \begin{definition}
        $E\subseteq \R, \begin{cases}
            M\in E\\
            \forall x\in E\quad 
            M\geqslant x\\
        \end{cases} \implies M:= \max E $
    \end{definition}

    $\rho(x,y) = \|x-y\|$

    $\rho_1(x,y) = \|x-y\|_1\qquad \rho_{\infty }(x,y) = \|x-y\|_{\infty }$

    Упражнение: Проверить, что $\rho_1, \rho_{\infty }$ -- метрики, нарисовать шар в $\R^2$ Относительно $\rho_1, \rho_{\infty }$

    \begin{definition}
        $\sqsupset (X, \rho)$ -- метрическое пространство

        $E \subseteq X, E $ Называется ограниченным, если  $$\exists a\in X, \exists R>0:\quad E\subseteq B_R(a) $$
    \end{definition}

    \begin{note}
        Эквивалентное определение: те же слова, но $B_R[a]$
    \end{note}

    \begin{definition}
        $\sqsupset  E \subseteq \R\quad E$ называется ограниченным сверху, если \[
        \exists m\in \R:\quad \forall x\in E\quad x\leqslant m
        .\] 

        При этом такое число $m$ называется \underline{мажорантой}

        Говорят: $m$ мажорирует $E$


        Аналогичное определение для ограниченности снизу. 
        Соответствующее~$m$ Называется \underline{минорантой}
    \end{definition}

    \begin{statement}
        $\sqsupset E\subseteq \R$

        \[
        E \text{ -- ограничено } \iff  \begin{cases}
            E \text{ --  ограничено сверху}\\
            E \text{ -- ограничено снизу}\\
        \end{cases}
        .\] 
    \end{statement}
    \begin{proof}
        $\implies $ По условию $\exists a: E\subseteq (a-R, a+R)$

        $M:=a+R$-- мажоранта $\implies E$ орграничено сверху. снизу -- аналогично    

        $\impliedby E$ -- ограничено сверху $\implies \exists M\in R: \forall x\in E\quad x\leqslant M$

        $\ldots \exists m\in \exists : \forall x\in E\quad x\geqslant m$

        $-x\leqslant -m \leqslant |m| \implies  |x| = max\{x, -x\}\leqslant max\{|M|, |m|\} = R$

        $\implies  x\in B_R[0]$. Т.к. это верно $\forall x\in E$, то $E \subseteq  B_R[0]$
    \end{proof}

    \begin{note}
        Если $E\subseteq \R$, то \[
            E \text{ -- ограничено} \iff  \exists R:\quad \forall x\in E\quad |x|\leqslant R
        .\] 
    \end{note}

    \begin{definition}
        $E\subseteq R, M\in E$, тогда \[
        M = \max E \iff  \forall x\in E\quad x\leqslant M
        .\] 

        $\min E$ аналогично
    \end{definition}

    \begin{statement}
        $\forall E\subseteq R:\quad E$ -- конечно и $E\neq \O \quad \exists \max E, \min E$
    \end{statement}
    \begin{definition}
        $E$ конечно, если $\exists  m\in \N $ и $\exists $ биекция $\phi:E \to \{1, 2, \ldots, n\}$
    \end{definition}
    \begin{proof}
        [доказательство Утверждения]
        По индукции по числу элементов в $E$

        База: $m=1\quad E = \{x\}\quad \max E = \min E  = x$

        Переход: $m\to m+1$

        Индукционное предаоложение: $\forall $ конечное множество из $M$ элементов имеет $\max$ и $\min$

        Пусть $E$ содержит $m+1$ элементов

        $E = \{x_1, \ldots, x_{m}, x_{m+1} \} = \overset{\sim }E \cup \{x_{m+1}\}$

        $M = \max \{\max \overset{\sim }E, x_{m+1}\}$

        $\begin{cases}
            M\in \overset{\sim }E&\subseteq E\\
            M = x_{m+1}&\in E
        \end{cases} \implies M\in E$

        $\begin{cases}
            M\geqslant x_{m+1}\\
            M\geqslant x \forall x\in \overset{\sim }E
        \end{cases} \implies  M\geqslant x \forall x\in E$, т.о. $M = \max E$
    \end{proof}

    \begin{corollary}
        $\sqsupset E \subseteq \Z , E $ -- ограничено сверху (снизу). \\Тогда $\exists \max(\min) E$
    \end{corollary}
    \begin{proof}
        По условию существует $M\in R: \forall x\in E\quad x\leqslant M, \overset{\sim }\sqsupset M \geqslant M$
        
        $\sqsupset n\in E\quad \sphericalangle \overset{\sim }E = \{x\in E: n\leqslant x\leqslant \overset{\sim }M\}$

        В $\overset{\sim }E$ не более $\overset{\sim }M - n+1$ элементов, оно конечно $\implies $ (по утверждению) $\exists \max \overset{\sim }E = C$

        $\forall x\in E^ x<n\vee x\geqslant n\qquad$

        $x<n\qquad n\in \overset{\sim }E \implies n\leqslant C \implies x\leqslant C$

        $x\geqslant n\qquad x\in \overset{\sim }E \implies x\leqslant C$
    \end{proof}
    \begin{corollary}
        $\sqsupset E\subseteq \N \quad E\neq \O $ Тогда $\exists \min E$

        (вытекает из следствия 1, т.к. $\N $ ограничено снизу)
    \end{corollary}

    $\left\lfloor x \right\rfloor$ -- целая часть числа. $\left\lfloor x \right\rfloor = \max\{k\in \Z :l\leqslant x\}$ (Существует по следствию 1)

    $\left\lfloor x \right\rfloor\leqslant x< \left\lfloor x \right\rfloor +1    $

    $x-1< \left\lfloor x \right\rfloor \leqslant x$

    \begin{statement}
        $\Q$ плотно в $\R$

        $\forall a, b\in \R, a<b\quad \exists c\in \Q\cap (a,b)$
    \end{statement}
    \begin{proof}
        $b-a>0 \implies \frac{1}{b-a}>0\quad\newline \exists N\in \N: N>\frac{1}{b-a} \iff  b-a>\frac{1}{N}$
        
        $c = \frac{\left\lfloor Na \right\rfloor+1}{N} \in \Q$

        $Na-1<\left\lfloor Na \right\rfloor\leqslant Na \implies  a = \frac{Na}{N}<c\leqslant \frac{Na+1}{N} = a + \frac{1}{N}<a+b-a = b$

        $\implies c\in (a,b)$


    \end{proof}

    \section{Равномощные множества}

    \begin{definition}
        $\sqsupset A,B$ -- множества.

        $A\sim B$ равномощны, если $\exists $ биекция между $A$ и $B$
    \end{definition}
    \begin{example}
        \begin{enumerate}
            \item $(a,b), a<b \sim (0,1)$

                $f(x) = a+(b-a)\cdot x, x\in (0,1)$
            \item [$\ov{1}$] $\forall (a,b)$ и $(c,d)$ равномощны
            \item $a<b \implies (a,b)\sim [a,b) \sim [a,b]$
            \item $\left( -\frac{\pi}{2}, \frac{\pi }{2} \right) \sim \R\quad (\tg)$
\begin{figure}[ht]
    \centering
    \incfig{circ}
    \caption{circ}
    \label{fig:circ}
\end{figure}
        \end{enumerate}
    \end{example}

    \begin{note}
        (равномощность) $\sim $ -- отношение эквивалентности.

        \begin{enumerate}
            \item $X\sim X\qquad id(x) \equiv x$ -- тождественное отображение $id_X$
            \item $X\sim Y \implies  Y\sim X$
            \item $X\sim Y\quad Y\sim Z \implies X\sim Z$
        \end{enumerate}
    \end{note}

    \begin{definition}
        Множество, равномощное $\N $ называется счётным
    \end{definition}
    \begin{example}
        \begin{itemize}
            \item $\{1, 4, 9, 16, \ldots\}$ -- счёттно. $f(x) = x^2$
            \item $\Z  = \{0,1,-1,2,-2,3,-3,4,-4\}$ -- считаем ихнатуральными числами  в таком порядке
            \item $\{m, m+1, m+2, \ldots\}, m\in \R\quad \phi(x) = m+x-1$
        \end{itemize}
    \end{example}
    \begin{theorem}
        Любое бесконечное множество содержит счётное подмножество
    \end{theorem}
    \begin{proof}
        $\sqsupset X$ -- бесконечное множество. $\implies  \exists a_1\in X\quad X\setminus \{a_1   \}\neq \O $ (Иначе $X = \{a_1\}$!!!) $\implies \exists a_2\in X\setminus \{a_1\}\quad a_2\neq a_1$

        Так можно продолжать для любого $n$. $X\setminus \{a_1, \ldots, a_n\}\neq \O $ (иначе $X$ конечно) $\implies \exists a_{n+1}\in X\setminus \{a_1, \ldots, a_n\}, \quad a{n+1}\not\in \{a_1, \ldots, a_n\}$

        $\forall n\in \N \quad \phi: n\to a_n\qquad A = \{a_n\}_{n\in \N }\quad \phi$ -- инъекция по построению. $A$ -- счётное
    \end{proof}

    \begin{definition}
        Если $X$ -- конечно $\vee X$ -- счётно, то $X$ называется \underline{не более чем счётным} (нбчс).
    \end{definition} 

    \begin{note}
        [уточнение понятие конечного]
        $X$ конечно $\iff \begin{cases}
            X\sim \{1, \ldots, n\}\\
            X = \O \\
        \end{cases}$

    \end{note}
    \begin{theorem}
        $\forall $ счётного $E$, если $X\subseteq E$, $X$ -- бсконечно, то $X$ --счётно.
    \end{theorem}
    \begin{note}
        Любое подмножество счётного не более чем счётно.
    \end{note}
    \begin{proof}
        $E$ -- счётно по условию $E = \{x_1, x_2, x_3, x_4, x_5 \ldots\}$

        В данном наборе есть элементы  из $X$. Пронумеруем их в порядке возникновения в наборе (*).
    \end{proof}
    \begin{theorem}
        Произведение счётных множеств счётно.

        $A,B$ -- счётны $\implies A\times B$ -- счётно
    \end{theorem}
    \begin{proof}
        Если $A = B = \N $ $\N ^2$ счётно

        $\begin{bmatrix} (1,1)&(1,2)&(1,2)&\ldots\\ (2,1)&(2,2)&(2,3)&\ldots\\ (3,1)&(3,2)&(3,3)&\ldots\\ \vdots& \vdots& \vdots& \\\end{bmatrix} $ -- нумеруем по диагоналям

        $A\times B = \{(a_K, b_j)\}_{k,j\in \N }\qquad l\to (k,j)\qquad =\{(k,j)_l\}_{l\in \N }$
    \end{proof}
    \begin{note}
        Любое конечное произведение $\underbrace{\N \times \N \times \ldots\times \N }_{m} = (\N ^m) \sim  \N $ не более чем счётно

    \end{note}
    \begin{theorem}
        Объединение счётного количества счётных множеств счётно.
    \end{theorem}
        $\{\{A_j\}_{j\in J}:\quad J \text{ -- не более чем счётно} \quad \forall j\in J\quad A_j \text{ не более чем счётно}\}$

        $\bigcup\limits_{j\in J} A_j$ -- не более чем счётно

        Не умаляя общности (н.у.о.) $J = \N \vee J = \{1, 2, \ldots, n\}$

        Элементы $A_1, A_2, \ldots$ (счётных!) множеств можно занумеровать.

        \[A_1:\quad a_{11}, a_{12} \ldots\]
        \[A_2:\quad a_{21}, a_{22} \ldots\]
        \[A_3:\quad a_{31}, a_{32} \ldots\]

        Перенумеруем по диагоналям лишь те, который встречаем в первый раз

    \begin{corollary}
        \begin{enumerate}
            \item 
                $\Q$ счётно. $\Q = \bigcup\limits_{N\in \N } \Q_N\qquad \Q_N = \{\frac{p}{n}\}_{p\in \Z }$
            \item $A = \{x: \exists \text{ многочлен с целыми коэффициенты } P(\cdot ):\quad P(x) = 0\}$

                $\mathds{P}_n = \{p(x) = a_0 + \ldots + a_nx^n:\quad a_0, \ldots, a_n\in \Z \}\quad \mathds{P}_n \leftrightarrow \Z ^{n+1}$

                $A_n = \{x:\exists p\in \mathds{P}_n: p(x) = 0\}\qquad A = \bigcup\limits_{n\in \N } A_n$
        \end{enumerate}

    \end{corollary}

    \begin{problem}
        $\N \times \N \times \ldots$ -- несчётно
    \end{problem}

    \begin{theorem}
        Сегмент несчётен ($\forall a, b:a<b\quad [a,b]$ -- не является счётным)
    \end{theorem}
    \begin{proof}
        Доказательство от противного. 

        $\sqsupset [a,b]$ -- счётен $\implies [a,b] = \{x_1, x_2, x_3, \ldots\}$

        $\sphericalangle$ три замкнутые ``трети'' $\Delta = b-a\qquad [a,a+\frac{\Delta}{3}], [a + \frac{\Delta}{3}, a + \frac{2\Delta}{3}], [a + \frac{2\Delta}{3}, b] $

        $x_1 \not\in $ одной из третей. Эту треть назовём $I_1$. Повторим действие для $I_1$ и $x_2$

        $I_2\subseteq I_1 \subseteq I_0 x_1\not\in I_1, x_2\not\in I_2$

        $I_n\subseteq I_{n-1}\subseteq  \ldots \subseteq  I_2\qquad x_{n} \not\in I_n $

        По аксиоме №16 $\bigcap\limits_{n\in \N } I_n \neq \O \quad \sqsupset x\in \bigcap\limits_{n\in N} I_n \implies c\in [a,b] \implies \exists n: c = x_n\not\in I_n \implies c\not\in \bigcap\limits_{n\in \N } I_n$ !!!

        Т.о. $[a,b]$ -- несчётно
    \end{proof}
    \begin{corollary}
        несчётные: $\R, \underset{a<b}{(a,b)}, \R\setminus \Q$

        $X\sim [0,1]$, то говорят, что $X$ -- мощности континуум (мощности $\C$)
    \end{corollary}
    \begin{problem}
         \begin{enumerate}
             \item 
        $\R\times\R\sim \R$
    \item $X$ -- множество, то $X\not\sim 2^X\qquad 2^X= \{A:A\subseteq X\}$

        $X = \O \quad 2^X = \{\O \}$

        $X = \{a\}\quad 2^X = \{\O , \{a\}\}$
    \item $\N ^{\N }\sim [0,1]$
         \end{enumerate}
    \end{problem}

    \begin{definition}
        $\sqsupset X$ -- любое множество
        Отображение из $\N $ в $X$ называется последовательностью в $X$

        вместо $f(n), n\in \N \quad f:\N \to X$ используют $\{x_n\}_{n=1}^{\infty }$ или $\left( x_n \right) _{n=1}^{\infty }\qquad n\to x_n\in X$
    \end{definition}

    \section{Предел числовой последовательности}

    \begin{definition}
        $\sqsupset \{x_n\}_{n=1}^{\infty }$ последовательность вещественных чисел. $x_{+}\in \R$

        \[
            \lim_{n \to \infty} x_n = x_+ \iff  \forall \varepsilon>0\exists N: \forall n>N \quad \left| x_n-x_+ \right| <\varepsilon
        .\]

        В метрическом пространстве $(X,\rho)$ шар $B_R(a)$ называется также $R$-окрестностью точки $a$
    \end{definition}

    \begin{definition}
        [Определение предела на языке окретсностей]

        $$\lim_{n \to \infty} x_n = x_* \iff \forall \text{ окрестности $U$ точки } x_*\quad \exists N\in \N : \forall n>N\quad x_{n} \in U$$
    \end{definition}
    \begin{example}
            $x_n = \frac{1}{n} \forall n\in \N $
                $x_* = 0$

            $\forall \varepsilon > 0 \exists N: \forall n>N\quad \left| \frac{1}{n} - 0 \right|  = \frac{1}{n}<\varepsilon\qquad n>\frac{1}{\varepsilon}\quad N:= \left\lfloor \frac{1}{\varepsilon} \right\rfloor +1$
    \end{example}
    \begin{note}
        Определение предела на ``языке окрестностей'' справедливо в случае последовательностей в метрическом пространстве

        $x_n \to x_* \iff  \forall \varepsilon >0 \exists N: \forall n>N\quad \rho(x_n, x_*)<\varepsilon$
    \end{note}
    \begin{statement}
        $\sqsupset x_n=c\quad\forall n\in \N , c\in X, X$ -- метрическое пространство $\implies \lim_{n \to \infty } x_n = c$
    \end{statement}
    \begin{proof}
        $x_* = c\quad \forall n\in \N \quad \rho(x_n, x_*) = 0<\varepsilon \forall \varepsilon > 0$

        $N=1$
    \end{proof}
    \begin{note}
        $\sqsupset \{x_n\}_{n=1}^{\infty }$ и $\{y_n\}_{n=1}^{\infty }$ -- последовательности в метрическом пространстве $X$ и $\exists m\in \N \quad x_n = y_n \forall n\geqslant m$. Тогда $\lim_{n \to \infty} x_n$ и $\lim_{n \to \infty} y_n$ совпадают (если существует один, то существует другой и равны при существовании)
    \end{note}
    \begin{statement}
        [единственность предела]

        $\sqsupset (X, \rho), \{x_n\}_{n=1}^{\infty }\subseteq X, y, z\in X $

        Если $x_n\to y$ и $x_n\to z$, то $y=z$
    \end{statement}
    \begin{proof}
        Если $y\neq z$, то $\rho(y,z) = \Delta>0\quad \varepsilon = \frac{\Delta}{2}$

        Т.к. $x_n \to y, x_{n} \to z$, то $\exists N_1, N_2:$

        $\forall n>N_1\quad \rho(x_n,y)<\varepsilon$

        $\forall n>N_2\quad \rho(x_n,z)<\varepsilon$

        $\forall n\geqslant \max\{N_1, N_2\}\quad \begin{cases}
            \rho(x_{n} , y)<\varepsilon \\
            \rho(x_{n} , z)<\varepsilon\\
        \end{cases} \implies  \Delta = \rho(y,z) \leqslant  \rho(y, x_{n}) + \rho(x_{n} , z)<2\varepsilon = \Delta \implies \Delta<\Delta $ !!!
    \end{proof}
    \begin{example}
        $x_n = (-1)^{-1} \forall n\in \N  \not\exists \lim_{n \to \infty} (-1)^{n-1}$

        Если бы $\exists x_* = \lim_{n \to \infty} (-1)^{n-1}$, то для $\varepsilon = 1 \exists N$

        $n=2N\quad \left| (-1)^{n-1} - x_* \right| = |-1-x_*|<1 $

        $n=2N+1\quad \left| (-1)^{n-1} - x_* \right| = |1-x_*|<1 $

    $2 = |1 - (-1)|\leqslant |1-x_* + x_*-(-1)| \leqslant |1-x_*| + |x_* - (-1)| <2$
    \end{example}
\end{document}




