\documentclass{book}
%nerd stuff here
\pdfminorversion=7
\pdfsuppresswarningpagegroup=1
% Languages support
\usepackage[utf8]{inputenc}
\usepackage[T2A]{fontenc}
\usepackage[english,russian]{babel}
% Some fancy symbols
\usepackage{textcomp}
\usepackage{stmaryrd}
% Math packages
\usepackage{amsmath, amssymb, amsthm, amsfonts, mathrsfs, dsfont, mathtools}
\usepackage[bb=boondox]{mathalfa}
\usepackage{cancel}
% Bold math
\usepackage{bm}
% Resizing
%\usepackage[left=2cm,right=2cm,top=2cm,bottom=2cm]{geometry}
% Optional font for not math-based subjects
%\usepackage{cmbright}

\author{Коченюк Анатолий}
\title{Математический анализ}

\usepackage{url}
% Fancier tables and lists
\usepackage{booktabs}
\usepackage{enumitem}
% Don't indent paragraphs, leave some space between them
\usepackage{parskip}
% Hide page number when page is empty
\usepackage{emptypage}
\usepackage{subcaption}
\usepackage{multicol}
\usepackage{xcolor}
% Some shortcuts
\newcommand\N{\ensuremath{\mathbb{N}}}
\newcommand\R{\ensuremath{\mathbb{R}}}
\newcommand\Z{\ensuremath{\mathbb{Z}}}
\renewcommand\O{\ensuremath{\emptyset}}
\newcommand\Q{\ensuremath{\mathbb{Q}}}
\renewcommand\C{\ensuremath{\mathbb{C}}}
\newcommand{\p}[1]{#1^{\prime}}
\newcommand{\pp}[1]{#1^{\prime\prime}}
\newcommand{\ov}[1]{\overline{#1}}
\renewcommand\phi{\varphi}
% Topology operators
\DeclareMathOperator{\Cl}{Cl}
\DeclareMathOperator{\Int}{Int}
\DeclareMathOperator{\Fr}{Fr}
% Easily typeset systems of equations (French package) [like cases, but it aligns everything]
\usepackage{systeme}
\usepackage{lipsum}
% limits are put below (optional for int)
\let\svlim\lim\def\lim{\svlim\limits}
%\let\svlim\int\def\int{\svlim\limits}
% Command for short corrections
% Usage: 1+1=\correct{3}{2}
\definecolor{correct}{HTML}{009900}
\newcommand\correct[2]{\ensuremath{\:}{\color{red}{#1}}\ensuremath{\to }{\color{correct}{#2}}\ensuremath{\:}}
\newcommand\green[1]{{\color{correct}{#1}}}
% Hide parts
\newcommand\hide[1]{}
% si unitx
\usepackage{siunitx}
\sisetup{locale = FR}
% Environments
% For box around Definition, Theorem, \ldots
\usepackage{mdframed}
\mdfsetup{skipabove=1em,skipbelow=0em}
\theoremstyle{definition}
\newmdtheoremenv[nobreak=true]{definition}{Определение}
\newmdtheoremenv[nobreak=true]{theorem}{Теорема}
\newmdtheoremenv[nobreak=true]{lemma}{Лемма}
\newmdtheoremenv[nobreak=true]{problem}{Задача}
\newmdtheoremenv[nobreak=true]{property}{Свойство}
\newmdtheoremenv[nobreak=true]{statement}{Утверждение}
\newmdtheoremenv[nobreak=true]{corollary}{Следствие}
\newtheorem*{note}{Замечание}
\newtheorem*{example}{Пример}
\renewcommand\qedsymbol{$\blacksquare$}
% Fix some spacing
% http://tex.stackexchange.com/questions/22119/how-can-i-change-the-spacing-before-theorems-with-amsthm
\makeatletter
\def\thm@space@setup{%
  \thm@preskip=\parskip \thm@postskip=0pt
}
\usepackage{xifthen}
\def\testdateparts#1{\dateparts#1\relax}
\def\dateparts#1 #2 #3 #4 #5\relax{
    \marginpar{\small\textsf{\mbox{#1 #2 #3 #5}}}
}

\def\@lecture{}%
\newcommand{\lecture}[3]{
    \ifthenelse{\isempty{#3}}{%
        \def\@lecture{Lecture #1}%
    }{%
        \def\@lecture{Lecture #1: #3}%
    }%
    \subsection*{\@lecture}
    \marginpar{\small\textsf{\mbox{#2}}}
}
% Todonotes and inline notes in fancy boxes
\usepackage{todonotes}
\usepackage{tcolorbox}

% Make boxes breakable
\tcbuselibrary{breakable}
\newenvironment{correction}{\begin{tcolorbox}[
    arc=0mm,
    colback=white,
    colframe=green!60!black,
    title=Correction,
    fonttitle=\sffamily,
    breakable
]}{\end{tcolorbox}}
% These are the fancy headers
\usepackage{fancyhdr}
\pagestyle{fancy}

% LE: left even
% RO: right odd
% CE, CO: center even, center odd
% My name for when I print my lecture notes to use for an open book exam.
% \fancyhead[LE,RO]{Gilles Castel}

\fancyhead[RO,LE]{\@lecture} % Right odd,  Left even
\fancyhead[RE,LO]{}          % Right even, Left odd

\fancyfoot[RO,LE]{\thepage}  % Right odd,something additional 1  Left even
\fancyfoot[RE,LO]{}          % Right even, Left odd
\fancyfoot[C]{\leftmark}     % Center

\usepackage{import}
\usepackage{xifthen}
\usepackage{pdfpages}
\usepackage{transparent}
\newcommand{\incfig}[1]{%
    \def\svgwidth{\columnwidth}
    \import{./figures/}{#1.pdf_tex}
}
\usepackage{tikz}
\usepackage{pgfplots}
\begin{document}
    \maketitle
    \tableofcontents
    \newpage
    \section{Введение}
    Преподаватель~--- Семёнова Ольга Львовна. 
    Почта: o\_semenova@mail.ru

   
   Литература: \begin{enumerate}
       \item Виноградов О.Л. Курс Математического анализа
       \item Виноградов, Громов --||--
       \item Фихтенгольц (курс)
       \item Зорич (курс, двухтомник)
       \item Кудрявцев (сборник задач, 1 том из трёх)
       \item Виноградова, Олехник, Саровничий (1 том из двух)
   \end{enumerate}

    \section{Баллы}
    Практика -- 70/100.
    Теория -- 30/100 -- 2-4 теста по теории (3 балла за присутствие на $\sim $всех лекциях.

    Если меньше 18/30 баллов, то всю теорию нужно будет пересдавать. 
    Иначе можно воспользоваться этим как баллами за экзамен.

    \chapter{Множества, отображения, $\R$}
    \section{Множества}
    ''Множество'' -- неопределямое слово. 
    Синонимы: набор, совокупность, класс.
    Множество состоит из элементов. 
    \[M = \{1,3,7,9\},\ \N,\ \Q,\ \Z,\ \Z_+ = \{0,1,2,3,4,\ldots\},\ \R,\ \R_+.\]

    Способы описания:
    \begin{itemize}
        \item явное описание $\{1, 2, 3\}$
        \item через некоторое свойство\\ 
        $M = \{x: P(x)\}\quad :$~--- читается как ''таких что''. 
        Тот же смысл имеет $\mid$.
        $P(x)$ обозначает какое-то свойство.\\
        $M = \{x:x\text{$x$ -- человек и $x$ 2002 г.р.}\}$
    \end{itemize}
    Кванторы:
    \begin{itemize}
        \item $\forall $ -- ''для любого'', любой, каждый, всякий $\ldots$
        \item $\exists $ -- ''существует''.
    \end{itemize}

    Пример: $\forall \varepsilon>0 \exists  \delta>0 :\ldots$\\
    Для любого положительного эпсилон существует положительное число дельта, т.ч. $\ldots$
   
    Обозначения:
    \begin{itemize}
        \item $\iff $~--- равносильно
        \item $\wedge$~--- ''и''
        \item $\vee$~--- ''или''
        \item $\sqsupset$~--- пусть
        \item $\sphericalangle$~--- допустим, рассмотрим
    \end{itemize}

    \begin{note}
        Множество всех множеств не существует.

        $\neg$ -- отрицание
        
        $\neg \exists $ -- не существует
        
        $\O $ -- пустое множество

        $x\in M \iff x$ -- элемент множества $M$

        $A\subseteq B \iff  (x\in A \implies  x\in B)$

        $B\supseteq A$ -- то же самое
    \end{note}
    $\forall $ множества $M\quad \O \subseteq M$

    $A = B \iff  \left( x\in A \iff x\in B \right)  \iff 
    \begin{cases}
        A\subseteq B\\
        B\subseteq A\\
    \end{cases}$
    
    
    $A,B$ -- множества 

    $A\cup B = \{x: (x\in A \vee x\in B)\}$
        
    $A\cap B = \{x: (x\in A \wedge x\in B)\}$

    $x\in A\cap B \iff  \begin{cases}
        x\in A\\
        x\in B\\
    \end{cases}$

    $A\setminus B = \{x:x \in A, x\not\in B \}$

    $A\subset C$

    $A^c = X\setminus A$ -- дополнение $A$ в $X$

    \begin{definition}
        $A,X_{\alpha}$ -- множества, $\forall \alpha \in A$

        $\{X_{\alpha}\}_{\alpha\in A}$ -- семейство множеств

        $A$ -- индексное множество

        $\bigcup\limits_{\alpha\in A}X_{\alpha} = \{x:\exists \alpha\in A\quad x\in X_{\alpha}\} $

        $\bigcap\limits_{\alpha\in A}X_{\alpha} = \{x: \forall \alpha\in A\quad x\in x_{\alpha}\} $
    \end{definition}

    \begin{example}
        $\{(x-1,x+1)\}_{x\in (0;1)}$

        $\bigcup\limits_{x\in (0;1)}(x-1, x+1) = (-1,2) , \bigcap\limits_{x\in (0,1)} (x-1,x+1) = (0;1) $
        \end{example}

    \begin{definition}
        [Формула Де Моргана]

        $A,B\subseteq X$

        $(A\cup B)^c = A^c\cap B^c$

        $(A\cap B)^c = A^c\cup B^c$
                

        $\{A_i\}$ -- семейство

        $(\bigcup\limits_{i\in I}A_i)^c  = \bigcap A_i^c$

        $\left( \bigcap\limits_{i \in  I} A_i \right) ^c = \bigcup A_i^c $
    \end{definition}
    \begin{note}
        $A^{c c} = A$ -- проверить-упражнение
    \end{note}

    \begin{proof}
        $x\in \left( \bigcup\limits_{i \in  I} A_i \right) ^c \iff  x\not\in \left( \bigcup\limits_{i \in  I} A_i \right) \iff \forall i\in I x\not\in A_i \iff  \forall i\in I x\in A_i^c \iff x\in \bigcap\limits_{i \in  I} A_i^c$

        $\left( \bigcap A_i \right) ^c = \left( \bigcap A_i^{c c} \right) ^c = \left( \bigcup A_i^c \right) ^{c c} = \bigcup A_i$
        \end{proof}

    \begin{definition}
        [упорядоченная пара]
        $A,B$

        $(a,b)$ -- упорядоченная пара, $a\in A, b\in B$. 
	    В этой паре важен порядок.

        $\{a,b\}$ -- неупорядоченная пара (двухэлементное множество), если $a\neq b$.

        $\{a,a\} = \{a\}$ (в множестве не различаются копии).
            
        Пример: координаты точек плоскости. 
    \end{definition}

        $X_1, \ldots, X_m\quad x_1\in X_1 \ldots x_m\in X_m\quad (x_{1}, \ldots, x_{m} )$ -- упорядоченная пара

        \begin{definition}
            [Декартово произведение]

            $X_1\times \ldots\times X_m = \{(x_1, \ldots, x_{m} ):x_k\in X_k\quad k = 1:m\}$

            $R^m = (R)^m$
        \end{definition}

        \begin{example}
            $X = \{1,2\}$

            $Y = \{3,5\}$

            $Z = \{0\}$

            $X\times Y\times Z = \{(1,3,0),(2,3,0),(2,3,0),(2,5,0)\}$
        \end{example}

        \section{Отображения}
        Формальное определение, которое не будет использовано или потребовано нигде (в том числе на экзамене).
        \begin{definition}
           $X,Y$ -- множества\\
           Если $R\subset X\times Y$ и $(x,y_1)\in R \vee (x,y_2)\in R \iff y_1=y_2$, R называется отображением или графиком.
    \end{definition}
    
    \begin{definition}
        Отображение -- это тройка $(X,Y,f)$, где $X,Y$ -- множества, а $f$ -- некое правило, по которому каждому элементу $x\in X$ сопоставляется некоторый единственный элемент $y\in Y$.
            
        $f:X\to Y$ -- синоним. читают ''$f$ действует из $X$ в $Y$''

        $X$ -- множество определения отображения

        $Y$ -- множество значений

        $\{y\in Y: \exists x\in X f(x) = y\}\subset Y$ (т.е. $Y$ -- необязательно точное множество значений)
    \end{definition}
    
    \begin{example}
        $x = \R,\ Y = \R, \quad x\mapsto x^2$

        Если $y=f(x)$, то $y$ называется образом элемента $x$ при отображении $f$.
    \end{example}

    $A\subseteq X\quad f(A) = \{f(x): x\in A\}$ -- образ множества $A$ под действием $f$.

    $B\subseteq Y\quad f^{-1}(B) = \{x:f(x)\in B\}$

    $f^{-1}(\{y\})$ -- необязательно одноэлементное.

    Упражнения:
    \begin{enumerate}
        \item $f(A\cup B), f(A)\cup f(B) $
        \item $f(A\cap B), f(A) \cap f(B)$
        $y\in f(A\cap B) \implies  \exists x\in A\cap B:f(x) = y. x\in A, x\in B, y\in f(A), y\in f(B) \implies  y\in f(A)\cap f(B)$

        $f(x) = const\quad f(A)\cap f(B)\neq \O , f(A\cap B) = \O$ , если $A\cap B = \O $
        \item $f^{-1}(A\cup B), f^{-1}(A)\cup f^{-1}(B)$
        \item $f(A\cap B), f(A) \cap f(B)$
    \end{enumerate}

    \begin{definition}
        Если $f:X\to Y,\ g:X_1\to Y\quad X_1\subseteq X $ и  $\forall x\in X_1 \quad g(x) = f(x)$, то $g$ называется сужением $f$ на $X_1$.

        Обозначение: $g = f\mid_{x_1}$.
        При этом $f$ называется продолжением $g$ с $X_1$ на $X$.
    \end{definition}

    \begin{figure}[h]
        \centering
        \begin{tikzpicture}
            \begin{axis}[
                axis equal,
                xmin= -1.57, xmax= 1.57,
                ymin= -1.5, ymax = 1.5,
                axis lines = middle,
            ]
            \addplot[domain=-1.57:1.57, samples=100,color=green]{sin(deg(x))};
            \end{axis}
        \end{tikzpicture}
        \caption{sinus}
        \label{sinus}
    \end{figure}

    \begin{example}
        $f(x) = \sin x\quad f\mid_{[-\frac{\pi}{2}, \frac{\pi}{2}]}$
    \end{example}
   
    \begin{definition}
        Если $f:X\to Y,\ g:Y\to Z$, то $g\circ f:X\to Z$.\\
        $g\circ f(x) = g(f(x))\ \forall x\in X$.\\
        $g\circ f$ называется композицией $f$ и $g$.
    \end{definition}
    \begin{example}
        Изобразить эскизы графиков функций для всех случаев
    \begin{enumerate}
        \item $f(x) = \sin  x, g(x) = x^2$
        \item $f(x) = x^2, g(x) = \sin  x$
        \item $f(x) = g(x)$

    \begin{tikzpicture}[scale=2]
   	    \draw[<->](0,1.5) -- (0,0) -- (1.5,0);
	    \draw (0,0) -- (0.5, 1);
	    \draw[dotted](0.5,0) -- (0.5,1);
	    \draw[dotted](0,1) -- (0.5,1);
	    \draw (0.5,0.5) -- (1,0);
	    \draw[dotted](0,0.5) -- (0.5,0.5);
	    \node[left] at (0,1) {1};
	    \node[below] at (1,0) {1};
	    \node[below] at (-0.1,0) {0};
	    \node[below] at (0.5,0) {$\frac{1}{2}$};
	    \node[left] at (0,0.5) {$\frac{1}{2}$};
    \end{tikzpicture}

    Построить $f^{(2)}, f^{(3)}, f^{(4)}$ без формул.
    \end{enumerate}
    \end{example}
    
    \begin{definition}
        Функция $f:X\to Y$ называется инъекцией, если 
        $\begin{cases}
        f(x_1) = y\\
        f(x_2) = y\\
        \end{cases} \implies  x_1=x_2$
    \end{definition}
    \begin{example}
    $f(x) = kx+b$ -- инъекция, $k\neq 0$

        $f(x) = \sin x$ -- не инъекция

    $f(x)\mid_{[-\frac{\pi}{2}, \frac{\pi}{2}]  }$ -- инъекция
    \end{example}
    
    \begin{definition}
        $f:X\to Y$ называется сюръекцией, если\\
        $\forall y\in Y\quad \exists x\in X:f(x) = y$.
    \end{definition}
    \begin{example}
        $\sin :\R\to \R$ -- не сюъекция

        $\sin :\R\to [-1,1]$ -- сюръекция

        $y = kx+b, k\neq 0$ -- сюръекцией
    \end{example}

    \begin{definition}[биективность]
        $f:X\to Y$ -- инъекция и сюръекция $\implies f$ называется биекцией
    \end{definition}

    \begin{example}
        $y = kx+b, k\neq 0$ -- биекция
    \end{example}

    \begin{definition}
        $f:X\to Y, g:Y\to X$.
        $g$ называется обратным к $f$ отображением, если $f(x) = y \iff  x = g(y)$.
        Обозначается: $g = f^{-1}$
    \end{definition}
    \begin{note}
        Обратимая функция должна быть биективной:
        \begin{enumerate}
            \item Инъективной -- обратная иначе не будет функцией.
            \item Сюръективной -- обратная иначе не будет определена на всём $Y$.
        \end{enumerate}
    \end{note}
    
    \begin{note}
        $f^{-1}(A)$ -- обычно прообраз $A$ под действием $f$, а не образ обратной функции (которая может не существовать).
    \end{note}

    \begin{example}
       $\arcsin = (\sin _{[-\frac{\pi}{2}, \frac{\pi}{2}]  })^{-1}$

       $\log_a(x) = (a^x)^{-1}$

       $\sqrt{x} = (x_{2}\mid_{[0,+\infty )} , \sqrt[3](x) = (x^3)^{-1} $
    \end{example}
    
    \section{Вещественные числа}
    \subsection{Аксиоматическое определение вещественных чисел}

    $(\R, +, \cdot, \leqslant )$ -- множество, две операции и отношение порядка, удовлетворяющие следующим 16 аксиомам:
        
        Аксиомы поля:
        \begin{enumerate}
            \item $\forall a, b\in \R\quad a+b=b+a$ (коммутативность сложения)
            \item $\forall a, b, c\in \R\quad (a+b)+c = a+(b+c)$ (ассоциативность сложения)
            \item $\exists $ нейтральный элемент $0$ по сложению $\forall a\in \R\quad a+0=a$ (существование нейтрального элемента по сложению)
            \item Существует обратный элемент по сложению.\\
            $\forall a\in \R\ \exists (-a)\in \R: \quad a+(-a) = 0$
            \item $\forall a, b\in \R\quad a\cdot b = b\cdot a$ (коммутативность умножения)
            \item $(a\cdot b)\cdot c = a\cdot (b\cdot c)$ (ассоциативность умножения)
            \item $\exists 1\in R\setminus \{0\}, \forall a\in \R\quad a\cdot 1 = a$
            \item $\forall a\neq 0\ \exists  a^{-1}\in \R: a\cdot a^{-1} = 1$
            \item $(a+b)\cdot c = a\cdot c + b\cdot c$ (дистрибутивность)
        \end{enumerate}
        примеры: $\R, \Q, \C, \{0,1\} (1+1=0$,  остальное как обычно$)$.

        Элементарные следствия:
        \begin{itemize}
            \item $\forall a\in K$ -- поля, обратный по сложению единственный. Если $b, \p b$ -- два обратных, то $b= b+(a+\p b)=(b+a)+\p b = \p b$
            \item обратный по умножению, нейтральные -- все единственны.
        \end{itemize}
        Аксиомы порядка:
        \begin{enumerate}
            \item $\forall a, b\in \R\quad a\leqslant b \vee b\leqslant a$
            \item $a\leqslant b, b\leqslant c \implies a\leqslant c$ (транзитивность)
            \item $a\leqslant b, b\leqslant a \implies a=b$ 
            \item $a\leqslant b, c\in \R\implies a+c\leqslant b+c$
            \item $a\geqslant 0, b\geqslant 0$, то $a\cdot b\geqslant 0$
        \end{enumerate}
        $0\cdot x + x\cdot x = (0+x)\cdot x = x\cdot x \implies  0\cdot x = 0$

        Упражнения:
        \begin{enumerate}
            \item $-x = (-1)\cdot x$
            \item $(-a)(-b) = a\cdot b$
            \item $1\geqslant 0$
        \end{enumerate}
        \begin{definition}
            Индуктивным множеством в упорядоченном поле $(K, +, \cdot , \leqslant )$ называется множество $N$:
           \begin{enumerate}
               \item $1\in N$
               \item $\forall x\in N \implies  x+1\in N$
           \end{enumerate}

           $\N $ -- наименьшее индуктивное множество. $\N  = \bigcap\limits_{N\text{ -- индуктивных}, N\subseteq \R}N $
        \end{definition}
        \begin{note}
            $x>b \iff \begin{cases}
                x\geqslant b\\
                x\neq b\\
            \end{cases}$
        \end{note}
        Аксиома Архимеда: $\forall x, y\in R: x>0, y>0\ \exists  n\in \N : nx>y$.

        Аксиома вложенных промежутков: 
        
        $\forall \{[a_n, b_n]\}_{n\in \N }: \forall n\in \N\ [a_{n+1}, b_{n+1}] \subseteq [a_n, b_n]\quad \bigcap\limits_{n\in \N } [a_n, b_n]\neq \O  $

        В аксиоме о вложенных промежутках предполагается, что\\
        $\forall  n\in \N , [a_n, b_n] \neq \O  \iff  a_n \leqslant  b_n$

        $[a,b] = \{x\in \R,: a\leqslant x\leqslant b\}$ -- замкнутый отрезок, промежуток, сегмент, замкнутый промежуток

        $(a,b) = \{x\in \R: a<x<b\}$ -- интервал, открытый промежуток

        $(a,b], [a,b)$ -- полуоткрытый промежуток

        $<a,b>$ -- некоторый промежуток $a\leqslant b, <a,b> \neq \O $

    \begin{note}
        [Расиширенная вещественная прямая]

        $\overline{R} = \R\cup \{+\infty \}\cup \{-\infty \}$

        $\forall a\in \R\quad a + (+\infty ) = (+\infty ) + a = +\infty $

        $a+(-\infty ) = (-\infty ) + a = -\infty $

        $(+\infty ) + (+\infty ) = +\infty $

        $(-\infty ) + (-\infty ) = -\infty $

        $(+\infty ) + (-\infty ), (-\infty ) + (+\infty )$ -- не определены

        $\forall a>0$
        \begin{itemize}
            \item $(+\infty )\cdot a = a\cdot (+\infty ) = +\infty $
            \item $(-\infty ) \cdot  a = a\cdot (-\infty ) = -\infty $
            \item $\pm \infty \cdot (-1) = \mp \infty $
            \item $(+\infty ) \cdot  (+\infty ) = (-\infty )\cdot (-\infty ) = +\infty $
            \item $(+\infty )\cdot (-\infty ) = (-\infty )\cdot (+\infty ) = -\infty $
            \item $(\pm \infty )\cdot 0, 0\cdot (\pm \infty )$ -- не определены
        \end{itemize}

        $\forall a\in \R$
        \begin{itemize}
            \item $+\infty \geqslant a\geqslant -\infty $
        \end{itemize}

        $[a,+\infty ] = [a,+\infty ) \cup \{+\infty \}$

        В ``$+\infty $'' иногда + опускают, но подразумевают её, если рассматривается $\ov R$
    \end{note}

    \section{Модуль}

    $a\in \R\qquad |a| = 
    \begin{cases}
        a&,\text{если} a\geqslant 0\\
        -a&, \text{если} a<0\\
    \end{cases}$

    \begin{property}
        $b=|a| \iff  
        \begin{cases}
            b\in \{a, -a\}\\
            b\geqslant 0\\
        \end{cases}$
    \end{property}

    Элементарные свойства модуля:
    \begin{enumerate}
        \item $\forall a\in \R\quad |-a| =|a|$
        \item $\forall a\in \R\quad \pm a \leqslant  |a|$
        \item $\forall  a, b\in \R\quad |a\cdot b| = |a|\cdot |b|$
        \item $\forall a\in \R, b\in \R\setminus \{0\}\quad(\left| \frac{a}{b} \right| ) = \frac{|a|}{|b|}$
        \item $\forall a, b\in \R\quad \left| |a| - |b| \right| \leqslant |a\pm b| \leqslant |a|+|b|$
    \end{enumerate}
    \begin{note}
        $a-b := a+(-b),\quad \frac{a}{b}:= a\cdot b^{-1}, b\neq 0$.
    \end{note}
    \begin{note}
        $a\leqslant b\quad\forall c\in \R\quad a+c\leqslant b+c$

        $a\leqslant b\quad b\leqslant c \implies a\leqslant c$

        $a\leqslant b, c\leqslant d\quad a+c\overset{?}{\leqslant} b+d$

        $a+c\leqslant b+c\quad b+c\leqslant b+d \implies a+c\leqslant b+d$
    \end{note}
    \begin{proof}
        $\forall a, b\in \R\quad \left| |a| - |b| \right| \leqslant |a\pm b| \leqslant |a|+|b|$

        $\pm a\leqslant |a|,\ \pm b\leqslant |b|\quad \pm(a+b)\leqslant |a| + |b| $ (аксиома порядка 4)

        $\implies |a+b|\leqslant |a| + |b| \implies |a-b|\leqslant |a| + |-b| = |a| + |b|$

        $|a| = |a-b+b| \leqslant |a-b| + |b|$

        $|a| - |b|\leqslant |a-b|\quad |b| - |a| \leqslant |b-a| = |a-b|$

        $\left| |a| - |b| \right| =  \pm (|a| - |b|) \leqslant |a-b|$
    \end{proof}
    \begin{note}
        $|+\infty | := +\infty \quad |-\infty | := +\infty $
    \end{note}

    \section{Комплексные числа}
    $\C$ -- обозначение для множества комплексных чисел. \\
    $\C = \{(x,y)|x\in \R, y\in \R\}$

    \begin{tikzpicture}
        \draw [<->] (0,1) -- (0,-1) -- (0,0) -- (-1,0) --  (1,0);
    \end{tikzpicture}
    Удобно представлять на плоскости.

    $(x_1,y_1) + (x_2,y_2) := (x_1+x_2,y_1+y_2)$

    $(x_1,y_1)\cdot (x_2,y_2) = (x_1x_2-y_1y_2,x_1y_2+x_2y_1) $
    \begin{note}
        $\C$ -- поле

        Аксиомы для сложения очевидны.

        $0 = (0,0), \quad 1 = (1,0)$

        $-(x,y) = (-x,-y)$

        $(x,y) \cdot  (1,0) = (x,y)\ \forall x, y\in \R$

        $i = (0,1)\quad i^2 = (0,1) \cdot  (0,1) = (-1,0)$
    \end{note}

    $\R \leftrightarrow \{(x,0) : x\in \R\}$ (именно такие пары, потому что так сохраняются операции)

    $F: \R \to \{(x,0)\}\quad F$ сохраняет + и сохраняет $\cdot$. 

    $(x_1,0) + (x_2,0) = (x_1+x_2,0) \quad  (x_1, 0) \cdot  (x_2,0) = (x_1x_2-0,0)$

    \begin{tikzpicture}
        \draw [<->] (0,1) -- (0,-1) -- (0,0) -- (-1,0) --  (1,0);
    \end{tikzpicture}

    Оси: вещественная(x) и мнимая(y)

    $(0,y)^2 = (-y^2,0)\ \forall y\in \R$

    $(x,y) = x+iy\quad i = (0,1)$ -- мнимая единица. 
    Алгебраическая форма записи комплексных чисел.

    $z = x+iy\quad x$ -- вещественная часть $z,\quad y$ -- мнимая часть $z$

    $Re z = x,\quad Im z = y\quad $ иногда встречается $rp, ip$ -- real/imaginary part.

    \begin{note}
        [Комплексное сопряжение]
        $z = x+iy\quad \ov z = x-iy$ -- отражённое от оси $x$, если смотреть на плоскость.

        $Re z = \frac{z+\ov z}{2}\qquad Im z = \frac{z-\ov z}{2i}$ -- вещественные числа!
    \end{note}

    \begin{note}
        [Модуль и аргумент]

        $|z| = \sqrt{z\cdot \ov z}  = \sqrt{x^2+y^2} $ 

        $z\cdot \ov z = (x+iy)\cdot (x-iy) = x^2-(iy)^2 = x^2+y^2$

        $r = |z|$

        Аргумент -- угол (ориентированный) между осью $Ox$ и $\overset{\to }{Oz}$.

        Аргументов много $Arg z, z\neq 0$ -- совокупность всех аргументов.

        Если $\varphi_0\in Arg (z)$, то $Arg z = \{\varphi_0 + 2\pi\cdot k, k\in \Z \}$.

        Если
        $\begin{cases}
            \varphi_0\in Arg z\\
            \varphi_0\in (-\pi , \pi ]\\
        \end{cases}$, 
        то  $\varphi_0$ называется главным значением аргумента $\varphi_0 = arg(z)$. 
    \end{note}

    $z = (x,y) = (r,\phi),r$ -- длина радиус-вектора, $\phi$ --  аргумент.

    $(r, \phi)$ -- полярные координаты, совмещённые с прямоугольными
    \begin{note}
        $r = \sqrt{x^2+y^2} $

        $x = r\cos \phi$

        $y = r\cos \phi$

        $x>0\quad arg z = \arctg \frac{y}{x} = \arcsin \frac{y}{\sqrt{x^2+y^2} }$

        $y>0\quad arg z = \arccos \frac{x}{\sqrt{x^2+y^2} }$

        $y<0\quad arg z = -\arccos \frac{x}{\sqrt{x^2+y^2} } $

        $\begin{cases}
            x<0\\
            y>0\\
        \end{cases}
        \quad arg z = \arctg \frac{y}{x} + \pi$

        остальное -- упражнение
    \end{note}

    Изобразить кривую заданную в полярных координатах: 
    \begin{enumerate}
        \item $r = 3$
        \item $r = \phi$ -- спираль Архимеда
        \item $r = e^{\phi}$
        \item $r = \frac{1}{\cos \phi}$
        \item $r = \frac{2}{\sin \phi}$
        \item $r = \frac{3}{\cos \phi + \sin \phi}$
        \item $r = 1+\cos \phi$
    \end{enumerate}

    $(0,0)$ -- полюс

    $r(\phi) \uparrow$ -- удаление от полюса

    $z = x+iy = r(\cos \phi + i\sin \phi)$ -- в скобках точка на единичной окружности с аргументом таким же, что и у $z$.

    Это называется тригонометрической формой записи числа.

    $-\frac{1}{2}(\cos \phi_0, \sin \phi_0) = \frac{1}{2} \cdot  \left( \cos (\phi_0+\pi ) + i\sin (\phi_0+\pi ) \right) $

    $e^{i\phi}:=\cos \phi + i\sin\phi\quad \phi\in \R$

    $r\cdot e^{i\phi}$ -- экспоненциальная (показательная) форма числа.

    $\cos \phi = \frac{e^{i\phi} + e^{-i\phi}}{2}\quad \sin \phi = \frac{e^{i\phi} + e^{-i\phi}}{2i}$

    Если $z_1 = r_1 \cdot  e^{i\phi_1}, z_2 = r_2 \cdot  e^{i\phi_2}$, то $z_1\cdot z_2 = r_1r_2\cdot e^{i\left( \phi_1 + \phi_2 \right) }$ (см. курс алгебра).

    $n\in \N \quad z^n = r^n \cdot  (\cos (n\phi) + i\sin(n\phi))$ -- формула Муавра.

    \section{Дополнение к разделу\\``Действия над множествами''}

    \begin{statement}
        $\sqsupset B$~--- множество, $\{A_i\}_{i\in I}$~--- семейство множеств. \\
        $B\cap \left( \bigcup\limits_{i \in  I} A_i \right)  = \bigcup\limits_{i \in  I} \left( B\cap A_i \right) $
    \end{statement}
    \begin{proof}
        $x\in B\cap \left( \bigcup\limits_{i \in  I} A_i \right) \iff \begin{cases}
            x\in B\\
            x\in \cup A_i\\
        \end{cases} \iff  
        \begin{cases}
            x\in B\\
            \exists i:x\in A_i\\
        \end{cases} \iff  \exists i: 
        \begin{cases}
            x\in B\\
            x\in A_i\\
        \end{cases} \iff \exists i: x\in B\cap A_i \iff x\in  \bigcup\limits_{i \in  I} \left( B\cap A_i \right)$.
        \end{proof}

    \section{Принцип математической индукции}

    $P_n$ - утверждение, зависящее от $n$

    $\begin{cases}
        P_1 \text{-- верно}\\
        P_n \to P_{n+1}\\
    \end{cases}\implies \forall n\in \N\quad P_n$ -- верно.

    $\{n: P_n \text{-- верно}\}$ -- индуктивно $\implies  \N \subseteq \{n:P_n \text{ -- верно}\}$

    Первый шаг (проверка $P_1$) называется базой индукции, а второй -- переходом

    \begin{example}
        $2^n\geqslant n^2\quad \forall n\geqslant 4, n\in \N $

        $P_4\quad 2^4\geqslant 4^2\quad 16\geqslant 16$ -- верно

        $\sqsupset P_n$ -- верно

        $P_{n+1}:\quad 2^{n+1}\geqslant (n+1)^2$

        $2^{n+1} = 2^n\cdot 2 \geqslant  n^2\cdot 2 \overset{?}{\geqslant} (n+1)^2 \iff \left( \frac{n+1}{n} \right)^2 \leqslant 2  $

        $\frac{n+1}{n} = 1+\frac{1}{n}\leqslant 1+\frac{1}{4}$

        $\left( 1+\frac{1}{n} \right) ^2 \leqslant \left( 1+\frac{1}{4} \right) ^2 = \frac{25}{16}\leqslant 2$
    \end{example}


    \begin{definition}
        $\forall n\in \N \quad n! := 1\cdot 2\cdot \ldots\cdot n$

        $0! := 1$ -- соглашение

        $(n+k)! = n!\cdot (n+1) \cdot  \ldots \cdot  (n+k)$

        $n!! = n\cdot (n-2) \cdot  \ldots$ (заканчивается либо 1, либо 2)

        $n$ -- чётно, $n!! = 2\cdot 4\cdot \ldots\cdot n$

        $n$ -- нечётно, $n!! = 1\cdot 3\cdot 5\cdot \ldots\cdot n$
    \end{definition}

    \begin{definition}
        [биноминальный коэффициент]

        $C_n^k = \frac{n!}{k!(n-k)!}$ -- биномиальный коэффициент, число сочетаний из $n$ по $k$

        $\begin{pmatrix} n\\k \end{pmatrix} $
    \end{definition}
    Элементарные свойства биномиальных кэффициентов:
       \begin{enumerate}
           \item $C_n^k = C_n^{n-k}$
           \item $C_n^0 = C_n^n = 1$
           \item $C_n^1 = C_n^{n-1} = n$
           \item $C_n^k + C_n^{k-1} = C_{n+1}^k$

               $\frac{n!}{k!(n-k)!} + \frac{n!}{(k-1)!\cdot (n-k+1)!} = \frac{n!}{k!(n+1-k)!} \cdot  (n+1-k+k) = C_{n+1}^k$ 
       \end{enumerate}

       $$1$$ $$1\quad 1$$ $$1\quad 2\quad 1$$ $$1\quad 3\quad 3\quad 1$$ $$1\quad 4\quad 6\quad 4\quad 1$$

       \begin{statement}
           $\forall a, b\in \C\quad \forall n\in \Z _+\quad (a+b)^n = \sum_{k=0}^{n} C_n^ka^kb^{n-k}$ -- бином Ньютона.
       \end{statement}

       \begin{note}
           $\sum_{k=1}^{N} a_k := a_1 + a_2 + \ldots + a_N$

           $\sum_{k=m}^{m+p} a_k := a_m + a_{m+1} + \ldots + a_{m+p}$

           $\prod_{k=m}^{m+p} a_k := a_m\cdot a_{m+1}\cdot \ldots\cdot a_{m+p} $
        \end{note}
        \begin{note}
            $x^0:=1\ \forall x\in \C $ -- определили функцию
        \end{note}
        \begin{proof}
            [Доказательство бинома по индукции]
            База: $n=1\quad (a+b)^1 = \sum_{k=0}^{1} C_1^k a^kb^{1-k} = C_1^0 a^0b^1 + C_1^1a^1b^0 = a+b$

            Переход: Пусть верно для $n$. Докажем для $n+1$:
            \begin{equation*}
                \begin{split}
                (a+b)^{n+1} &= (a+b)^n(a+b) =
                \left(\sum_{k=0}^{n} C_n^k a^k b^{n-k}\right) \cdot  (a+b) =\\ &=\sum_{k=0}^{n} C_n^k a^{k+1}b^{n-k} + \sum_{k=0}^{n} C_n^ka^kb^{n-k+1} \underset{(j=k+1)} =\\
                &=\sum_{j=1}^{n+1} C_n^{j-1}a^jb^{n+1-j} + \sum_{k=0}^{n} C_n^k a^kb^{n+1-k} \underset{k=j}=\\
                &=\sum_{k=1}^{n+1} C_n^{k-1} a^k b^{n+1-k} + \sum_{k=0}^{n} C_n^ka^kb^{n+1-k}  =\\ 
                &=C_{n}^na^{n+1}b^0 + \sum_{k=1}^{n}\left( C_n^{k-1}a^kb^{n+1-k} + C_n^k a^kb^{n+1-k}\right) + C_n^0a^0b^{n+1}=\\ 
                &= C_{n+1}^{n+1}a^{n+1}b^0 + \sum_{k=1}^{n} C_{n+1}^k a^{n+1}b^{n+1-k} + C_{n+1}^0 a^0b^{n+1} = \sum_{k=0}^{n+1} C_{n+1}^ka^kb^{n+1-k}
               \end{split}
           \end{equation*}
           

           %\begin{align*}
            %   (a+b)^{n+1} &= (a+b)^n(a+b) = \left(\sum_{k=0}^{n} C_n^k a^k b^{n-k}\right) \cdot  (a+b)\\  
           %\end{align*}
           что и требовалось доказать
       \end{proof}
    \section{Метрические пространства}
    \begin{definition}
        $\sqsupset X$ -- любое множество, а $\rho: X\times X \to [0,+\infty )$

        Тогда пара $(X, \rho)$ называется метрическим пространством, если функция $\phi$ удовлетворяет аксиомам метрики:
        \begin{enumerate}
            \item $\rho(x,y) = 0 \iff x = y$ (невырожденность)
            \item $\rho(x,y) = \rho(y,x)$ (симметричность)
            \item $\rho(x,z) \leqslant \rho(x,y) + \rho(y,z) \forall x, y, z\in X$ (неравенство треугольника)
        \end{enumerate}

        Тогда $\rho$ называется метрикой или расстоянием на $X$.
    \end{definition}
    \begin{example}
        \begin{enumerate}
            \item $(X, \rho_{D})$ -- метрическое пространство

            $\rho_D(x,y) = \begin{cases}
                0&, \text{если} x=y\\
                1&, \text{если} x\neq y\\
            \end{cases}$
            \item $X=\R, \rho(x,y) = |x-y|$

            $x-y = a, y-z = b\quad \rho(x,z) = |a-b|\leqslant |a| + |b| = \rho(x,y) + \rho(y,z)$

            Обычная или Евклидова метрика
            \item [$\overset{\sim }2$] $X = \C\quad \rho(z,w) = |z-w|$ (аксиома 3 будет проверена позже)
            \item [$\overset{\approx }{2}$]  $X = \R^n, x = (x_1, \ldots, x_{n} ), y = (y_1, \ldots, y_{n} )$ $\rho(x,y) = \sqrt{\sum_{k=1}^{n} \left( x_k-y_k \right) ^2}$.

            $v = (v_1, \ldots, v_n)\quad \|v\| = \sqrt{\sum_{k=1}^{n} v_k^2} $ -- евклидова норма вектора $v$.
            \item $\sqsupset (X, \rho)$ -- метрическое пространство

            $\sqsupset X_1\subseteq X\quad \rho_1 = \rho|_{X_1\times X_1}$

            Тогда $(X_1, \rho_1)$ -- есть метрическое пространство, а $\rho_1$ называется индуцированной метрикой.
            \item $X$ -- множество станций метрополитена г. Санкт-Петербурга.
            Пусть между соседними станциями расстояние -- 2 минуты. $\rho(u,v) = \min$ длин путей из $u$ в $v$

            $\rho$ -- метрика
        \end{enumerate}
    \end{example}

    \begin{definition}
        Открытый шар с центром в точке $a$ радиусом $R$ в метрическом пространстве $\left( X,\rho \right) :$

        $B_R(a) = \{x\in X:\quad\rho(X,a)<R\}$

        $B_R[a] = \{x\in X:\quad \rho\left(x, a)\leqslant R  \right) \}$
    \end{definition}
    \begin{example}
        \begin{enumerate}
            \item ($0 \vee 1$) $B_R(a) = 
            \begin{cases}
                \{a\} &, \text{если } R\leqslant 1\\
                X&, \text{если } R>1\\
            \end{cases}$
            \item $(a-R,a+R)$
            \item круг (без окружности)
            \item $n$-мерный шар
        \end{enumerate}

        в $\R^n\quad \|v\|_1 = \sum_{k=1}^{n} |v_k|\quad \|v\|_{\infty } = \max\limits_{k=1:n} |v_k|$
    \end{example}
    \begin{definition}
        $E\subseteq \R, \begin{cases}
            M\in E\\
            \forall x\in E\quad 
            M\geqslant x\\
        \end{cases} \implies M:= \max E $
    \end{definition}

    $\rho(x,y) = \|x-y\|$

    $\rho_1(x,y) = \|x-y\|_1\qquad \rho_{\infty }(x,y) = \|x-y\|_{\infty }$

    Упражнение: проверить, что $\rho_1, \rho_{\infty}$ -- метрики, нарисовать шар в $\R^2$ относительно $\rho_1, \rho_{\infty}$.

    \begin{definition}
        Пусть $ (X, \rho)$ -- метрическое пространство.
        $E \subseteq X, E $ называется ограниченным, если  
        $$\exists a\in X, \exists R>0:\quad E\subseteq B_R(a) $$
    \end{definition}

    \begin{note}
        Эквивалентное определение: те же слова, но $B_R[a]$
    \end{note}

    \begin{definition}
        Пусть $E \subseteq \R$. 
        $E$ называется ограниченным сверху, если 
        \[ \exists m\in \R:\quad \forall x\in E\quad x\leqslant m.\] 

        При этом такое число $m$ называется \underline{мажорантой}.
        Говорят: $m$ мажорирует $E$

        Аналогичное определение для ограниченности снизу. 
        Соответствующее~$m$ называется \underline{минорантой}.
    \end{definition}

    \begin{statement}
        Пусть $E\subseteq \R$.
        \[
        E \text{ -- ограничено } \iff  \begin{cases}
            E \text{ --  ограничено сверху}\\
            E \text{ -- ограничено снизу}\\
        \end{cases}
        .\] 
    \end{statement}
    \begin{proof}
    \begin{itemize}
        \item[$\implies:$]
            по условию $\exists a: E\subseteq (a-R, a+R)$.

            $M:=a+R$-- мажоранта $\implies E$ ограничено сверху. 
            Снизу -- аналогично.
        
            \item[$\impliedby:$]
            $E$ -- ограничено сверху $\implies \exists M\in R: \forall x\in E\quad x\leqslant M$.

            $\ldots \exists m\in \exists : \forall x\in E\quad x\geqslant m$

            $-x\leqslant -m \leqslant |m| \implies  |x| = max\{x, -x\}\leqslant max\{|M|, |m|\} = R$

            $\implies  x\in B_R[0]$. Т.к. это верно $\forall x\in E$, то $E    \subseteq  B_R[0]$
        \end{itemize}
    \end{proof}

    \begin{note}
        Если $E\subseteq \R$, то 
        \[
            E \text{ -- ограничено} \iff  \exists R:\quad \forall x\in E\quad |x|\leqslant R.
        \] 
    \end{note}

    \begin{definition}
        $E\subseteq R, M\in E$, тогда 
        \[ M = \max E \iff  \forall x\in E\quad x\leqslant M. \] 

        $\min E$ аналогично.
    \end{definition}

    \begin{statement}
        $\forall E\subseteq R:\ E$ -- конечно и $E\neq \O \implies \exists \max E, \min E$
    \end{statement}
    \begin{definition}
        $E$ конечно, если $\exists  m\in \N $ и $\exists $ биекция $\phi:E \to \{1, 2, \ldots, n\}$
    \end{definition}
    \begin{proof}
        [Доказательство утверждения]
        Индукцией по числу элементов в $E$.

        База: $m=1\quad E = \{x\}\quad \max E = \min E  = x$

        Переход: $m\to m+1$

        Индукционное предположение: любое конечное множество из $M$ элементов имеет $\max$ и $\min$.

        Пусть $E$ содержит $m+1$ элементов.
        $E = \{x_1, \ldots, x_{m}, x_{m+1} \} = \overset{\sim }E \cup \{x_{m+1}\}$.

        $M = \max \{\max \overset{\sim }E, x_{m+1}\}$.

        $\begin{cases}
            M\in \overset{\sim }E&\subseteq E\\
            M = x_{m+1}&\in E
        \end{cases} \implies M\in E$

        $\begin{cases}
            M\geqslant x_{m+1}\\
            M\geqslant x \forall x\in \overset{\sim }E
        \end{cases} \implies  M\geqslant x\ \forall x\in E$, т.о. $M = \max E$
    \end{proof}

    \begin{corollary}
        $\sqsupset E \subseteq \Z , E $ -- ограничено сверху (снизу). \\Тогда $\exists \max(\min) E$
    \end{corollary}
    \begin{proof}
        По условию существует $M\in R: \forall x\in E\quad x\leqslant M, \overset{\sim }\sqsupset M \geqslant M$.
        
        $\sqsupset n\in E\quad \sphericalangle \overset{\sim }E = \{x\in E: n\leqslant x\leqslant \overset{\sim }M\}$

        В $\overset{\sim }E$ не более $\overset{\sim }M - n+1$ элементов, оно конечно $\implies $ (по утверждению) $\exists \max \overset{\sim }E = C$

        $\forall x\in E^ x<n\vee x\geqslant n\qquad$

        $x<n\qquad n\in \overset{\sim }E \implies n\leqslant C \implies x\leqslant C$

        $x\geqslant n\qquad x\in \overset{\sim }E \implies x\leqslant C$
    \end{proof}
    \begin{corollary}
        $\sqsupset E\subseteq \N \quad E\neq \O $ Тогда $\exists \min E$

        (вытекает из следствия 1, т.к. $\N $ ограничено снизу)
    \end{corollary}

    $\left\lfloor x \right\rfloor$ -- целая часть числа. $\left\lfloor x \right\rfloor = \max\{k\in \Z :l\leqslant x\}$ (Существует по следствию 1)

    $\left\lfloor x \right\rfloor\leqslant x< \left\lfloor x \right\rfloor +1    $

    $x-1< \left\lfloor x \right\rfloor \leqslant x$

    \begin{statement}
        $\Q$ плотно в $\R$

        $\forall a, b\in \R, a<b\quad \exists c\in \Q\cap (a,b)$
    \end{statement}
    \begin{proof}
        $b-a>0 \implies \frac{1}{b-a}>0\quad\newline \exists N\in \N: N>\frac{1}{b-a} \iff  b-a>\frac{1}{N}$
        
        $c = \frac{\left\lfloor Na \right\rfloor+1}{N} \in \Q$

        $Na-1<\left\lfloor Na \right\rfloor\leqslant Na \implies  a = \frac{Na}{N}<c\leqslant \frac{Na+1}{N} = a + \frac{1}{N}<a+b-a = b$

        $\implies c\in (a,b)$


    \end{proof}

    \section{Равномощные множества}

    \begin{definition}
        Пусть $A,B$ -- множества.
        $A$ равномощно $B$, если $\exists $ биекция между $A$ и $B$. 
        Пишут  $A\sim B$.
    \end{definition}
    \begin{example}
        \begin{enumerate}
            \item $(a,b), a<b \sim (0,1)$
            $f(x) = a+(b-a)\cdot x, x\in (0,1)$
            \item[$\ov{1}$] $\forall (a,b)$ и $(c,d)$ равномощны
            \item $a<b \implies (a,b)\sim [a,b) \sim [a,b]$
            \item $\left( -\frac{\pi}{2}, \frac{\pi }{2} \right) \sim \R\quad (\tg)$
            \begin{figure}[ht]
                \centering
                \incfig{circ}
        	\caption{circ}
        	\label{fig:circ}
    	    \end{figure}
        \end{enumerate}
    \end{example}

    \begin{note}
        (равномощность) $\sim $ -- отношение эквивалентности.

        \begin{enumerate}
            \item $X\sim X\qquad id(x) \equiv x$ -- тождественное отображение $id_X$
            \item $X\sim Y \implies  Y\sim X$
            \item $X\sim Y\quad Y\sim Z \implies X\sim Z$
        \end{enumerate}
    \end{note}

    \begin{definition}
        Множество, равномощное $\N$, называется счётным
    \end{definition}
    \begin{example}
        \begin{itemize}
            \item $\{1, 4, 9, 16, \ldots\}$ -- счётно. 
            $f(x) = x^2$
            \item $\Z  = \{0,1,-1,2,-2,3,-3,4,-4\}$ -- считаем их натуральными числами  в таком порядке.
            \item $\{m, m+1, m+2, \ldots\}, m\in \R\quad \phi(x) = m+x-1$
        \end{itemize}
    \end{example}
    \begin{theorem}
        Любое бесконечное множество содержит счётное подмножество.
    \end{theorem}
    \begin{proof}
        $\sqsupset X$ -- бесконечное множество. $\implies  \exists a_1\in X\quad X\setminus \{a_1   \}\neq \O $ (Иначе $X = \{a_1\}$!!!) $\implies \exists a_2\in X\setminus \{a_1\}\quad a_2\neq a_1$

        Так можно продолжать для любого $n$. $X\setminus \{a_1, \ldots, a_n\}\neq \O $ (иначе $X$ конечно) $\implies \exists a_{n+1}\in X\setminus \{a_1, \ldots, a_n\}, \quad a{n+1}\not\in \{a_1, \ldots, a_n\}$

        $\forall n\in \N \quad \phi: n\to a_n\qquad A = \{a_n\}_{n\in \N }\quad \phi$ -- инъекция по построению. $A$ -- счётное
    \end{proof}

    \begin{definition}
        Если $X$ -- конечно $\vee X$ -- счётно, то $X$ называется \underline{не более чем счётным} (нбчс).
    \end{definition} 

    \begin{note}
        [уточнение понятие конечного]~\\
        $X$ конечно $\iff \begin{cases}
            X\sim \{1, \ldots, n\}\\
            X = \O \\
        \end{cases}$

    \end{note}
    \begin{theorem}
        $\forall $ счётного $E$, если $X\subseteq E$, $X$ -- бсконечно, то $X$ --счётно.
    \end{theorem}
    \begin{note}
        Любое подмножество счётного не более чем счётно.
    \end{note}
    \begin{proof}
        $E$ -- счётно по условию $E = \{x_1, x_2, x_3, x_4, x_5 \ldots\}$.
        В данном наборе есть элементы  из $X$. Пронумеруем их в порядке возникновения в наборе (*).
    \end{proof}
    \begin{theorem}
        Произведение счётных множеств счётно.\\
        $A,B$ -- счётны $\implies A\times B$ -- счётно
    \end{theorem}
    \begin{proof}
        Если $A = B = \N$, то $\N ^2$ счётно.

        $\begin{bmatrix} (1,1)&(1,2)&(1,2)&\ldots\\ (2,1)&(2,2)&(2,3)&\ldots\\ (3,1)&(3,2)&(3,3)&\ldots\\ \vdots& \vdots& \vdots& \\ \end{bmatrix}$~--- нумеруем по диагоналям.

        $A\times B = \{(a_K, b_j)\}_{k,j\in \N }\qquad l\to (k,j)\qquad =\{(k,j)_l\}_{l\in \N }$
    \end{proof}
    \begin{note}
        Любое конечное произведение $\underbrace{\N \times \N \times \ldots\times \N }_{m} = (\N ^m) \sim  \N $ не более чем счётно

    \end{note}
    \begin{theorem}
        Объединение счётного количества счётных множеств счётно.
    \end{theorem}
        $\{\{A_j\}_{j\in J}:\quad J \text{ -- не более чем счётно} \quad \forall j\in J\quad A_j \text{ не более чем счётно}\}$

        $\bigcup\limits_{j\in J} A_j$~--- не более чем счётно.
        Не умаляя общности (н.у.о.)\\
        $J = \N \vee J = \{1, 2, \ldots, n\}$

        Элементы $A_1, A_2, \ldots$ (счётных!) множеств можно занумеровать.
        \begin{align*}
            A_1:\quad a_{11}, a_{12} \ldots\\
            A_2:\quad a_{21}, a_{22} \ldots\\
            A_3:\quad a_{31}, a_{32} \ldots\\
        \end{align*}
        Перенумеруем по диагоналям лишь те, который встречаем в первый раз

    \begin{corollary}
        \begin{enumerate}
            \item 
            $\Q$ счётно. 
            $\Q = \bigcup\limits_{N\in \N } \Q_N\qquad \Q_N = \{\frac{p}{n}\}_{p\in \Z }$
            \item $A = \{x: \exists \text{ полином с целыми коэффициентами } P(\cdot):\ P(x) = 0\}$

            $\mathds{P}_n = \{p(x) = a_0 + \ldots + a_nx^n:\quad a_0, \ldots, a_n\in \Z \}\quad \mathds{P}_n \leftrightarrow \Z ^{n+1}$

            $A_n = \{x:\exists p\in \mathds{P}_n: p(x) = 0\}\qquad A = \bigcup\limits_{n\in \N } A_n$
        \end{enumerate}

    \end{corollary}

    \begin{problem}
        $\N \times \N \times \ldots$ -- несчётно
    \end{problem}

    \begin{theorem}
        Сегмент несчётен ($\forall a, b:a<b\quad [a,b]$ -- не является счётным)
    \end{theorem}
    \begin{proof}
        Доказательство от противного. 

        $\sqsupset [a,b]$ -- счётен $\implies [a,b] = \{x_1, x_2, x_3, \ldots\}$.

        $\sphericalangle$ три замкнутые ``трети'' $\Delta = b-a\qquad [a,a+\frac{\Delta}{3}], [a + \frac{\Delta}{3}, a + \frac{2\Delta}{3}], [a + \frac{2\Delta}{3}, b] $

        $x_1 \not\in $ одной из третей. 
        Эту треть назовём $I_1$. 
        Повторим действие для $I_1$ и $x_2$

        $I_2\subseteq I_1 \subseteq I_0 x_1\not\in I_1, x_2\not\in I_2$

        $I_n\subseteq I_{n-1}\subseteq  \ldots \subseteq  I_2\qquad x_{n} \not\in I_n $

        По аксиоме № 16 $\bigcap\limits_{n\in \N } I_n \neq \O \quad \sqsupset x\in \bigcap\limits_{n\in N} I_n \implies c\in [a,b] \implies \exists n: c = x_n\not\in I_n \implies c\not\in \bigcap\limits_{n\in \N } I_n$ !!!

        Т.о. $[a,b]$ -- несчётно
    \end{proof}
    \begin{corollary}
        несчётные: $\R, \underset{a<b}{(a,b)}, \R\setminus \Q$

        $X\sim [0,1]$, то говорят, что $X$ -- мощности континуум (мощности $\C$)
    \end{corollary}
    \begin{problem}
        \begin{enumerate}
            \item $\R\times\R\sim \R$
            \item Если $X$ -- множество, то $X\not\sim 2^X\qquad 2^X= \{A:A\subseteq X\}$

            $X = \O \quad 2^X = \{\O \}$

            $X = \{a\}\quad 2^X = \{\O , \{a\}\}$
            \item $\N ^{\N }\sim [0,1]$
         \end{enumerate}
    \end{problem}

    \begin{definition}
        $\sqsupset X$ -- любое множество
        Отображение из $\N $ в $X$ называется последовательностью в $X$

        вместо $f(n), n\in \N \quad f:\N \to X$ используют $\{x_n\}_{n=1}^{\infty }$ или $\left( x_n \right) _{n=1}^{\infty }\qquad n\to x_n\in X$
    \end{definition}

    \section{Предел числовой последовательности}
    \begin{definition}
        Пусть $\{x_n\}_{n=1}^{\infty}$ последовательность вещественных чисел. $x_{+}\in \R$
        \[
            \lim_{n \to \infty} x_n := x_+ \iff  \forall \varepsilon>0\exists N: \forall n>N \quad \left| x_n-x_+ \right| <\varepsilon.
        \]
        В метрическом пространстве $(X,\rho)$ шар $B_R(a)$ называется также $R$-окрестностью точки $a$.
    \end{definition}

    \begin{definition}
        [Определение предела на языке окрестностей]
        $$\lim_{n \to \infty} x_n = x_* \iff \forall \text{ окрестности $U$ точки } x_*\quad \exists N\in \N : \forall n>N\quad x_{n} \in U$$
    \end{definition}
    \begin{example}
            $x_n = \frac{1}{n} \forall n\in \N $
            $x_* = 0$

            $\forall \varepsilon > 0 \exists N: \forall n>N\quad \left| \frac{1}{n} - 0 \right|  = \frac{1}{n}<\varepsilon\qquad n>\frac{1}{\varepsilon}\quad N:= \left\lfloor \frac{1}{\varepsilon} \right\rfloor +1$
    \end{example}
    \begin{note}
        Определение предела на ``языке окрестностей'' справедливо в случае последовательностей в метрическом пространстве.

        $x_n \to x_* \iff  \forall \varepsilon >0 \exists N: \forall n>N\quad \rho(x_n, x_*)<\varepsilon$.
    \end{note}
    \begin{statement}
        Пусть $X$ -- метрическое пространство и $c\in X$. 
        Если $\forall n\in \N , x_n = c$, то  $\lim_{n \to \infty } x_n = c$.
    \end{statement}
    \begin{proof}
        $x_* = c\quad \forall n\in \N \quad \rho(x_n, x_*) = 0<\varepsilon\  \forall \varepsilon > 0$

        $N=1$
    \end{proof}
    \begin{note}
        $\sqsupset \{x_n\}_{n=1}^{\infty }$ и $\{y_n\}_{n=1}^{\infty }$ -- последовательности в метрическом пространстве $X$ и $\exists m\in \N \quad x_n = y_n\ \forall n\geqslant m$. 
        Тогда $\lim_{n \to \infty} x_n$ и $\lim_{n \to \infty} y_n$ совпадают (если существует один, то существует другой и равны при существовании).
    \end{note}
    \begin{statement}
        [единственность предела]
        $\sqsupset (X, \rho), \{x_n\}_{n=1}^{\infty }\subseteq X, y, z\in X $

        Если $x_n\to y$ и $x_n\to z$, то $y=z$.
    \end{statement}
    \begin{proof}
        Если $y\neq z$, то $\rho(y,z) = \Delta>0\quad \varepsilon = \frac{\Delta}{2}$

        Т.к. $x_n \to y, x_{n} \to z$, то $\exists N_1, N_2:$

        $\forall n>N_1\quad \rho(x_n,y)<\varepsilon$

        $\forall n>N_2\quad \rho(x_n,z)<\varepsilon$

        $\forall n\geqslant \max\{N_1, N_2\}\quad 
        \begin{cases}
            \rho(x_{n} , y)<\varepsilon \\
            \rho(x_{n} , z)<\varepsilon\\
        \end{cases} \implies  \Delta = \rho(y,z) \leqslant  \rho(y, x_{n}) + \rho(x_{n} , z)<2\varepsilon = \Delta \implies \Delta<\Delta $ !!!
    \end{proof}
    \begin{example}
        $x_n = (-1)^{-1} \forall n\in \N  \not\exists \lim_{n \to \infty} (-1)^{n-1}$

        Если бы $\exists x_* = \lim_{n \to \infty} (-1)^{n-1}$, то для $\varepsilon = 1 \exists N$

        $n=2N\quad \left| (-1)^{n-1} - x_* \right| = |-1-x_*|<1 $

        $n=2N+1\quad \left| (-1)^{n-1} - x_* \right| = |1-x_*|<1 $

    $2 = |1 - (-1)|\leqslant |1-x_* + x_*-(-1)| \leqslant |1-x_*| + |x_* - (-1)| <2$
    \end{example}

    \begin{definition}
        \underline{Ограниченной} называется такая последовательность $\{x_n\}_{n=1}^{\infty}$, что ограничено множество её значений $\{x_n\}_{n\in \N}$. 
    \end{definition}

    \begin{definition}
        В метрическом пространстве \underline{сходящейся} последовательностью называется последовательность, у которой существует предел (в этом пространстве).
    \end{definition}

    \begin{theorem}
        Сходящаяся в метрическом пространстве последовательность ограничена.
    \end{theorem}
    \begin{proof}
        Пусть $\{x_n\}_{n=1}^{\infty }$ -- сходящаяся в метрическом пространстве $(X,\rho)$ последовательность, т.е. $\exists x^*\in X\forall \varepsilon>0 \exists N\in \N :\forall n\in N \quad\rho( x_n,x^*)<\varepsilon$

        $\sqsupset \varepsilon = 1, \sqsupset N = N(\varepsilon)$, т.е. $\forall n>N\quad \rho(x_n, x^*)<1$

        $R = max\{\rho(x_1, x^*), \rho(x_2, x^*), \ldots, \rho(x_N, x^*), 1\} \implies \forall n\in \N  x_{n} \in B_r[x^*] \implies \{x_n\}$ -- ограничена
    \end{proof}

    \begin{theorem}
        [предельный переход в неравенствах]

        $\sqsupset \{x_{n} \}, \{y_{n} \}$ -- вещественные последовательности. $x_n\to x_*, y_n\to y_*\quad x_*, y_*\in \R\quad \forall n\quad x_{n} \leqslant y_{n} \implies  x_*\leqslant y_*$

        Отметим, что из $x_{n} <y_{n} $ НЕ следует, что $x_*<y_*$.

        Пример: $x_{n} =0, y_{n}  = \frac{1}{n}, x_* = y_* =  0$, но при этом $x_n<y_n \forall n\in \N $
    \end{theorem}
    \begin{proof}
        от противного. $\sqsupset  x_*>y_*\quad \varepsilon = \frac{x_*-y_*}{2}$

        Т.к. $x_{n} \to x_*$, то $\exists N_1\left( =N(\varepsilon) \right): \forall n\in \N , n>N_1\quad \left| x_n-x_* \right| <\varepsilon $

        $\exists N_2\left( =N(\varepsilon) \right) \forall n>N_2\quad \left| y_n-y_* \right| <\varepsilon$

        Если $N = \max\{N_1, N_2\}$ и $n\in \N ~n>N \implies \begin{cases}
            \left| x_n-x_* \right| <\varepsilon\\
            \left| y_n-y_* \right| <\varepsilon\\
        \end{cases} \implies \begin{cases}
            x_n-x_* >-\varepsilon\\
            y_n-y_* >-\varepsilon\\
        \end{cases} \implies \begin{cases}
            x_n>x_*-\varepsilon = x_* - \frac{x_*-y_*}{2} = \frac{x_*+y_*}{2}\\
            y_n<y_*+\varepsilon = y_* - \frac{x_*-y_*}{2} = \frac{x_*+y_*}{2}\\
        \end{cases} \implies  y_n<x_n$ !!!
    \end{proof}

    Частные случаи(следствия):
    Пусть $\{x_n\}$ -- вещественная последовательность
    \begin{enumerate}
        \item $\sqsupset \forall n\in \N \quad x_n\leqslant b, b\in \R$ и $\exists \lim_{n \to \infty} x_n \implies \lim_{n \to \infty} x_n\leqslant b$
        \item $\ldots \geqslant a \ldots \implies  \lim_{n \to \infty} x_n\geqslant a$
        \item $\sqsupset n\in \N \quad x_{n} \in[a,b]$ и $\exists \lim_{n \to \infty} x_n \implies \lim_{n \to \infty} x_n\in [a,b]$
    \end{enumerate}

    \begin{theorem}
        [о зажатой последовательности, ``Принцип двух милиционеров'']
        Пусть $\{x_n\}_{n=1}^{\infty }, \{y_{n} \}_{n=1}^{\infty}, \{z_n\}_{n=1}^{\infty }$ -- вещественные последовательности, и $\forall n\in \N  x_{n} \leqslant y_{n} \leqslant z_n$.

        Если $\lim_{n \to \infty} x_n = \lim_{n \to \infty}  z_n = a $ (и пределы существуют), то $\exists \lim_{n \to \infty} y_n$ и $a = \lim_{n \to \infty} y_{n}$.
    \end{theorem}
    \begin{proof}
        $\sphericalangle\forall  \varepsilon>0$

        Т.к. $x_n \to a$, то $\exists N_1\in \N : n\in \N,\ n\geqslant N_1 \implies \left| x_n-a \right| <\varepsilon$.

        Т.к. $z_n \to  a$, то $\exists N_2\in \N :n\in \N,\ n>N_2, \quad\left| z_n-a \right| <\varepsilon$.

        $N = \max\{N_1, N_2\}$, тогда $n\in \N \quad n>N$.

        $\begin{cases}
            \left| x_n-a \right| <\varepsilon\\
            \left| z_n-a \right| <\varepsilon\\
        \end{cases}\quad \begin{cases}
            x_n> a-\varepsilon\\
            y_n<a+\varepsilon\\
        \end{cases}$

        $a-\varepsilon <x_n\leqslant y_{n} \leqslant z_n<a+\varepsilon \implies \left| y_n-a \right| <\varepsilon \implies y_n\to a$
    \end{proof}
    
    \begin{definition}
        $\sqsupset \{x_n\}_{n=1}^{\infty }$ -- числовая последовательность. 

        $\{x_n\}_{n=1}^{\infty }$ называется \underline{бесконечно малой}, $x_{n} \to 0, n\to +\infty $
    \end{definition}
    
    \begin{note}
        $\{x_n\}_{n=1}^{\infty }$ -- б.м. $\iff \{\left| x_{n}  \right| \}_{n=1}^{\infty }$ -- б.м.

        $\{x_n\}$ -- б.м. $\iff  \forall \varepsilon>0 \exists N = N(\varepsilon)\in \N: \forall n\in \N  n>N\quad \left| x_n \right| <\varepsilon \implies \left| x_n \right|  $ -- б.м. ($\left| \left| x_{n}  \right| -0 \right| <\varepsilon$).
    \end{note}

    \begin{definition}
        Число $N$ из определения предела последовательности $x_{n} $ называется \underline{$\varepsilon$ -допуском} этой последовательности, Д$\left( \varepsilon \right) $ -- набор всех $\varepsilon$ -допусков для данной последовательности
    \end{definition}

    \begin{example}
        Найти (какой-нибудь) $\varepsilon$-допуск для последовательности $\sqrt{\frac{n+1}{n}} = x_{n}  $ для $\varepsilon>0$
    \end{example}
    \begin{proof}
        Найти $N\in \N :\forall n\in \N ~n>N\quad \left| \sqrt{\frac{n+1}{n}} -1 \right| <\varepsilon$\\ 
        $\sqrt{\frac{n+1}{n}-1} <\varepsilon \iff  \frac{1}{\sqrt{n} }<\varepsilon\quad n>\frac{1}{\varepsilon^2}\quad N = \left\lfloor \frac{1}{\varepsilon^2} \right\rfloor +1$
    \end{proof}

    \begin{definition}
        $(X,K)\quad X$ -- множество, $K$ -- поле ($K = \R\lor K = \C$)

        ``+'' определено в $X$, $\cdot $ на элемент $K$

        $\forall x, y\in X\quad x+y\in X\qquad \forall \alpha \in K\quad \alpha\cdot k \in X$

        $(X,K)$ называется векторным (линейным) пространством, если
        \begin{enumerate}
            \item $\forall x, y\in X\quad x+y = y+x$
            \item $\forall x, y, z\in X\quad (x+y)+z = z + (y+z)$
            \item $\exists 0\in X\quad x+0=x$
            \item $\forall \alpha, \beta \in K, \forall x\in X\quad \alpha\cdot \left( \beta\cdot x \right)  = (\alpha\cdot \beta)\cdot x$
            \item $\forall \alpha, \beta\in K\quad \forall x\in X\quad (\alpha+\beta)\cdot x = \alpha\cdot x + \beta\cdot x$
            \item $\forall \alpha\in K\forall x, y\in X\quad \alpha\cdot (x+y) = \alpha\cdot x + \alpha\cdot y$
            \item $\forall x\in X\quad 1\cdot x = x$
        \end{enumerate}
    \end{definition}

    \begin{example}
        \begin{enumerate}
            \item $X=\R=K\quad X = \C = K\quad X=\C, K=\R$
            \item $X = \R^n, K+\R$ -- основной пример векторного пространства.
            \item $X = \{f:<a,b> \to \R\}, K = \R\qquad (\alpha\cdot f)(x) = \alpha\cdot f(x) \forall \alpha\in K, \forall f\in X\qquad (f_1+f_2)(x) = f_1(x) + f_2(x)$

                $\mathbb{0}(x):\equiv 0$
        \end{enumerate}
    \end{example}

    \begin{definition}
        Пусть $(X,K)$ -- векторное пространство.

        $p:X \to [0,+\infty )$ называется \underline{нормой} на $X$, если
        \begin{enumerate}
            \item $p(x) = 0 \iff  x = \mathbb{0}$ (невырожденность)
            \item $\forall \alpha\in K\forall x\in X\quad p(\alpha x) = \left| \alpha \right| p(x)$ (положительная однородность)
            \item $\forall x, y\in X\quad p(x+y)\leqslant p(x) + p(y)$ (неравенство треугольника)
        \end{enumerate}
        Функция $p:X\to [0,+\infty )$ и обладает свойствами $2, 3$ называется полунормой.
    \end{definition}

    Элементарные свойства полунормы:
    \begin{enumerate}
        \item $\forall  x, y\in X \forall \alpha, \beta\in K\quad p(\alpha x+\beta y) \leqslant |\alpha|p(x) + |\beta|p(y)$
        \item $\forall x\in X\quad p(-x) = p(x)$
        \item $p(x-y) \geqslant \left| p(x) - p(y) \right| $

            $p(x) = p(x-y+y) \leqslant p(x-y) + p(y)$

            $p(x) - p(y) \leqslant p(x-y)$
            $p(y) - p(x) \leqslant p(y-x) =p(x-y)$
    \end{enumerate}
    \begin{note}
        Норма порождает метрику. $(X, p)$, $X$ -- векторное пространство.

        $\rho(x, y) = p(x-y) \leqslant p(x-z) + p(z-y) \leqslant  \rho(x, z) + \rho(z, y)$
    \end{note}

    \begin{note}
        ``Обычное'' обозначение нормы $\|x\|$ вместо $p(x)$

        $\|x\|_2 = \sqrt{\sum_{k=1}^{n} x_k^2} $ -- евклидова норма, $x\in \R^n$

        $\|x\|_1 = \sum_{k=1}^{n} |x_k|$

        $\|x\|_{\infty } = \max\limits_{k=1:n} |x_k|$

        $\|x\|_p = \left( \sum_{k=1}^{n} |x_k|^p \right)^{\frac{1}{p}} , p\in [1, +\infty )$ -- норма (проверка позже).
    \end{note}
    \begin{example}
        $F(x)$ -- строго монотонно возрастает, $F:\R\to \R$

        $\rho_F(x,y) = \left| F(x) - F(y) \right| \forall x, y\in \R $

        $\left| F(x) - F(z) \right|  = \left| F(x) - F(y) + F(y) + F(z) \right| \leqslant \left| F(x)  - F(y) \right|  + |F(y) - F(z)| = \rho(x, y) + \rho(y, z)$

        ``$\|x\|$'' $ = \rho(x, )$ -- не обязательно положительно однородна, т.е не всякая метрика порождена нормой.
    \end{example}
    
    Забегая вперёд:  $C[a,b] = \{f \text{-- непрерывная на } [a,b], f:[a,b] \to  \R \}\quad \|f\| = \max\limits_{x\in [a,b]} \{|f(x)|\}$. 
    Упражнение: доказать, что это норма.

    \begin{definition}
        $\sqsupset (X,K)$ -- векторное пространство.

        $<x,y> :X\times Y \to  K, \quad <\cdot,\cdot>$ называется \underline{скалярным произведением} на $X$, если
        \begin{enumerate}
            \item $<x,x> \geqslant 0$ и $\forall x\in X\quad <x,x> = 0 \iff x=\mathbb{0} $
            \item $\forall x, y, z \in X \forall \alpha, \beta \in K\quad <\alpha x+\beta y,z> = \alpha <x,z> + \beta <y,z>$
            \item $\forall x, y\in X\quad <x,y> = \overline{<y,x>}$
        \end{enumerate}
    \end{definition}

    Элементарные следствия:
    \begin{enumerate}
        \item $<z,\alpha x + \beta y> = \overline{\alpha}<z,x> + \overline{\beta}<z,y>$

            $<z,\alpha x + \beta y> = \overline{<\alpha x + \beta y,z}  = \overline{\alpha<x,z> + \beta <y,z>} = \overline{\alpha}\overline{<x,z>} + \overline{\beta}\overline{<y,z>} = \overline{\alpha}<z,x> + \overline{\beta}<z,y>$
        \item
    \end{enumerate}

    \begin{statement}
        Если $(X,K)$ -- векторное пространство, $<\cdot , \cdot >$ -- скалярное произведение, то $\forall x, y\in X\quad \left| <x,y> \right| ^2\leqslant <x,x><y,y>$
    \end{statement}
    \begin{proof}
        \begin{enumerate}
            \item $y = \mathbb{0}, y = 0\cdot\mathbb{0}\quad <x,y> = <x,0\cdot \mathbb{0}> = 0$
            \item $y\neq \mathbb{0} \implies <y,y>\neq 0\quad z = \frac{<x,y>}{<y,y>} \in K$

            $0\leqslant <x-\alpha y, x - \alpha y> = <x,x> - \overline{\alpha} <x,y> - \alpha <x,y> + \alpha\cdot \overline{\alpha}<y,y>$

            $0\leqslant <x,x> - Re \frac{<x,y>}{<y,y>} \cdot  <x,y> + \frac{<x,y>^2}{<y,y>}$
            \item $2Re <x,y> \cdot  <x,y> - <x,y>^2 \leqslant <x,x><y,y>$ 
        \end{enumerate}
    \end{proof}

   \begin{statement}
        [неравенство Коши-Буняковского-Шварца]
        ~\\
        $x = (x_1, \ldots, x_{n} )\quad y = (y_1, \ldots, y_{n} )$

        $\left<x, y \right>_2 = \sum x_ky_k$

        $\left<x, x \right> = \|x\|_2^2$

        $\left<y, y \right> = \|y\|_2^2$

        $\left( \sum_{k=1}^{n} x_ky_k \right) ^2 \leqslant \left( \sum_{k=1}^{n} x_k^2 \right) \left( \sum_{k=1}^{n} y_k^2 \right) $

        $\sum_{k=1}^{n} |x_ky_k|\leqslant \sqrt{\sum_{k=1}^{n} x_k^2} \sqrt{\sum_{k=1}^{n} y_k^2} $
    \end{statement}

    \begin{theorem}
        [О связи пределов и арифметических действий в нормированных пространствах]
        Пусть $(X,K)$~--- нормированное векторное пространство (векторное пространство, снабжённое нормой).

        $\{x_n\}_{n=1}^{\infty }, \{y_n\}_{n=1}^{\infty }$ последовательности в $X$, $\{\alpha_n\}_{n=1}^{\infty }$ в $K$.

        $x_{n} \to x, y_{n} \to y, x, y\in X\quad \alpha_n\to \alpha\in K$
        Тогда 
        \begin{enumerate}
            \item $x_{n} \pm y_{n} \to  x\pm y$
            \item $\alpha_n\cdot x_n \to \alpha x$
            \item $\|x_n\| \to  \|x\|$
            \item Если $X = \R \lor X = \C, K=X, y\neq 0, \forall n y_{n} \neq 0$, то $\frac{x_{n} }{y_{n} } \to \frac{x}{y}$
        \end{enumerate}
    \end{theorem}
    \begin{proof}
        \begin{enumerate}
            \item $x_{n} + y_{n} \to  x+y? $

                $\forall \varepsilon>0\ \exists \varepsilon$-допуск для $x_{n} +y_{n} $

                $\sqsupset N_1\in $Д$(\frac{\varepsilon}{2}, \{x_{n} \})$

                $\sqsupset N_2\in$Д$\left( \frac{\varepsilon}{2}, \{y_{n} \} \right) $

                $N = \max\{N_1, N_2\}$, если $n\in \N , n>N$

                $\|(x_{n} +y_{n} ) - (x+y)\|\leqslant \|x_{n} -x\| + \|y_{n} -y\| <\frac{\varepsilon}{2} + \frac{\varepsilon}{2} = \varepsilon$, т.о. $N\in $Д$\left( \varepsilon, \{x_{n} + y_{n} \} \right) $

                Для разности аналогично.

                \begin{lemma}
                    Пусть $\{a_n\}_{n=1}^{\infty }, \{b_n\}_{n=1}^{\infty }    $ -- числовые последовательности, $\{a_n\}$ -- ограничена, $\{b_n\}$ -- б.м. $\implies  \{a_nb_n\}$ -- б.м.
                \end{lemma}
                \begin{proof}
                    [Доказательство леммы]

                    $\{a_n\}$ -- ограничено $\implies \exists R>0: \forall n\in \N \quad |a_n| \leqslant R$

                    $b_n$ -- б.м. $\implies$ для $\forall \varepsilon>0\ \exists n\in \text{Д}\left( \frac{\varepsilon}{2}, \{b_n\} \right) $, т.е. $|b_n|<\frac{\varepsilon}{R}\ \forall n\in \N  ~ n>N$. 
                    Тогда $|a_nb_n| = |a_n| |b_n| <R\cdot \frac{\varepsilon}{R} = \varepsilon \implies N\in \text{Д}\left( \varepsilon, \{a_nb_n\} \right) \implies a_nb_n \to 0, n\to  +\infty  $
                \end{proof}
            Доказательство теоремы:

        \item $\alpha_nx_n - \alpha x = \alpha_nx_n - \alpha x_n + \alpha x_n - \alpha x = (\alpha_n-\alpha)x_n + \alpha(x_n - x)$

            $\|(\alpha_n - \alpha)\cdot x_n\| = |\alpha_n - \alpha|\|x_n\|\qquad \|\alpha(x_n-x)\| = |\alpha|\|x_n-x\|$

            В каждой один из множителей ограничен, а другой бесконечно малый

            $\implies  \|\alpha_nx_n - \alpha x\|$ -- б.м. $\implies  \alpha_nx_n - \alpha x \to  0 \implies  \alpha_nx_n \to  \alpha x$
        \item $\left| \|x_{n} \| - \|x\| \right| \leqslant \|x_{n} -x\|$ -- б.м.

        \item $\frac{x_{n} }{y_{n} } - \frac{x}{y} = x_n\cdot \frac{1}{y_n} \to  x\cdot \frac{1}{y} \impliedby (2)$, если $\frac{1}{y_{n} } \to  \frac{1}{y} \iff  \frac{1}{y_{n} } - \frac{1}{y}$ -- б.м. $\frac{y-y_{n} }{yy_{n} } = (y-y_{n} )\frac{1}{y}\frac{1}{y_{n} }$

        $\varepsilon = \frac{1}{2}|y| >0\quad \sqsupset N \in \text{Д}\left( \varepsilon, \{y_{n} \} \right) \quad n>N \implies  |y_n-y| <\varepsilon$

        $m = \min\{|y_1|, |y_2|, \ldots, |y_n|, \varepsilon\}$ и $m>0$

        $\forall n\in N n\leqslant N \lor n>N\quad |y_n| \geqslant m \lor |y_n| \geqslant |y| - |y_n-y| = 2\varepsilon - \varepsilon = \varepsilon \geqslant m \implies  \left| \frac{1}{y_n} \right| \leqslant \frac{1}{m} \implies \{\frac{1}{y_{n} }\}$ -- ограничено.
        \end{enumerate}
    \end{proof}
    \begin{definition}
        $\sqsupset \{x_{n}\}_{n=1}^{\infty } $ -- вещественная последовательность

        $\lim_{n \to \infty} x_n = +\infty  \iff  \forall M\in \R \exists N\in \N : \forall n\in \N , n>N\quad x_n>M$

        $\lim_{n \to \infty} x_n = -\infty  \iff  \forall M\in \R \exists  N\in \N : \forall n\in \N , n>N\quad x_n<M$

        $\sqsupset \{x_n\}_{n=1}^{\infty } \subset \C$

        $\lim_{n \to \infty} x_n = \infty \iff \forall M\in \R\exists N\in \N : \forall n\in \N , n>N\quad |x_n| >M$
    \end{definition}    
    \begin{note}
        \begin{enumerate}
            \item $x_n \to \infty  \iff  |x_n| \to +\infty $
            \item $x_n\to  +\infty  \lor x_n\to -\infty  \implies  x_n \to  \infty $ (обратное неверно: $x_n = (-1)^n\cdot n$
        \end{enumerate}
    \end{note}

    \begin{definition}
        Последовательности $x_n:\quad x_n\to \infty $ называются \underline{бесконечно большими}.
    \end{definition}
    \begin{note}
        $\{x_n\}$ -- б.б. $\implies $ неограничена (обратное неверно: $x_{n}  = (1+(-1)^n)\cdot n$ -- неограничена и не б.б.).
    \end{note}
    \begin{lemma}
        [О связи бесконечно больших и бесконечно малых]

        Пусть $\{x_n\}_{n=1}^{\infty }$ -- числовая последовательность и $\forall n\in \N  \quad x_n\neq 0$. Тогда $x_{n} $ -- б.б. $\iff \frac{1}{x_{n} }$ -- б.м. 

        $x_{n} $ -- б.м. $\iff  \frac{1}{x_{n} }$ -- б.б.
    \end{lemma}
    \begin{proof}
        $x_{n} $ -- б.б. $\iff  \forall M>0 \exists  N\in \N : \forall n\in \N  n>N\quad |x_n| >M \iff \frac{1}{x_{n} }<\frac{1}{M}\quad M = \frac{1}{\varepsilon}$

        $\iff \forall \varepsilon >0 \exists \ldots\ldots \left| \frac{1}{x_{n} } \right| <\varepsilon$
    \end{proof}

    $\{x_k\}_{k=1}^{\infty }, n\in \N \quad \{x_k\}_{n=k}^{\infty }$ -- хвост последовательности $x_k$

    Если $\{x_k\}_{k=1}^{\infty }$, то последняя лемма применима к некоторому хвосту этой последовательности. 
    \begin{note}
        $\overset{\land}{\C} = \C\cup \{\infty \}$
    \end{note}
    \begin{theorem}
        [Арифметические действия над бесконечно большими]

        $\sqsupset \{x_{n} \}_{n=1}^{\infty }, \{y_{n} \}_{n=1}^{\infty }$

    \begin{note}
        $x_{n} \to \pm \infty $ имеет смысл тогда и только тогда, когда $x_{n} \in \R$.
    \end{note}

        \begin{itemize}
            \item [I)] Если $\{x_{n}\} \to +\infty , \{y_{n}\} $ ограничено снизу $\implies  x_{n} +y_{n} \to  +\infty $ 
            \item [II)] Если $\{x_{n} \} \to -\infty , \{y_{n} \}$ ограничена сверху $\implies  x_{n} +y_{n} \to -\infty $
            \item [III)] Если $\{x_{n} \} \to \infty$, $y_n\to a\in \C$ ограничено снизу $\implies x_{n} +y_{n} \to \infty $
            \item [IV)] Если $\{x_{n} \} \to  +\infty(-\infty ) \quad \exists \delta >0: y_{n} >\delta \forall n\in \N  \implies x_{n} \cdot y_{n} \to +\infty (-\infty )$
            \item [V)] Если $\{x_{n} \} \to +\infty (-\infty )\quad \exists \delta >0:\forall n\in \N \quad y_{n} <-\delta \implies x_{n} \cdot y_{n}  \to -\infty (+\infty )$
            \item [VI)] Если $\{x_{n} \} \to  \infty \quad \exists \delta >0: \forall n\in \N  \quad |y_{n}|>\delta \implies x_{n} *y_{n}  \to \infty $
            \item [VII)] Если $\{x_{n} \} \to a\in \C, \quad y_{n} \to \infty \& \forall n\in \N \quad y_{n} \neq 0 \implies \frac{x_{n} }{y_{n} }  \to  0$
            \item [VIII)] Если $x_{n} \to a\in \overset{\land}{\C}\setminus \{0\}, \quad y_{n} \to  0 \implies  \frac{x_{n} }{y_{n} }\to \infty $
            \item [IX)] Если $\{x_{n} \} \to  \infty , \quad \forall n\in \N \quad y_{n} \neq 0\quad y_{n} \to a\in C \implies \frac{x_{n} }{y_{n} } \to \infty $
        \end{itemize}
    \end{theorem}
    \begin{proof}
        \begin{itemize}
            \item [(III)] $z_n\to \infty  \iff  \forall M>0 \exists N\in \N : \forall n>N, n\in \N \quad |z_n|>M$

            $\sphericalangle \forall M>0$ По условию $y_{n} \to a\in \C \implies  \exists C: \forall n\in \N \quad |y_{n} |\leqslant C$

            Т.к. $\{x_{n} \} \to \infty,\ \exists \p N\in \N :\quad x_{n} > M+C \forall n>\p N, n\in \N$.

            $a_n = |x_{n} +y_{n} | \geqslant  |x_{n} | - |y_{n} | >M+C-C = M$

            $N = \p N$
            \item [(V)] $\forall M>0 \exists N\in \R: \forall n\in \N , n>N\quad x_{n} y_{n} <-M$

            Т.к. $x_{n} \to +\infty $, то для  $\frac{M}{\delta}\ \exists \p N\in \N : \forall n>\p N\quad n\in \N \quad x_{n} >\frac{M}{\delta} \implies  x_{n}y_{n} <\frac{M}{\delta}\cdot (-\delta) = -M, \quad N = \p N$ 
            \item[(IX)] $y_{n} \neq 0 \forall n\in \N $
                \begin{enumerate}
                    \item $a\neq 0 \implies \frac{1}{y_{n} }\to \frac{1}{a}, \exists \delta >0: \left| \frac{1}{y_{n} } \right| >\delta$ (см теорему об арифметических действиях над сх. последовательностями) $\implies \frac{x_{n} }{y_{n} } \to \infty \infty $ (по пункту VI)
                \end{enumerate}
        \end{itemize}
    \end{proof}

    $\overset{\land}{\R} = \R\cup \{\infty \}$

    Окрестность точки $\infty $ в $\overset{\land}{\R}$ -- множество вида $\{\infty \}\cup \{x\in \R:|x|>R\}$.

    \begin{note}
        $x_{n} \to a\quad a\in \overset{\land}{\C}, a\in \ov{\R}, a\in \overset{\land}{\R} \iff  \forall $ окрестности $U_a$ в пространстве $X$ $\exists $ окрестность $V_{+\infty }: \forall n\in \N \cap V_{+\infty} \implies  x_{n} \in U_a$ (ещё одна формулировка на языке окрестностей).
    \end{note}

    \begin{note}
    [замечание к теореме]
        $x_{n} \to \pm \infty (\infty ), y_{n} \to \pm \infty (\infty )\quad -\frac{\infty }{\infty }, \frac{0}{0}, \infty \cdot 0, +\infty +(-\infty ), \infty +(\infty )$ -- неопределённость, т.е. нет универсального утверждения про предел.
    \end{note}
    \begin{example}
        $\frac{\infty }{\infty }$

        \begin{enumerate}
            \item $x_{n} =n=y_{n} \to \infty  \quad \frac{x_{n} }{y_{n} } \to 1$
            \item $x_{n} =n^2, y_{n} =n \to \infty\quad \frac{x_{n} }{y_{n} } \to \infty  $
            \item $x_{n} =n, y_{n} =n^2 \to \infty \quad \frac{x_{n} }{y_{n} } \to  0$
            \item $x_{2n}=2n, x_{2n+1} = n^2\quad y_{2n} = x_{2n+1}, y_{2n+1} = x_{2n}$

            $\not \exists \lim_{n \to \infty} \frac{x_{n} }{y_{n} } $ -- упражнение.
        \end{enumerate}
    \end{example}

    \begin{proof}
        [Продолжение доказательства неравенства Коши-Буняковского-Шварца]

        $\left| \left< x, y \right> \right| ^2 \leqslant \left<x, x \right> \cdot  \left< y, y \right> \forall $ скалярного произведение

        \begin{example}
        [Примеры скалярных произведений]

            $C\left( [a,b] \right)  = \{f: f\text{ непрерывна на }[a,b]\}\qquad f:[a,b] \to  \R$

            $<f,g> = \int \limits_a^b f(x)g(x)dx$

            $\ell ^2 = \{(x_1, x_2, \ldots):x_k\in \C\quad \sqrt{ \sum_{k=1}^{\infty } \left| x_k \right|^2} <+\infty  \}\qquad \left< x,y \right> = \sum_{k=1}^{\infty } x_ky_k$
        \end{example}

        $y=0 \implies \left< x, y \right> = \left< x,\mathbb{0} \right> = \left< x, 0\cdot \mathbb{0} \right> = o\cdot \left< x, \mathbb{0} \right> = 0\qquad \left< y, y \right> = 0\implies $ КБШ верно

        $y\neq 0 \implies  \left< y, y \right> >0\quad \lambda = \frac{\left< x, y \right>}{\left< y, y \right>}$

        $0\leqslant \left< x-\lambda y, x-\lambda y \right> = \left< x, x \right> - \lambda \left< y, x \right> - \left< x, \lambda y \right> + \left< \lambda x, \lambda y \right> = \left< x, x \right> - \lambda \left< y, x \right> - \ov{\lambda} \left< x, y \right> + \lambda \cdot  \ov{\lambda} \left< y, y \right> = \left< x, x \right> - \frac{\left| \left< x, y \right> \right| ^2}{\left< y, y \right>} - \frac{\left| \left< x, y \right> \right| ^2}{\left< y, y \right>} + \frac{\left| \left< x, y \right> \right| ^2}{\left< y, y \right>} \implies  \left< x, x \right>\cdot \left< y, y \right> - \left| \left< x, y \right> \right| ^2 \geqslant 0$ 

        Равенство в КБШ $\iff  x-\lambda y = 0\quad x = \lambda y\quad (x$ коллинеарен $y)$  

        $\left| \left< x, y \right> \right| ^2 \leqslant \left< x, x \right> \cdot  \left< y, y \right>$

        $x = \left( x_1, x_2, \ldots, x_{n} ), y = \left( y_1, y_2, \ldots y_{n}  \right)  \in \R^n \right) $

        $\left| \sum_{k=1}^{n} x_{k} y_{k}\right| \leqslant \sqrt{\sum_{k=1}^{n} x_k^2} \cdot \sqrt{\sum_{k=1}^{n} y_k^2}  $

        $x\in \C^n, y\in \C^n\quad \left( |x_1|, \ldots, |x_{n} |\right), \left( |y_1|, \ldots, |y_{n} | \right)   $

        $\sum_{k=1}^{n} |x_ky_k| \leqslant \sqrt{\sum_{k=1}^{n} |x_k|^2} \sqrt{\sum_{k=1}^{n} |y_k|^2}  $
    \end{proof}

    \begin{statement}
        Пусть $(X, \mathcal{K})$~--- векторное пространство над полем $\mathcal{K}( = \R, \C)$, в котором определено скалярное произведение $ \left< x, y \right>$. 
        Тогда $p(x) = \sqrt{\left< x, x \right>} $ есть норма на $X$.
    \end{statement}
    \begin{proof}
        $p(\lambda x) = |\lambda|p(x)\quad \lambda \in \mathcal{K}$

        $\sqrt{\left< \lambda x, \lambda x \right>} = \sqrt{\lambda \cdot  \ov{\lambda} \left< x, x \right>}  = |\lambda| \cdot \sqrt{\left< x, x \right>} $

        $p(x+y)\leqslant p(x) + p(y)$

        $p^2(x+y) \overset ? {\leqslant} (p(x) + p(y))^2 = p^2(x) + p^2(y) + 2p(x)p(y) = \left< x, x \right> + \left< y, y \right> + 2\sqrt{\left< x, x \right> \left< y, y \right>}$

        $p^2(x+y) = \left< x+y, x+y \right> = \left< x, x \right> + \left< y, y \right> + \left< x, y  \right> + \left< y, x \right>$

        $2 Re\left<x, y \right> \leqslant 2 \left| \left< x,y \right> \right| \leqslant 2\sqrt{\left< x, x \right>} \sqrt{\left< y, y \right>} $
    \end{proof}

    Неравенство КБШ через норму: $\left| \left< x, y \right> \right| \leqslant \|x\| \cdot \|y\|\quad \|x\| = P(x)$ для любой нормы порождённой скалярным произведением.

    \begin{statement}
        Пусть $\left( X, \mathcal{K} \right)$~--- векторное пространство со скалярным произведениемм $\left<\cdot , \cdot  \right>$. 
        Если  $\{x_{n} \}, \{y_{n} \} \subset X\quad x, y\in X\quad x_{n} \to x, y_{n} \to y\quad n \to  +\infty$, то  $\left<x_{n} , y_{n}  \right> \to \left< x, y \right>, n\to  +\infty$.
    \end{statement}
    \begin{proof}
        $\left< x_{n} , y_{n}  \right> - \left< x, y \right> \to 0$

        $\left( \left<x_{n} , y_{n}  \right> - \left<x_{n} , y \right> \right)  + \left( \left< x_{n} , y \right> - \left< x, y \right> \right)  = \left< x_{n} , y_{n} -y \right> + \left< x_{n} -x, y_{n}  \right> \to 0$

        Потому что $\left| \left< x_{n} , y_{n} -y \right> \right| \leqslant \|x_{n} \|\cdot \|y_{n} -y\| =  \text{ огр. } \cdot  (\to 0) \to 0$. И аналогично со вторым.
    \end{proof}

    \section{Топологические свойства множеств в метрических пространствах}

    \begin{definition}
        Пусть $\left( X, \rho \right)$~--- метрическое пространство. $E\subseteq X, a\in E$

        $a$ называется внутренней для  $E\quad (a\in \Int E)$ если $\exists R>0: B_E(a) \subset E$. 
    \end{definition}

    \begin{definition}
        Множество в метрическом пространстве называется \underline{открытым}, если $E = \Int E$.

        Для ``Остальных'' множеств верно $\Int E \subset E$.
    \end{definition}
    \begin{example}
        \begin{enumerate}
            \item $E = X,\ X = \Int E,\quad X$ - открыто
            \item $\O $ -- открыто
            \item $(0,1)$ -- открытое множество $ R = \min \{1-a, a\}$
    \end{enumerate}
    \end{example}

    \begin{statement}
        В любом метрическом пространстве открытый шар является открытым множеством.
    \end{statement}
    \begin{proof}
        $\sphericalangle$ открытый шар $B_R(a)$ в метрическом пространстве  $(X, \rho)$.

        $\sphericalangle \forall b\in B_R(a) \implies  \rho(b,a)<r<R$

        $\delta = R-r>0\quad \sphericalangle B = B_{\delta}(b)$

        $\sqsupset c\in B \implies  \rho(c,a) \leqslant \rho(c,b) + \rho(b,a) < \delta + r = R-r +r = R$

        Т.о. $b\in Int(B_R(a)) \implies  B_R(a)$ -- открытое.
    \end{proof}

    \begin{note}
        Свойство внутренней точки (и внутренности) зависит от объемлющего пространства. 

        $E\subseteq X,\quad E\subseteq Y,\ Int_X E \neq \Int_Y E$ (может оказаться неравным).
    \end{note}
    \begin{example}
        $E = [0,1],\ X=\R,\ Y = E\quad \Int_X [0,1] = (0,1), \Int_E E = [0,1]$
    \end{example}

    \begin{theorem}
        [свойства открытых множеств в метрических пространствах]

        Пусть $(X, \rho)$~--- произвольное метрическое пространство. 
        Тогда 
        \begin{enumerate}
            \item [(I)]$\O , X$ -- открыты
            \item [(II)] $\forall  \{O_i\}_{i\in I}$ -- открытых $ \implies  \bigcup\limits_{i \in  I} O_i$
            \item [(III)] $\forall n\in \N \forall \{O_1, \ldots O_n\}$ открытых $\implies  O_1\cap \ldots \cap O_N$ открыто

            \begin{note}
                $O_n = \left( -\frac{1}{n}, \frac{1}{n} \right) \quad \bigcap\limits_{n\in \N } O_n = \{0\}$ -- не открыто
            \end{note}
        \end{enumerate}
    \end{theorem}
    \begin{proof}
        \begin{itemize}
            \item [(II)] $\sqsupset x\in \bigcap\limits_{i \in  I} O_i \implies  \exists i\in I: x\in O_i\qquad O_i$ -- открыто $\implies x\in Int O_i \implies  \exists \delta >0:\quad B_{\delta}(x) \subseteq O_i \implies  B_{\delta}(x) \subseteq \bigcup\limits_{i \in  I}  O_i \implies  x\in Int\left( \bigcup\limits_{i \in  I} O_i \right)  \implies  \cup O_i$ -- открыто.
            \item [(III)] $\sqsupset x\in \bigcap\limits_{i \in  I} O_i \implies \forall i = 1:n\quad x\in O_i \implies  \forall i = 1:N\quad \exists \delta_i>0: B_{\delta i}(x) \subseteq O_i$

            $\delta = \min \{\delta_1, \ldots, \delta_n\} >0 \implies  B_{\delta}(x) \subseteq O_i \implies  B_{\delta}(x) = \bigcap\limits_{i=1} ^N O_i \implies  x\in Int \bigcap\limits_{i =1}^N \implies \cap O_i $ -- открыто.
        \end{itemize}
    \end{proof}

    \begin{definition}
        ~\\
        $X$ -- множество,  $\Omega \subseteq 2^X$. 
        $\Omega$ называется \underline{топологической структурой} (или \underline{топологией}) на $X$, если
            \begin{enumerate}
                \item $\O , X\in \Omega$
                \item $\forall \{O_i\}_{i\in I} \subseteq \Omega \implies  \bigcup\limits_{i \in  I} O_i \in \Omega$
                \item $\forall N\in \N \forall O_1, \ldots, O_N \in \Omega \implies  O_1\cap \ldots\cap O_N \in \Omega$
            \end{enumerate} 

    При этом $(X, \Omega)$ называют \underline{топологическим пространством}. Элементы  $\Omega$ называют открытыми множествами в  $X$ (в  $(X, \Omega)$).

    Т.о. любое метрическое пространство является топологическим.
    \end{definition}

    \begin{example}
       \begin{enumerate}
            \item $X$~--- любое множество $\quad \Omega = \{\O , X\}$~--- антидискретная (не метризуемо, если состоит больше, чем из одной точки).
            \item $X$~--- любое множество, $\quad \Omega = 2^X$~--- дискретная топология.
            \item $X = \R, \Omega = \{\R\setminus A: A \text{ -- конечное множество}\}$ или $\Omega_2 = \{\R\setminus A: A\text{ -- не более, чем счётное множество}\}$ или $\Omega_2 = \{(a, +\infty ):a\in \R\}$.
        \end{enumerate} 
    \end{example}

    \begin{definition}
        Пусть $(X, \Omega)$ -- топологическое пространство, $E\subseteq X, a\in X$.

        $a$ называется \underline{предельной}  для $R$ (или \underline{точкой сгущения}, $a\in \p E$) $\iff \forall $ открытого $O:\quad a\in O\quad O\setminus \{a\}\cap E\neq \O$.

        ``Открытая окретсность'' точки $a$ в топологическом пространстве $X$ -- это любое открытое, содержащее точку $a$.

        $\overset{\cdot } U(a) = U\setminus \{a\}$~--- проколотая окрестность точки $ a$.
    \end{definition}

    \begin{statement}
        Пусть $E\subseteq X, (X, \rho)$~--- метрическое пространство.

        $a\in \p E \iff  \exists \{x_{n} \} \subseteq E\setminus \{a\}: x_{n} \to a, n\to +\infty $
    \end{statement}
    \begin{proof}
        \begin{itemize}
            \item [$\impliedby$:] из определения.
            \item [$\implies $:] $\sphericalangle B_{\frac{1}{n}}(a)\setminus \{a\} \cap E\neq \O  \implies  \exists x_n\in (B_{\frac{1}{n}}\setminus \{a\})\cap E\qquad \rho(x_{n} , a) <\frac{1}{n} \to 0, n\to +\infty  \implies  x_{n} \to a$
        \end{itemize}
    \end{proof}

    \begin{definition}
        Множество $E$  в топологическом пространстве  $(X, \Omega)$ называется замкнутым, если $\p E \subseteq E$.
    \end{definition}
    \begin{example}
        \begin{enumerate}
            \item $E = (0,1]\quad \p E = [0,1]$
            \item  $E = \{\frac{1}{k}\}_{k\in \N }, \p E = \{0\}$ 
            \item $E = \R, \p E = \R$
        \end{enumerate}
    \end{example}
    \begin{theorem}
        В любом топологическом пространстве множество открыто тогда и только тогда, когда его дополнение замкнуто.
    \end{theorem}
    \begin{proof}
        Пусть $(X, \Omega)$ -- топологическое пространство и $O\subseteq X$.
    
        $O$ -- открыто. 
        $F = X\setminus O\quad F$ -- замкнуто?.
    
        $\sphericalangle a\in \p F, a\in F?$.
    
        Пусть нет, тогда $a\in O \implies (O\setminus \{a\})\cap F = \O $ !!! $\implies  a\in F \implies F$ -- замкнуто. 
    
        Пусть $F$~--- замкнуто $\implies ?\quad O = X\setminus F$~--- открыто. 
        Если $O = \O$,то $O$~--- открыто.
    
        Если $O\neq \O , \sphericalangle a\in O \implies a \not\in F \implies a\not\in \p F \implies \exists $ окрестность $U$ точки $a:\quad \left( U\setminus \{a\} \right) \cap F = \O \implies  U\cap F = \O $ (т.к. $a\not\in F$) $\implies  U\subset O = X\setminus F \implies  X\in U\subseteq O \implies x\in Int O \implies (a-\forall )O $~--- открытое.
    \end{proof}
    
    \begin{corollary}
        \begin{enumerate}
            \item Пусть $(X, \Omega)$ -- топологическое пространство. 
            Тогда
                \begin{enumerate}
                    \item [I)] $\O , X$ -- замкнутые
                    \item [II)] $\forall $ семейства $\{F_i\}_{i\in I}$ замкнутых  $\implies  \bigcap\limits_{i \in  I} f_i$ -- замкнуто
                    \item [III)] $\forall N\in \N , \forall \{F_1, \ldots, F_N\}$ замкнутых $F_1\cup \ldots\cup  F_N$ -- замкнуто.
                \end{enumerate}
        \end{enumerate}
    \end{corollary}
    
    \begin{definition}
        $ \sqsupset (X, \Omega)$ -- топологическое пространство. 
        $E\subseteq X, a\in E$.
    
        $a$ называется \underline{изолированной точкой} $E$, если  $\exists $ окрестность $U$ точки  $a: \quad U\cap E = \{a\}$.
    \end{definition}
    
    \begin{definition}
        Пусть $(X, \rho)$~--- метрическое пространство, $E\subseteq X, a\in X$.
    
        $a$ называется \underline{точкой прикосновения}  $E \left( a\in Cl ~E \right) $, если $\forall $ окрестности $U$ точки $a$  $U\cap E \neq \O $.
    
        $\Cl E$ -- замыкание множества $E\qquad E\subseteq \Cl E$
    \end{definition}
    
    \begin{theorem}
        [О замыкании]
    
        Пусть $E \subseteq X, (x, \rho)$ -- топологическое пространство. 
        Тогда
        \begin{enumerate}
            \item [(I)] $\Cl E = E\cup \p E = \p E \cup (E\setminus \p E)$ -- последнее является множеством изолированных точек. 
            \item [(II)]  $\Cl E = \cap \{F: F \text{ -- замкнуто }, E\subseteq F\} $
            \item [(III)] $\Cl E$ -- минимальное (по включению) замкнутое, содержащее  $E$
    
            (Если $F$~--- замкнуто, и $E\subseteq F$, то $\Cl E\subseteq F$) 
            \item [(IV)] Если $X$~--- метрическое пространство, то  $x\in Cl E \iff \exists \{x_{n} \}_{n=1}^{\infty }: x_{n} \to x$.
        \end{enumerate}
    \end{theorem}
    \begin{proof}
        \begin{enumerate}
            \item [(I)] $x\in \Cl E \iff 
            \begin{cases}
                x\in E\\
                x\not\in E \quad \forall \text{ окрестности } U \text{ точки } x\quad \overline{\cdot}U\cap E \neq \O\\ 
            \end{cases} \iff  
            \begin{cases}
                x\in E\\
                x\in \p E\\
            \end{cases} \iff  x\in E\cup \p E$
        \item [(II)] $c\in \Cl E\quad \sphericalangle F$ -- замкнутое$: E\subseteq F \implies  x\in F \implies x\in$ Пр.ч (правой части)
    
            $\sqsupset x\in$ Пр.ч. $=F$, если  $x\not\in \Cl E \implies  \exists $ окрестность $U$ точки $x:\quad U\cap E = \O $ 
    
             $\sphericalangle F_1 = F\setminus U$ -- замкнуто, $ F_1\supseteq$, т.к. $ E\subseteq F\quad E\cap U\neq \O , \quad F = F_1 \cap \ldots\quad x\in F = F_1, x\not\in F_1$!!!
         \item [III] $F$ -- правая часть равенства в II.  
         $F$ -- замкнуто как пересечение и для любого замкнутого $F_1: E\subseteq F_1 \implies F\subseteq F_1$ (по определению пересечения)
         \item [IV] $?x\in \Cl E \iff  
         \begin{cases}
                 x \text{ изолированная для } E &x_{n} \equiv x\\
                 x \text{ предельная для } E & \text{ тога по характеристике предельной точки}
         \end{cases}$
        \end{enumerate}
    \end{proof}
    
    \begin{problem}
        \begin{enumerate}
            \item $\Int E\subseteq E\qquad \Int(\Int(E)) = \Int E$
            \item $(\Int E)^c = \Cl(E^c)$
            \item  $\Cl(\Cl(E)) = \Cl(E)$
            \item  $\Int(A\cap B) = \Int(A)\cap \Int(B)$
            
                $\Cl(A\cap B) = \Cl(A)\cup \Cl(B)$ 
                
                $\Int(A\cup B)\neq \Int(A) \cup \Int(B)$ 
    
                $\Cl(A\cap B) = \ldots$
        \end{enumerate}
    \end{problem}

    
    \begin{note}
        $\Int O$~--- наибольшее открытое множество, содержащееся в $O$.
    \end{note}
    
    \begin{definition}
        Пусть $(X, \Omega)$ -- топологическое пространство, $E\subseteq X\quad a\in X$.
    
        $a$ называется \underline{граничной точкой} для  $E$, если  $\forall $ окрестности $U$ точки  $a\qquad U\cap E\neq \O $ И $U\cap (E^c)\neq \O $.
    
        $\Fr E = \{x: x\text{ -- граничная для } E\} = \partial E$
    \end{definition}
    
    \begin{example}
        \begin{enumerate}
            \item $X = \R, E = [0,1]\quad \Int E = (0,1). \Cl E = [0,1]\quad \Fr R = \{0, 1\}$
            \item $X = \R^2\ E = \{(x,0): x\in [0,1]\}\quad \Int E = \O , \Cl E = E, \Fr E = R$.
            \item $ X = [0,1], E = [0,1]\quad \Int E = E, \Cl E = E, \Fr E = \O $
        \end{enumerate}
    \end{example}
\end{document}
=======
\end{definition}

\begin{note}
    Если $(X, \rho)$ -- метрическое пространство.

    $\Omega = \left\{ \exists \text{ семейство открытых шаров } \{B_I\}_{i\in I}: O = \bigcup\limits_{i \in  I}  B_i \right\} $

    ($\forall O = \bigcup\limits_{x\in O} B_{r(x)}(x)$)
\end{note}

\begin{statement}
    $\sqsupset (X, \rho)$ -- м.п., $Y\subseteq X$ Тогда $\Omega_Y$ совпадает с топологией, порождённой индуцированной метрикой.
\end{statement}
\begin{proof}
    $\Omega_{\rho,Y} = \{O: \exists  \text{ открытые шары в } Y \{B_i\}_{i\in I}: O = \bigcup\limits_{i \in  I} B_i = \bigcup ( \ov{B_i} \cap Y) \}$

    $\ov{B_i}$ -- шар с тем же центром и радиусом, но в $X$
\end{proof}

\begin{note}
    $Y\subseteq X\quad O $ открыто в $Y \iff \exists \ov{O}$ открыто в $X:\quad O = \ov O \cap Y$

    $\sqsupset F\subseteq Y\quad F$ -- замкнуто (в $Y$)  $\iff  \exists $ замкнутое $\ov F $ в  $X\quad F = \ov F\cap Y$
\end{note}
\begin{proof}
    $F$ -- замкнуто в  $Y \iff  Y\setminus F\in \Omega_Y \iff  Y\setminus F = \ov O \cap Y$,(где $\ov O \in \Omega \iff  X\setminus \ov O$ -- замкнуто)

    $F = Y\setminus \left( Y\setminus F \right)  = Y\setminus (\ov O\cap Y) = Y\setminus \ov O = Y\cap (X\setminus \ov O)$
\end{proof}

\begin{example}
    Примеры:
    \begin{enumerate}
        \item $X = \R^2\quad Y = B_1(0)\cup \left\{ (2,0) \right\} \cup \left\{ (x,y): xy=1, x>0, y>0 \right\} $ 

            $B_1[0] = B_1(0)$

            $B_{\frac{1}{2}}[2] = \left\{ (2,0) \right\} \cup \{(x,\frac{1}{x}): (x-2)^2+\frac{1}{x^2}\leqslant \frac{1}{4}\}$

        \item $Y$ -- график  $y(x) = \sin \frac{1}{x}, x\neq 0$

            $\Cl E = Y\cup \left\{ (0,y):y\in [-1,1] \right\} $
    \end{enumerate}
\end{example}

\begin{note}
    Если $X$ -- метрическое пространство  $E\subseteq X$

    $a\in \p E \iff \forall $ открытой окрестности $U$ точки  $a\quad \overset{\cdot}U \cap E$ бесконечно
\end{note}

\section{Компактность, сходимость в себе, полнота пространств}

\begin{definition}
    $\sqsupset (X, \Omega)$ -- т.пр. $\sqsupset E\subseteq X, \left\{ A_i \right\} _{i\in I}\subseteq X$

    Если $E \subseteq \bigcup\limits_{i \in  I} A_i$, то говорят, что $\left\{ A_i \right\} _{i\in I}$ образует покрытие множества $E$ 
\end{definition}

\begin{definition}
    $\sqsupset (X,\Omega)$ -- т.п. $K\subseteq X$

    $K$ -- называется \underline{компактным} (в  $X$), если  $\forall $ покрытия $\left\{ O_i \right\} _{i\in I}$ множества $K$ открытыми можно извлечь конечное подпокрытие:  $\exists N: \exists i_1, i_2, \ldots i_n: K\subseteq O_{i_1}\cup O_{i_2} \cup  \ldots \cup  O_{i_n}$
\end{definition}

Элементарные свойства компактных множеств:
\begin{statement}\label{st1}
    Если $K$ -- компактно (в  $(X, \Omega)$) и  $F\subseteq K$ и $F$ -- замкнуто, то  $F$ -- компактно (в  $(X, \Omega)$)
\end{statement}
\begin{proof}
    $\sphericalangle $ открытое покрытие  $\left\{ O_i \right\} _{i\in I}$ множества $F$

    $\sphericalangle O = X\setminus F$ -- открытое $ \implies  \left\{ O_i \right\} _{i\in I} \cup \{O\}$ -- покрытие $ K$

    $K$ -- компакт  $\implies \exists i_1, \ldots, i_n:\quad K\subseteq O_{i_1} \cup \ldots\cup O_{i_n} \cup O$

    $F\subseteq O_{i_1} \cup  \ldots\cup O_{i_N} \implies $ (покрытие -- $\forall $) $F$ -- компакт
\end{proof}
\begin{statement}
    $\sqsupset (X, \Omega)$ -- топ. пр-во $\quad K\subseteq X$. Тогда следующие утверждения равносильны
    \begin{align}
        K \text{ комп. в } X\\
        $K$  \text{комп. в себе} \left( \text{комп в } (K, \Omega_K) \right) 
    .\end{align}
\end{statement}
\begin{proof}
     \begin{itemize}
         \item []
         \item [$1.1 \to 1.2$] $\sqsupset \left\{ G_i \right\} _{i\in I}$ -- открытое покрытие $K$  в  $K \implies  \forall i\in I \exists $ открытое в $X\quad O_i: G_i = O_i \cap K \implies  K \subseteq \bigcup\limits_{i \in  I} O_i \implies \exists i_1, \ldots, i_N: K\subseteq \bigcup\limits_{j=1}^n O_{i_j} \implies K\subseteq \left( \bigcup\limits_{j=1}^N O_{i_j}  \right)\cap K = \bigcup\limits_{j=1}^N\left( O_{i_j}\cap K \right) = \bigcup\limits_{j=1}^N G_{i_j}    $, т.о. $\left\{ G_{i_j} \right\} $ -- конечное подпокрытие $K$ в  $ K$
         \item [$1.2\to 1.1$] $\sphericalangle \forall $ открытое покрытие $\left\{ O_i \right\} _{i\in I} K$ в $X. \sqsupset G_i = O_i\cap K$ -- открытые в $K$

             $K\subseteq \bigcup\limits_{i \in  I} O_i \implies K\subseteq \left( \bigcup\limits_{i \in  I} O_i \right) \cap K = \bigcup\limits_{i \in  I} G_i$

             Т.к. $K$ компактно в  $K$, то  $\exists  i_1, \ldots, i_N:\quad K\subseteq G_{i_1}\cup \ldots \cup g_{i_N} \implies K\subseteq O_{i_1}\cup \ldots \cup O_{i_N}$ $\left( G_{i_1}\subseteq O_{i_1} \ldots \right) $. Т.о. $\{O_{i_1}, \ldots, O_{i_N}\}$ -- конечное покрытие
    \end{itemize}
\end{proof}
\begin{corollary}
    $K\subseteq Y\subseteq X\quad (X, \Omega)$ -- т.п. Тогда $K$ компактно в  $Y \iff  K$ компактно в $X$
\end{corollary}
\begin{note}
    $(0,1) = \bigcup\limits_{n=2}^{\infty }\left( \frac{1}{n}, 1 \right)  $ -- не извлекается конечное подпокрытие
\end{note}
\begin{statement} \label{st3}
    $\sqsupset (X, \rho)$ -- м.п., $K$ -- компактно в  $X$. Тогда  $K$ замкнуто и ограничено (в  $X$)
\end{statement}
\begin{proof}
    $\sqsupset a\in X\quad \left\{ B_n(a) \right\} _{n\in \N } \implies  K\subseteq X\subseteq \bigcup\limits_{n\in \N } B_n(a)$

    Если $K$ компактно, то  $\exists n_1, \ldots, n_m: K\subseteq \bigcup\limits_{k=1}&m B_{n_k}(a) = B_N(a), N = \max\{n_1, \ldots, n_m\}$, т.е. $K$ ограничено

     $K$ -- замкнуто?  $\iff X\setminus K$ -- открыто

     $\sphericalangle \forall p\in X\setminus K$

     $\forall q\in K\quad 0<\frac{\rho(q,p)}{2} = r_q \implies B_{r_q}(q)\cap B_{r_q}(p) = \O $ 

     $\rho(x,y) >\rho(p,q) - \rho(p,y) - \rho(q,x) >2r_q-r_q-r_q=0$

     $\left\{ B_{r_q}(q) \right\} _{q\in K}$ -- открытое покрытие $K \implies \left( K \text{ -- компакт } \right)  \exists q_1, \ldots, q_N: K\subseteq \bigcup\limits_{j=1}^N B_{r_{q_j}}(q_j), r = \min\left\{ r_{q_1}, \ldots, r_{q_N} \right\}  $ 

     $B_r(p)\subseteq B_{r_{q_j}} (p)\qquad B_r(p)\cap B_{r_{q_j}}(q_j) \neq \O \quad \forall j=1:N \implies  B_{r}(p)\cap K\neq \O  \implies  B_r(p)\subseteq X\setminus K$

     Т.о. $X\setminus K$ -- открыто
\end{proof}

Аксиома о вложенных промежутках справедлива и для ``обобщённых замкнутых промежутков''  $Q = [a_1, b_1]\times \ldots\times [a_n, b_n] \subseteq \R^n$ -- $\overline{\text{куб}}$

$Q^{\left( j \right) } = \prod_{k=1}^{n} [a_k^{\left( j \right) }, b_k^{\left( j \right)}] \quad \forall j\in \N $ и $\forall k=1:n\quad \forall j\in \N \quad \left[ a_k^{\left( j+1 \right) }, b_k^{\left( j+1 \right) } \right] \subseteq\left[ a_k^{\left( j\right) }, b_k^{\left( j \right) } \right] $

По аксиоме о вложенных промежутках $ \forall k=1:n\quad \exists c_k \subseteq \bigcap\limits_{j=1}^{\infty } [a_k^{\left( j \right) }, b_k^{\left( j \right) }] \implies  c = \left( c_1, \ldots, c_n \right) \in \prod_{k=1}^{n} [a_k^{\left( j \right) }, b_k^{\left( j \right) } ] \forall j\in \N $

\begin{theorem}
    [Гейне-Бореля] В $ \R^n$ любой замкнутых куб компактен 

    $Q = [a_1, a_1+\delta] \times [a_2, a_2+\delta] \times \ldots\times [a_n, a_n+\delta]\quad \delta$ -- длина ребра
\end{theorem}
\begin{proof}
\begin{figure}[ht]
    \centering
    \incfig{cube}
    \caption{cube}
    \label{fig:cube}
\end{figure}

От противного $\sqsupset \left\{ O_i \right\} _{i\in I}$ -- открытое покрытие куба $Q$, из которого нельзя извлечь конечного подпокрытия

 $Q_1 = Q;$ делением рёбер пополам представим $Q$  виде объединения  $2^N$ кубов со стороной  $\frac{\delta}{2}$ 

 Хотя бы один из них -- ``плохой'' (т.е. не имеет конечного подпокрытия в $\{O_i\}$)ю Назовём  $O_2, \ldots$

 В результате $\left\{ Q_j \right\} $ -- последовательность кубов

 $x, y\in \ov O\quad \ov O$ -- куб со стороной  $\Delta \implies  \|x-y\| \leqslant \Delta \cdot  \sqrt{n} \\ \left( \|x-y\| = \sqrt{\sum_{k=1}^{n} (x_k-y_k)^2} = \sqrt{\sum_{k=1}^{n} \Delta^2} = \sqrt{n} \Delta   \right) $

 По предыдущему утверждению $\exists c\in \bigcap\limits_{j=1}^{\infty } Q_j \quad c\in \bigcup\limits_{i \in  I} O_i \implies \exists i_c: c\in O_{i_c}$

 $\left\{ O_i \right\} $ -- открыто $\implies  \exists r>0: B_r(c)\subset O_{i_c}$/

 Т.к. $\frac{\delta}{2^{j-1}} \to 0$, то $\exists J: \forall j\geqslant J\quad \frac{\delta \sqrt{n} }{2^{j-1}}<r$ 

 Тогда $y\in Q_j$, где $j \geqslant J$

 $\implies \left| y-c \right| \leqslant \frac{\delta}{2^{j-1}}\cdot \sqrt{n}<r\quad \forall j\geqslant J\implies  Q_j\subseteq B_r(c)\subseteq O_{i_c}\quad \forall j\geqslant J $. $Q_j$ -- ``плохой'' !!!
\end{proof}
\begin{theorem}
    [критерий компактности в $\R^n$]

    $\sqsupset K\subseteq \R^n$. Тогда следующие утверждения равносильны:
    \begin{enumerate}
        I: $K$  -- замкнуто и ограничено\\
        II: $K$ -- компактно\\
        III: $K$ -- секвенциально компактно \\(из любой последовательности в $K$ можно извлечь сходящуюся в  $K$ подпоследовательность )\\
    \end{enumerate}
\end{theorem}

\begin{definition}
    $\sqsupset \left\{ x_k \right\} _{k=1}^{\infty }$ -- последовательность в топологическом пространстве $(X, \Omega)$.

    Если  $\left\{ k_j \right\} _{j=1}^{\infty }$ -- строго возрастающая последовательность натуральных чисел, то $\left\{ x_{k_j} \right\} _{j=1}^{\infty }$ называется \underline{подпоследовательностью} исходной последовательности 
\end{definition}

\begin{lemma}
    Если $\left\{ k_j \right\} _{j\in \N }\subseteq \N $ и строга возрастает, то $\forall j\in \N \quad k_j\geqslant j$
\end{lemma}

\begin{definition}
    $\sqsupset \left\{ y_k \right\} _{k=1}^{\infty }\subseteq \R$ -- последовательность. 

    $\left\{ y_k \right\} $ называется \underline{возрастающей}, если $\forall k\in \N \quad y_{k+1}\geqslant y_k \left( \forall k, m, m\geqslant k \text{, то } y_m\geqslant y_k \right) $ 

    $\left\{ y_k \right\} $ -- строго возрастает $\iff  \forall k, m\in \N m>k\implies y_m>y_k$
\end{definition}
\begin{definition}
    $\sqsupset \left\{ x_k \right\} _{k=1}^{\infty }$ -- последовательность в т.п. $(X, \Omega), x\in X\quad \\ x = \lim_{k \to \infty} x_k \iff \forall $ открытой окрестности $U$ точки  $x\quad \exists N\in \R\forall k\in \N , k\geqslant N\quad x_n\in U$

    $\iff  \forall $ открытой окрестности $U$ точки $x\quad \exists $ окрестность $V(+\infty ): \forall k\in V(+\infty )\cap \N \quad x_k\in V$
\end{definition}
\begin{example}
    $x_k = (-1)^k, k\in \N $

    $x_{2k} \equiv 1 \to 1$

    $x_{2k+1}\equiv = -1 \to -1$
\end{example}
\begin{statement}
    Если $\left\{ x_{n}  \right\} $ сходится (в $(X, \Omega)$), то любая её подпоследовательность сходится, причём к тому же пределу.
\end{statement}
\begin{proof}
    $\sphericalangle \forall $ подпоследовательность $\left\{ x_{k_j} \right\} _{j=1}^{\infty }\quad \sqsupset x = \lim_{k \to \infty} x_k\quad \sqsupset U, N, x$ -- из определения предела.

    Тогда $\forall j\geqslant N \implies  $ по лемме $ k_j\geqslant j\geqslant N \implies  x_{k_j}\in U \implies  x_{k_J} \to x, j\to +\infty $
\end{proof}

\begin{statement}
    
\end{statement}
$\sqsupset \left\{ x_k \right\} $ -- последовательность в т.п. $(X, \Omega)$

$\left\{ x_{k_j} \right\} _{j=1}^{\infty }\quad \left\{ x_{l_i} \right\} _{i=1}^{\infty }$ -- подпоследовательности

$\left\{ x_k \right\} _{\N } = \left\{ x_{k_j} \right\} _{j\in \N } \cup \left\{ x_{l_i} \right\} _{i\in \N }$

$\sqsupset \exists \lim_{j \to \infty} x_{k_j} = \lim_{i \to \infty} x_{l_i} = x$ Тогда $\exists \lim_{k \to \infty} x_k$ и $\lim_{k \to \infty} x_k = x$
\end{statement}
\begin{proof}
    Т.к. $x_{k_j} \to  x$, то $\forall $ окрестности $U \exists J = J(U): \forall j>J\quad x_{k_j}\in U$

    Т.к. $x_{l_i} \to x$, то $\forall $ окрестности $U\quad \exists I = I(U): \forall i>I\quad x_{l_i}\in U$
    $N = \max\{k_J, l_I\}, \forall n>N\quad x_n = \begin{cases}
        x_{k_{j}} \implies  j>J\\
        x_{l_i} \implies  i>I
    \end{cases} \implies x_n\in U$
\end{proof}

\begin{example}
    $\Q = \left\{ x_k \right\} _{k=1}^{\infty }$ (при некоторой нумерации)

    \begin{problem}
        $\forall x\in \ov {\R} \exists $ подпоследовательность $\{x_{k_j}\}_{j=1}^{\infty }: x_{k_j} \to x, j\to \infty $
    \end{problem}
\end{example}

\begin{proof}
    [Доказательство критерия компактности в пространстве $\R^n$]

    \begin{itemize}
        \item []
        \item [II $\implies $ I] Утверждение \ref{st3}
        \item [I $\implies $ II] $K$ -- замкнуто и ограничено  $\implies K\subseteq B\subseteq Q\quad B$ -- шар, $Q$ -- куб  $\implies $ по \ref{st1} $K$ -- компакт
        \item [III $\implies $ I] $K$ -- ограничено? от противного. Если  $K$ не ограничено   $\forall n\in \N  \exists x_k\in K:\quad \|x_k\|>k\quad \|x_k\|\to +\infty \quad 

            x_k\to \infty \implies  \forall $ подпоследовательность $\left\{ x_{k_i} \right\} \to \infty \quad i\to \infty $

            $K$ -- замкнуто,  $\forall p\in \p K \implies p\in K$ От противного $\sqsupset p\in \p K, p\not\in K \implies \exists $ последовательность $\left\{ x_k \right\} _{k=1}^{\infty }\subset K: x_k\to p, k\to +\infty \implies \forall $ подпоследовательность $\{x_{k_j}\} \to \not\in K, j\to \infty  !!!(III)$. Т.о. $K$ -- замкнуто
        \item [III $\impliedby $II] $\sphericalangle \forall \left\{ x_k \right\} _{k=1}^{\infty } \times K$

            Если $\left\{ x_k \right\} _{k\in \N }$ -- конечно, то $\exists \left\{ k_j \right\} _{j=1}^{\infty }\subseteq \N $ -- возрастающая $\quad x_{k_j} = const \implies  x_{k_j} \to x_{k_1}$

            Если $\left\{ x_k \right\} _{k\in \N }$, тогда существует предельная точка в $ \R^n$ для F $\implies F$. Если нет, то $F$ -- замкнуто в $\R^n \implies F$ -- компакт ($F\subseteq K$)

            $\forall x\in F\quad x$ -- изолированная точка $\exists \delta_x>0: B_{\delta_x}(x)\cap F = \{x\}$

            $\cup B_{\delta_x}(x)\supseteq F $, но нельзя извлечь конечное подпокрытие.

            Т.о. $\exists p\in \p F \implies  \forall \varepsilon>0\quad \overset{\circ }B_{\varepsilon}(p)\cap F\neq \O $

            $\varepsilon=1\quad \sqsupset x_{k_1}\in B_1(p)\cap F \implies k_2\in \overset{\circ }B_{y_2}(p)\cap F\setminus \{x_1, \ldots, x_{k_1}\}\neq \O \quad k_2>k_1$

            $\ldots$

            $\exists x_{k_{j+1}}\in \overset{\circ}B_{2j}(o)\cap F\setminus \{x_1, \ldots, x_{k_j}\}\neq \O $

            $\left\{ x_{k_j} \right\} \subseteq F, k_i\uparrow\quad \|x_{k_j} - p\|\leqslant \frac{1}{2j} \to 0, j\to \infty $
    \end{itemize}
\end{proof}
\begin{corollary}
    [принцип выбора Коши-Больцано] Из любой ограниченной последовательности в $\R^n$ можно извлечь сходящуюся подпоследовательность. (последовательность содержится в некотором компакте)
\end{corollary}

\begin{definition}
    $\sqsupset (X, \rho)$ -- м.п. $\quad \left\{ x_k \right\} _{k=1}^{\infty }\subseteq X$. $\left\{ x_k \right\} _k$ называется \underline{сходящейся в себе} (она же последовательность Коши, она же фундаментальная последовательность), если $\forall \varepsilon>0\exists N\in \R: \forall n, m\in \N , n, m \geqslant N\quad \rho(x_{n} , x_{m} )<\varepsilon$
\end{definition}

\begin{statement}
    Если $\left\{ x_k \right\} _{k=1}^{\infty } \subseteq (X, \rho)$, то она сходится в себе
\end{statement}
\begin{proof}
    $\sqsupset \left\{ x_k \right\} $ сходится $\implies \exists x\in X: \forall \varepsilon>0\quad \exists N\in \R\quad \forall n\in \N , n\geqslant \N \quad \rho(x_{n} , x)<\frac{\varepsilon}{2}$. Тогда $\forall n, m\geqslant \N , n, m\in \N \quad \eho(x_{n} , x_{m} )\leqslant \rho(x_{n} , x) + \rho(x_{m} , x)<\frac{\varepsilon}{2}\cdot 2 = \varpesilon \implies \left\{ x_k \right\} _{k=1}^{\infty }$ сходится в себе
\end{proof}
\end{document}

