\documentclass{book}
%nerd stuff here
\pdfminorversion=7
\pdfsuppresswarningpagegroup=1
% Languages support
\usepackage[utf8]{inputenc}
\usepackage[T2A]{fontenc}
\usepackage[english,russian]{babel}
% Some fancy symbols
\usepackage{textcomp}
\usepackage{stmaryrd}
% Math packages
\usepackage{amsmath, amssymb, amsthm, amsfonts, mathrsfs, dsfont, mathtools}
\usepackage[bb=boondox]{mathalfa}
\usepackage{cancel}
% Bold math
\usepackage{bm}
% Resizing
%\usepackage[left=2cm,right=2cm,top=2cm,bottom=2cm]{geometry}
% Optional font for not math-based subjects
%\usepackage{cmbright}

\author{Коченюк Анатолий}
\title{Дискретная математика }

\usepackage{url}
% Fancier tables and lists
\usepackage{booktabs}
\usepackage{enumitem}
% Don't indent paragraphs, leave some space between them
\usepackage{parskip}
% Hide page number when page is empty
\usepackage{emptypage}
\usepackage{subcaption}
\usepackage{multicol}
\usepackage{xcolor}
% Some shortcuts
\newcommand\N{\ensuremath{\mathbb{N}}}
\newcommand\R{\ensuremath{\mathbb{R}}}
\newcommand\Z{\ensuremath{\mathbb{Z}}}
\renewcommand\O{\ensuremath{\emptyset}}
\newcommand\Q{\ensuremath{\mathbb{Q}}}
\renewcommand\C{\ensuremath{\mathbb{C}}}
\newcommand{\p}[1]{#1^{\prime}}
\newcommand{\pp}[1]{#1^{\prime\prime}}
% Easily typeset systems of equations (French package) [like cases, but it aligns everything]
\usepackage{systeme}
\usepackage{lipsum}
% limits are put below (optional for int)
\let\svlim\lim\def\lim{\svlim\limits}
%\let\svlim\int\def\int{\svlim\limits}
% Command for short corrections
% Usage: 1+1=\correct{3}{2}
\definecolor{correct}{HTML}{009900}
\newcommand\correct[2]{\ensuremath{\:}{\color{red}{#1}}\ensuremath{\to }{\color{correct}{#2}}\ensuremath{\:}}
\newcommand\green[1]{{\color{correct}{#1}}}
% Hide parts
\newcommand\hide[1]{}
% si unitx
\usepackage{siunitx}
\sisetup{locale = FR}
% Environments
% For box around Definition, Theorem, \ldots
\usepackage{mdframed}
\mdfsetup{skipabove=1em,skipbelow=0em}
\theoremstyle{definition}
\newmdtheoremenv[nobreak=true]{definition}{Определение}
\newmdtheoremenv[nobreak=true]{theorem}{Теорема}
\newmdtheoremenv[nobreak=true]{lemma}{Лемма}
\newmdtheoremenv[nobreak=true]{problem}{Задача}
\newmdtheoremenv[nobreak=true]{property}{Свойство}
\newmdtheoremenv[nobreak=true]{statement}{Утверждение}
\newmdtheoremenv[nobreak=true]{corollary}{Следствие}
\newtheorem*{note}{Замечание}
\newtheorem*{example}{Пример}
\renewcommand\qedsymbol{$\blacksquare$}
% Fix some spacing
% http://tex.stackexchange.com/questions/22119/how-can-i-change-the-spacing-before-theorems-with-amsthm
\makeatletter
\def\thm@space@setup{%
  \thm@preskip=\parskip \thm@postskip=0pt
}
\usepackage{xifthen}
\def\testdateparts#1{\dateparts#1\relax}
\def\dateparts#1 #2 #3 #4 #5\relax{
    \marginpar{\small\textsf{\mbox{#1 #2 #3 #5}}}
}

\def\@lecture{}%
\newcommand{\lecture}[3]{
    \ifthenelse{\isempty{#3}}{%
        \def\@lecture{Lecture #1}%
    }{%
        \def\@lecture{Lecture #1: #3}%
    }%
    \subsection*{\@lecture}
    \marginpar{\small\textsf{\mbox{#2}}}
}
% Todonotes and inline notes in fancy boxes
\usepackage{todonotes}
\usepackage{tcolorbox}

% Make boxes breakable
\tcbuselibrary{breakable}
\newenvironment{correction}{\begin{tcolorbox}[
    arc=0mm,
    colback=white,
    colframe=green!60!black,
    title=Correction,
    fonttitle=\sffamily,
    breakable
]}{\end{tcolorbox}}
% These are the fancy headers
\usepackage{fancyhdr}
\pagestyle{fancy}

% LE: left even
% RO: right odd
% CE, CO: center even, center odd
% My name for when I print my lecture notes to use for an open book exam.
% \fancyhead[LE,RO]{Gilles Castel}

\fancyhead[RO,LE]{\@lecture} % Right odd,  Left even
\fancyhead[RE,LO]{}          % Right even, Left odd

\fancyfoot[RO,LE]{\thepage}  % Right odd,something additional 1  Left even
\fancyfoot[RE,LO]{}          % Right even, Left odd
\fancyfoot[C]{\leftmark}     % Center

\usepackage{import}
\usepackage{xifthen}
\usepackage{pdfpages}
\usepackage{transparent}
\newcommand{\incfig}[1]{%
    \def\svgwidth{\columnwidth}
    \import{./figures/}{#1.pdf_tex}
}
\usepackage{tikz}
\begin{document}
    \maketitle
    
    \section{Введение}
    Связаться:
    \begin{itemize}
        \item stankev@gmail.com Собирать культуру общения: указывать Фамилию, Имя
        \item Телеграм @andrewzta  (для немедленного ответа. Если нет, оно утонет).
        \item +79219034426 (для катастрофических ситуаций, ожидается, что звонить никто не будет) (ни в коем случае не писать смс)
    \end{itemize}

    Обращаться можно по методическим вопросам. Если проблема группы -- пишет староста.

    Не писать по учебно-методическим проблемам (общежитие, медосмотр, армия ..) для этого есть зам. декана Харченко (легко найти контакты в ису)

    Про отчётность будет на первой практике.

    Лекции есть в ютубе andrewzta
    \chapter{1 курс}
    \section{Фундамент}

        Множество -- неопределяемое понятие. Множество состоит из элементов. $a\in A$ а-маленькое принадлежит множеству А-большое

        $A = \{2, 3, 9\}$

        $A = \{n \mid n\text{ чётно}, n \in \N \}$ -- фильтр

        $A, B:$
        \begin{itemize}
            \item $A\cup B = \{a \mid a\in A \text{ или } a\in B\}$
            \item $A\cap B = \{a \mid a\in A \text{ и } a\in B\}$
            \item $A\setminus B = \{a | a\in A \text{ и } a\not\in B\}$
            \item $\overline{A} = \{a | a\not\in A\}$??? $U$ -- универсум

                $\overline{A} = U\setminus A$
            \item[] $A\setminus B = A\cap \overline{B}$
            \item $A \triangle B = A\oplus B = (A\cup  B)\setminus (A\cap B)$
        \end{itemize}



        \begin{note}
            Если множество -- любой набор чего-угодно возникает парадокс Рассела

            $A = \{a|a\text{ -- множество, } a\not\in a\}$
            
            Вопрос лежит ли в себе $A$? 
        \end{note}

        \begin{definition}
            [Пара]

            $A, B$ -- множества. Мы можем рассмотреть множество пар, где первый элемент из $A$, а второй из $B$

            $A \times B = \{(a,b) | a\in A, b\in B\}$

            $A\times A = A^2$
        \end{definition}

        $(A\times B)\times C = \{(x,y)|x\in A\times B, y\in C\} = \{((a,b),y) |a\in A, b\in B, y\in C\}$ 
   
        $A\times (B\times C) = \{(a,(y,z))| a\in A, y\in B, z\in C\}$
        
        $A\times B\times C = \{(a,b,c)|a\in A, b\in B, c\in C\}$

        Для простоты, здесь и далее эта операция будет считаться ассоциативной и первые две строчки будут давать то же, что третья -- множество троек.

        $A\times A\times A = A^3$
        $A^n = \begin{cases}
            A&,n=1\\
            A\times A^{n-1}&,n>1\\
        \end{cases}$

        $A^0 = \{[]\} = \{\varepsilon\}$ -- пустая последовательность.         
        \begin{example}
            $A = {2, 3, 9}$
->    
            $A\times A =\{(2,2),(2,3),(2,9),(3,2),(3,3),\ldots\}$ 
        \end{example}

        \begin{note}
            У множества есть элемента и для любого элемента из универсума, он либо входит (1 раз) либо не входит.
        \end{note}

        \begin{definition}
            Функция -- отображение, которое каждому элементу из одного множества ставит в соответветвие единственный элемент из другого множества

            $f:A\to B$ 

            График $\{(x,f(x))\}$.

            Формально будем отождествлять функцию и её график.

            $f\subset A\times B\quad \forall a\in A \exists ! b\in B\quad (a,b)\in f$
        \end{definition}

        \begin{note}
            Не путайте принадлежность и включение

            $a\in A$

            $A, B, \forall a$ (если $a\in A$, то $a\in B$) $A\subset B$

            $D_4 = \{n|n\text{ кратно } 4\}$

            $E = \{n|n\text{ чётно}\}$

            $D_4\subset E$

            $\{2, 3, 9\} \subset  \{2, 3, 4, \ldots, 9\}$

            $A\subset A$

            $\O \subset A$

            $A\subset U$
        \end{note}

        \begin{note}
            Необязательно все $b$ попадают в график.
            
            $sqr:\N \to \N $ -- только квадраты чисел
        \end{note}

        \begin{definition}
            $\forall b\in B\exists a\in A: b = f(a)$ -- сюръекция
        \end{definition}

        \begin{definition}
            $\forall a\in A \forall b\in B\quad a\neq b \implies f(a)\neq f(b)$
        \end{definition}

        \begin{note}
            Принцип Дирихле -- нет инъекции из большего в меньшее множества. Если кроликов больше, чем клеток, то какому-то кролику не хватит клетки
        \end{note}

        \begin{definition}
            Если $f$ -- инъекция и сюръекция, то $f$ -- называется биекцией

            Если между двумя конечными множествами есть биекция, то у низ равное количество элементов.
        \end{definition}

        \begin{definition}
            Два множества называется равномощными, если между ними есть дикция
        \end{definition}

        $B^A$ -- множество функций из $A$ в $B$

        $|A| = a, |B| = b\quad |A\times B| = a\cdot b\quad |B^A| = b^a$

        $|A^{\O }| = 1$ эфемерная функция, которой ничего не передать

        $\O ^A = \O , A \neq \O $

        $\O ^{\O } = 1$
       
        \begin{definition}
            $R\subset A\times B$ -- отношение (бинарное)

        \end{definition}
        \begin{example}
            $A=B=\N \quad R = \{(a,b)|a<b\}\quad R= <$

            $a\vdots b\quad 6\vdots 2\quad 6\not\vdots 5$

            $A$ = люди, $B$ = собаки, $R = \{(a,b)|a\text{ -- хозяин} b\}$
        \end{example}

        Рассмотрим 5 классов отношение на квадрате множества:
        \begin{enumerate}
            \item рефлексивные $\forall a \quad aRa$

                $RC(R)$ -- рефлексивное замыкание, включаем все пары $(a,a)$
            \item антирефлексивные $\forall a \quad \neg aRa$
            \item симметричные $aRb \implies bRa$
            \item антисимметричные $aRb, a\neq b \implies \neg bRa$

                или $aRb$ и $bRa \implies a=b$
            \item транзитивность $aRb, bRc \implies  aRc$
        \end{enumerate}
        
        \begin{definition}
            1+3+5 -- рефлексивные, симметричные и тразитивные -- называются отношениями эквивалентности. 
        \end{definition}
        \begin{theorem}
            $R$ -- отношение эквивалентности на $X$, то элементы $X$ можно разбить на классы эквивалентности так, что:

            $a$ и $b$  в одном классе $\implies  aRb$
            $a$ и $b$  в разных  классах $\implies \neg aRb$

            множество таких классов обозначается $X / R$

            $N / \equiv_3 = $
            \begin{align*}
                \{\{1,4,7,10,\ldots)\\
                        \{2,5,8,11,\ldots)\}\\
                        \{3,6,9,12,\ldots)\}\}\\
            .\end{align*}   
        \end{theorem}
        \begin{note}
            Отношение равномощности -- отношение эквивалентности.

            Классы эквивалентности -- порядки. Для конечного случая обозначаются числами
        \end{note}

        \begin{definition}
            1+4+5 -- рефлексивные, антисимметричные и транзитивные -- частичные порядки

            Множество, на котором введён частичный порядок, то оно называется частично упорядоченным. (ч.у.м -- частично упорядоченное множество, poset -- partially organised set)
        \end{definition}
        
        $R\subset X\times X$

   $X, Y, Z\quad R:X\times Y\quad S:Y\times Z$

   \begin{definition}
       Композиция отношений: 

       $T = R\circ S\quad xTy \iff \exists z: xRz \text{ и } zSy$

       т.е. есть $z$, через который можно пройти, чтобы попасть в $y$ из $x$
   \end{definition}

    \begin{note}
        $R\subseteq X\times X\quad S\subseteq X\times X$

        $R\circ S \subseteq X\times X$
        
        $R\circ R\subseteq X\times X$ -- пройти два раза по стрелкам

        $R^3 = R\circ R^2 = R^2\circ R$ -- пути длины ровно 3

        $S\circ T\circ U$ -- идём по стрлке из $S$ в $T$, а потом в $U$
    \end{note}

    \begin{definition}
        Транзитивное замыкание.

        $R^+ = \bigcup\limits_{k=1}^{\infty } R^k$

        $R^0 = \{(x,x)|x\in X\}$ -- они не включаются по дефолту в $R^+$

        $R^* = \bigcup\limits_{k=0} ^{\infty } R^k = R^+ \cup R^0$ -- если между двумя вершинами существует какой-либо путь
    \end{definition}
    \begin{note}
        Транзитивное замыкание -- транзитивно

        Пусть $xR^+y \implies x R^i y$

        Пусть $yR^+z \implies yR^j z$

        $\implies x(R^i\circ R^j)z \implies  xR^k z$
    \end{note}
    \begin{note}
        $\forall T:T$ -- транзитивно. $T\subset R \implies T^+ \subset R$
    \end{note}
    \begin{proof}
        По индукции:

        База: $R^1 \subset T$ -- дано

        Переход: $R^i\subset T \implies  R^{i+1}\subset T$

        $xR^{i+1}y \implies x(R\circ R^i)y\implies \exists z: xRz\&zR^iy \implies xTz\&zTy \implies xTy$ (по транзитивности $T$)
    \end{proof}

    \section{Булевы функции}

    $\O $ -- пустое множество. С функциями из/в него всё достаточно грустно.

    $\{unit\}$

    $void$ -- ничего, константная функция

    $\mathbb{B} = \{0,1\}$

    $f:A_1\times A_2\times \ldots\times A_n \to B$  -- функция от нескольких аргументов. Из одного, но декартового произведения

    Булева функция: $f:\mathbb{B}^n\to B$

    $n=0$ -- ноль аргументов $\mathbb{B}^0 = \{[]\}$

    $\mathbb{0}, \mathds{1}$
     
    $n=1\quad$
    \begin{table}[htpb]
        \centering
        \caption{n=1}
        \label{tab:n1}
        \begin{tabular}{c|cccc}
            x&$\mathbb{0}$&id&$\neg$&$\mathds{1}$\\\hline
        0 &0  &0  &1  &1\\
        1&0&1&0&1 \\
        \end{tabular}
    \end{table}
   \begin{note}
       Подобные таблицы называются таблицами истинности функций
   \end{note} 
    $n=2$
    \begin{table}[htpb]
        \centering
        \caption{n=2}
        \label{tab:n2}
        \begin{tabular}{cc|cccccccccccccccc}
            x&y  & $\mathbb{0}$ & $\land$   &$\not\to $  & $P_1$ & $\not\leftarrow$ & $P_2$ &$\oplus$ & $\vee$ & $\downarrow$ & = & $\neg P_2$ & $\leftarrow$ & $\neg P_1$ & $\to $ & $\uparrow$ & $\mathds{1}$ \\\hline
         0&  0&  0& 0 & 0 & 0 & 0 & 0 & 0 & 0 & 1 & 1 & 1 & 1 & 1 &1  &1  &1 \\
         0&  1&  0&  0& 0 & 0 &1  & 1 &  1&  1& 0 & 0 & 0 & 0 &1  & 1 & 1 & 1\\
         1&  0&  0&  0&  1&  1& 0 &0  & 1 & 1 & 0 & 0 & 1 & 1 & 0 & 0 &1  &1 \\
         1&  1&  0&  1&  0&  1& 0 & 1 & 0 & 1 & 0 & 1 & 0 & 1 & 0 & 1 & 0 & 1\\
        \end{tabular}
    \end{table}

    С помощью стрелки Пирса ($\downarrow$) и штриха Шефера ($\uparrow$) можно выразить любую другую: $\neg x  = x \downarrow x$

    \section{Задания булевых функций}

    Самый простой способ -- таблица истинности

    $\oplus_n$ -- $2^n$ значений. глупо их все отдельно описывать

    \begin{enumerate}
        \item Задание функции формулой.

            Определим базисные функции, систему связок

            например: $\land, \vee, \neg, \oplus$

            $x_1 \oplus x_2 \oplus x_3 \ldots $

            $\{f_1, f_2, .., f_n\}$ -- базисные.

            строка -- формула. $f_i(x_1, \ldots, x_k)$ -- формула

            \begin{definition}
                Дерево разбора формулы. Если у функции арность -- $k$, то у ноды будет ровно $k$ сыновей
            \end{definition}
    \end{enumerate}

    $\overline{F}$ --  функции, которые записываются формулами, используя $F$ (замыкание $F$)

    \begin{theorem}
        [Теорема о стандартном базисе] $\overline{\{\land, \vee, \neg\}} = \mathbb{B}$
    \end{theorem}
    \begin{proof}
        Рассмотрим таблицу истинности функции $f$ Она принимает $n$ аргументов и в ней $2^n$ строк

        Пусть $f\neq \mathbb{0}$. Рассмотрим строчки, в которых единицы.

        По аргументам запишем с не -- аргументы, которые 0, и без не -- те, которые 1

        $\neg x_1 \land \neg x_2 \land x_3 \land \neg x_4 \land x^5$ -- 1 на ровно одном наборе элементов. А теперь возьмём "или" по всем строкам, в которых $1$ 

        Одна такая строка называется термом. 

        Такая форма называется совершенной дизъюнктивной нормальная формой
    \end{proof}
    \begin{lemma}
        Любая функция, кроме тождественного 0  -- есть СДНФ

        $x\vee \neg x$ -- тождественный ноль
    \end{lemma}
 \end{document}
